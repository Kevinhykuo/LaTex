\documentclass[12pt, english]{book}
\usepackage[letterpaper, portrait, margin=2.5cm]{geometry} %Set paper type, format, margin
\usepackage[svgnames]{xcolor}
\usepackage{%abstract, 
	amsmath, amssymb, array, babel, booktabs, caption, censor, fancyhdr, flafter, float, framed, gensymb, hyperref, inputenc, indentfirst, mathtools, MnSymbol, multicol, multirow, longtable, lscape, ltablex, polynom, pgfplots, physics, placeins, rotating, scalerel, setspace, stackengine, tabularx, threeparttable, titlesec, titling, wasysym, wrapfig}
\usepackage{cleveref}
\usetikzlibrary{calc,angles,positioning,intersections,quotes,decorations.markings,backgrounds,patterns}
\usepgfplotslibrary{groupplots}
\usepackage{amsthm} %Do not use amsthm with ntheorem unlees ntheorem is using the amsthm option
\usepackage{thmtools}
\pgfplotsset{compat=1.8}
\usepgfplotslibrary{external}
\tikzexternalize[prefix=tikz/] 

\setlength{\parindent}{0em} %Set indent on paragraph
\setlength{\parskip}{0.5em} %Set spaces between paragraphs

\newcommand\showdiv[1]{\overline{\smash{\hstretch{.5}{)}\mkern-3.2mu\hstretch{.5}{)}}#1}}
\newcommand\ph[1]{\textcolor{white}{#1}}
\setstackgap{S}{1.5pt}

% Swap the definition of \abs* and \norm*, so that \abs
% and \norm resizes the size of the brackets, and the 
% starred version does not.


\declaretheoremstyle[
	spaceabove=6pt, 		
	spacebelow=6pt,
	%	headfont=\normalfont\bfseries,
	notefont=\bfseries, notebraces={}{},
	bodyfont=\slshape,
	postheadspace=1em,
	headpunct=,
	headformat={\NAME~\NUMBER:\NOTE \hfill \smallskip\linebreak},%
	]{axiomstyle}

\declaretheoremstyle[
	spaceabove=6pt, 		
	spacebelow=6pt,
	headfont=\bfseries, %\color{blue},
	notefont=\bfseries, notebraces={}{},
	bodyfont=\slshape \color{DarkBlue},
	postheadspace=1em,
	headpunct=,
	headformat={\NAME~\NUMBER:\NOTE \hfill \smallskip\linebreak},%
	]{theoremstyle}
	
\declaretheoremstyle[
	spaceabove=6pt, 		
	spacebelow=6pt,
	headfont=\bfseries, %\color{Blue}%
	notefont=\bfseries,%
	notebraces={}{},%
	headpunct=,%
	bodyfont=\itshape\color{DarkGreen},%
	headformat={\NAME~\NUMBER:\NOTE \hfill \smallskip\linebreak},%
	%	preheadhook=\begin{leftbar},%
	%	postfoothook=\end{leftbar},%
	]{minorstyle}
	
\declaretheoremstyle[
	spaceabove=6pt, 		
	spacebelow=6pt,
	headfont=\bfseries,%\color{Blue},%
	notefont=\bfseries,%
	notebraces={}{},%
	headpunct=,%
	bodyfont=\slshape \color{OrangeRed},%
	headformat={\NAME~\NUMBER:\NOTE \hfill \smallskip\linebreak},%
%	preheadhook=\begin{leftbar},%
%	postfoothook=\end{leftbar},%
	]{defstyle}
	
\declaretheoremstyle[
	spaceabove=6pt, 		
	spacebelow=6pt,
	%	headfont=\normalfont\bfseries,
	bodyfont=\slshape,
%	postheadspace=1em,
	headpunct=,
%	headformat={\NAME~\NUMBER:\NOTE \hfill \smallskip\linebreak},%
	]{examplestyle}

\declaretheorem{axiom}[
	numberwithin=chapter, style=axiomstyle, parent=chapter,
	]
	
\declaretheorem{theorem, conjecture}[
	numberwithin=section, style=theoremstyle, parent=section,
	]
	
\declaretheorem{lemma, proposition, corollary}[
	style=minorstyle,
	parent=theorem,
%	numberwithin=section
	]

\declaretheorem{definition}[parent=section, style=defstyle]

\declaretheorem{remark, observation, question}[numbered=no]

\declaretheorem{example}[numberwithin=section, style=examplestyle]


% Indenting the Proof enviroment
\makeatletter
\renewenvironment{proof}[1][\proofname]{\par
	\pushQED{\qed}%
	\normalfont \topsep6\p@\@plus6\p@\relax
	\list{}{%
		\settowidth{\leftmargin}{\itshape\proofname:\hskip\labelsep}%
		\setlength{\labelwidth}{0pt}%
		\setlength{\itemindent}{-\leftmargin}%
	}%
	\item[\hskip\labelsep\itshape#1\@addpunct{:}]\ignorespaces
}{%
	\popQED\endlist\@endpefalse
}
\makeatother

% Letters after part
\makeatletter
\renewcommand{\@endpart}{\vfil\newpage}
\makeatother
\newenvironment{partintro}
{\vspace*{\fill}
	\section*{\centering Resources used in part \thepart}
	\begin{quotation}}
	{\end{quotation}\vspace*{\fill}\newpage}
\newcommand{\nopartintro}{%
	\vspace*{\fill}
	\thispagestyle{empty}
	\newpage
}


\begin{document}
	\title{The Book of Physics (Notes)}
	\author{Kevin Kuo}
	
	\pagestyle{fancy}
	\fancyhead{} % clear all header fields
	\fancyhead[LO]{  }
	\fancyhead[CO]{  }
	\fancyhead[RO]{  }
	\renewcommand{\headrulewidth}{0pt}
	
	\fancyfoot{} % clear all footer fields
	\fancyfoot[LO]{}
	\fancyfoot[CO]{\thepage}
	\fancyfoot[RO]{}
	\renewcommand{\footrulewidth}{0pt}
	\vspace{0cm}	
	
	\frontmatter
	
	\maketitle
	
	\newpage
	\section*{Forward and Disclaimer}
	
	
	\section*{Symbols}
	

	\newpage
	\section*{Book Constitution}
	\subsection*{Intents and Purpose}
	
	
	\subsection*{Layout and Organization}
	

	
	\section*{}
	\tableofcontents
	
	\mainmatter
	
	\part{Classical Mechanics} \label{Classical Mechanics Part}
	\begin{partintro}
		\noindent Primary:
		\begin{itemize}
			\item[1.] Brown and Churchill - Complex Variables and Applications
		\end{itemize}
		Supplement: 
		\begin{itemize}
			\item[1.] A. David Wunsch - Complex Variables with Applications
		\end{itemize}
	\end{partintro}

	\part{Electromagnetism} \label{Electromagnetism Part}
	\begin{partintro}
		\noindent Primary:
		\begin{itemize}
			\item[1.] Brown and Churchill - Complex Variables and Applications
		\end{itemize}
		Supplement: 
		\begin{itemize}
			\item[1.] A. David Wunsch - Complex Variables with Applications
		\end{itemize}
	\end{partintro}

	\part{Thermodynamics} \label{Thermodynamics Part}
	\begin{partintro}
		\noindent Primary:
		\begin{itemize}
			\item[1.] Brown and Churchill - Complex Variables and Applications
		\end{itemize}
		Supplement: 
		\begin{itemize}
			\item[1.] A. David Wunsch - Complex Variables with Applications
		\end{itemize}
	\end{partintro}

	\part{Statistical Mechanics} \label{Statistical Mechanics Part}
	\begin{partintro}
		\noindent Primary:
		\begin{itemize}
			\item[1.] Brown and Churchill - Complex Variables and Applications
		\end{itemize}
		Supplement: 
		\begin{itemize}
			\item[1.] A. David Wunsch - Complex Variables with Applications
		\end{itemize}
	\end{partintro}

	\part{Quantum Mechanics} \label{Quantum Mechanics Part}
	\begin{partintro}
		\noindent Primary:
		\begin{itemize}
			\item[1.] Brown and Churchill - Complex Variables and Applications
		\end{itemize}
		Supplement: 
		\begin{itemize}
			\item[1.] A. David Wunsch - Complex Variables with Applications
		\end{itemize}
	\end{partintro}

	\part{Particle Physics} \label{Particle Physics Part}
	\begin{partintro}
		\noindent Primary:
		\begin{itemize}
			\item[1.] Brown and Churchill - Complex Variables and Applications
		\end{itemize}
		Supplement: 
		\begin{itemize}
			\item[1.] A. David Wunsch - Complex Variables with Applications
		\end{itemize}
	\end{partintro}

	\part{General Reletivity} \label{General Reletivity Part}
	\begin{partintro}
		\noindent Primary:
		\begin{itemize}
			\item[1.] Brown and Churchill - Complex Variables and Applications
		\end{itemize}
		Supplement: 
		\begin{itemize}
			\item[1.] A. David Wunsch - Complex Variables with Applications
		\end{itemize}
	\end{partintro}
	
	
	
	\backmatter
	\part{Index} \label{Index Part}
	
	\part{Bibliography}
	\bibliographystyle{unsrt}
	\typeout{}
	\bibliography{Bibliography}
	

\end{document}
