% Created Oct 2020
\documentclass[12pt, english]{book}
\usepackage[letterpaper, portrait, margin=2.5cm]{geometry} %Set paper type, format, margin
\usepackage[svgnames]{xcolor}
\usepackage{%abstract, 
	amsmath, amssymb, array, babel, booktabs, caption, censor, fancyhdr, flafter, float, framed, gensymb, hyperref, inputenc, indentfirst, mathtools, MnSymbol, multicol, multirow, longtable, lscape, ltablex, polynom, pgfplots, physics, placeins, rotating, scalerel, setspace, stackengine, tabularx, threeparttable, titlesec, titling, wasysym, wrapfig}
\usepackage{cleveref}
\usetikzlibrary{calc,angles,positioning,intersections,quotes,decorations.markings,backgrounds,patterns}
\usepgfplotslibrary{groupplots}
\usepackage{amsthm} %Do not use amsthm with ntheorem unlees ntheorem is using the amsthm option
\usepackage{thmtools}
\pgfplotsset{compat=1.8}
\usepgfplotslibrary{external}
\tikzexternalize[prefix=tikz/] 

\setlength{\parindent}{0em} %Set indent on paragraph
\setlength{\parskip}{0.5em} %Set spaces between paragraphs

\newcommand\showdiv[1]{\overline{\smash{\hstretch{.5}{)}\mkern-3.2mu\hstretch{.5}{)}}#1}}
\newcommand\ph[1]{\textcolor{white}{#1}}
\setstackgap{S}{1.5pt}

% Swap the definition of \abs* and \norm*, so that \abs
% and \norm resizes the size of the brackets, and the 
% starred version does not.


\declaretheoremstyle[
	spaceabove=6pt, 		
	spacebelow=6pt,
	%	headfont=\normalfont\bfseries,
	notefont=\bfseries, notebraces={}{},
	bodyfont=\slshape,
	postheadspace=1em,
	headpunct=,
	headformat={\NAME~\NUMBER:\NOTE \hfill \smallskip\linebreak},%
	]{axiomstyle}

\declaretheoremstyle[
	spaceabove=6pt, 		
	spacebelow=6pt,
	headfont=\bfseries, %\color{blue},
	notefont=\bfseries, notebraces={}{},
	bodyfont=\slshape \color{DarkBlue},
	postheadspace=1em,
	headpunct=,
	headformat={\NAME~\NUMBER:\NOTE \hfill \smallskip\linebreak},%
	]{theoremstyle}
	
\declaretheoremstyle[
	spaceabove=6pt, 		
	spacebelow=6pt,
	headfont=\bfseries, %\color{Blue}%
	notefont=\bfseries,%
	notebraces={}{},%
	headpunct=,%
	bodyfont=\itshape\color{DarkGreen},%
	headformat={\NAME~\NUMBER:\NOTE \hfill \smallskip\linebreak},%
	%	preheadhook=\begin{leftbar},%
	%	postfoothook=\end{leftbar},%
	]{minorstyle}
	
\declaretheoremstyle[
	spaceabove=6pt, 		
	spacebelow=6pt,
	headfont=\bfseries,%\color{Blue},%
	notefont=\bfseries,%
	notebraces={}{},%
	headpunct=,%
	bodyfont=\slshape \color{OrangeRed},%
	headformat={\NAME~\NUMBER:\NOTE \hfill \smallskip\linebreak},%
%	preheadhook=\begin{leftbar},%
%	postfoothook=\end{leftbar},%
	]{defstyle}
	
\declaretheoremstyle[
	spaceabove=6pt, 		
	spacebelow=6pt,
	%	headfont=\normalfont\bfseries,
	bodyfont=\slshape,
%	postheadspace=1em,
	headpunct=,
%	headformat={\NAME~\NUMBER:\NOTE \hfill \smallskip\linebreak},%
	]{examplestyle}

\declaretheorem{axiom}[
	numberwithin=chapter, style=axiomstyle, parent=chapter,
	]
	
\declaretheorem{theorem, conjecture}[
	numberwithin=section, style=theoremstyle, parent=section,
	]
	
\declaretheorem{lemma, proposition, corollary}[
	style=minorstyle,
	parent=theorem,
%	numberwithin=section
	]

\declaretheorem{definition}[parent=section, style=defstyle]

\declaretheorem{remark, observation, question}[numbered=no]

\declaretheorem{example}[numberwithin=section, style=examplestyle]


% Indenting the Proof enviroment
\makeatletter
\renewenvironment{proof}[1][\proofname]{\par
	\pushQED{\qed}%
	\normalfont \topsep6\p@\@plus6\p@\relax
	\list{}{%
		\settowidth{\leftmargin}{\itshape\proofname:\hskip\labelsep}%
		\setlength{\labelwidth}{0pt}%
		\setlength{\itemindent}{-\leftmargin}%
	}%
	\item[\hskip\labelsep\itshape#1\@addpunct{:}]\ignorespaces
}{%
	\popQED\endlist\@endpefalse
}
\makeatother

% Letters after part
\makeatletter
\renewcommand{\@endpart}{\vfil\newpage}
\makeatother
\newenvironment{partintro}
{\vspace*{\fill}
	\section*{\centering Resources used in part \thepart}
	\begin{quotation}}
	{\end{quotation}\vspace*{\fill}\newpage}
\newcommand{\nopartintro}{%
	\vspace*{\fill}
	\thispagestyle{empty}
	\newpage
}

\begin{document}
	\title{The Book of Math (Notes)}
	\author{Kevin Kuo}
	
	\pagestyle{fancy}
	\fancyhead{} % clear all header fields
	\fancyhead[LO]{  }
	\fancyhead[CO]{  }
	\fancyhead[RO]{  }
	\renewcommand{\headrulewidth}{0pt}
	
	\fancyfoot{} % clear all footer fields
	\fancyfoot[LO]{}
	\fancyfoot[CO]{\thepage}
	\fancyfoot[RO]{}
	\renewcommand{\footrulewidth}{0pt}
	\vspace{0cm}	
	
	\frontmatter
	
	\maketitle
	
	\newpage
	\section*{Forward and Disclaimer}
	These are math notes made by a student (with a physics major and math minor) based off text books. It may contain misconceptions and misinterpretations, thus should not be viewed in the same light of a text book. Use at your own risk and mental sanity.
	
	\section*{Symbols}
	\begin{tabularx}{\textwidth}{ l c p{0.6\linewidth}}
		\multicolumn{3}{l}{\textbf{{\large Logic}}} \\ [10pt]
		\hline
		\textbf{Name} & \textbf{Symbol} & \textbf{Comment} \\
		\hline
		Exists 					& $\exists$ 		& There exists at least one\\
		For all 				& $\forall$ 		& \\
		Not exists 				& $\nexists$ 		& There does not exist\\ 
		Exists one				& $\exists!$ 		& There only exists one and only one \\
		And 					& $\land$			& \\
		Or						& $\lor$			& Inclusive or \\
		Not 					& $\neg$			& \\
		Logically implies 		& $\implies$ 		& If \\
		Logically implied by 	& $\Longleftarrow$ 	& Only if \\  
		Logically equivalent 	& $\iff$ 			& If and only if \\
		Implies 				& $\longrightarrow$	& \\
		Implied by 				& $\longleftarrow$ 	& \\  
		Double Implication 		& $\longleftrightarrow$	& \\
		\hline	
		
		& & \\
		\multicolumn{3}{l}{\textbf{{\large Set Notation}}} \\ [10pt]
		\hline
		\textbf{Name} & \textbf{Symbol} & \textbf{Comment} \\
		\hline
 		Empty Set 				& $\emptyset$ 		& The set that is empty \\
 		Natural Numbers 		& $\mathbb{N}$		& Set of natural numbers not containing 0, equivalent to the set of positive integers \\
 		Integers 				& $\mathbb{Z}$		& Set of integers \\
 		Rational Numbers 		& $\mathbb{Q}$		& \\
 		Algebraic Numbers		& $\mathbb{A}$		& \\
 		Real Numbers 			& $\mathbb{R}$		& \\
 		Complex Numbers 		& $\mathbb{C}$		& \\
 		
 		In 						& $\in$ 			& \\
 		Not in 					& $\nin$			& \\
 		Owns 					& $\ni$				& Has an element \\
 		
 		Proper Subset 			& $\subset$			& Subset that is not itself \\
 		Subset 					& $\subseteq$		& \\
 		Superset 				& $\supset$ 		& Superset that is not itself\\
 		Proper Superset 		& $\supseteq$		& \\
 		Power set				& $\wp$				& \\
 		Union 					& $\cup$			& \\
 		Intersection			& $\cap$			& \\
 		Difference				& $\setminus$		& \\
 		\hline
 		
 		& & \\
 		\multicolumn{3}{l}{\textbf{{\large Relationships}}} \\ [10pt]
 		\hline
 		\textbf{Name} & \textbf{Symbol} & \textbf{Comment} \\
 		\hline
 		Defined 				& $\doteq$ 			& \\
 		Approximate 			& $\approx$			& \\
 		Equivalent				& $\equiv$	 		& Isomorphic (Group Theory) \\
 		Congruent 				& $\cong$			& Homomorphic (Group Theory) \\
 		Proportional 			& $\propto$			& \\
 		\hline
 		
 		& & \\
 		\multicolumn{3}{l}{\textbf{{\large Operators}}} \\ [10pt]
 		\hline
 		\textbf{Name} & \textbf{Symbol} & \textbf{Comment} \\
 		\hline
 		& $\oplus$ & \\
 		& $\otimes$ & \\
 		& $\odot$ & \\
 		& $\circ$ & Convolution \\
 		Dagger& $\dagger$ & Complex conjugate transpose of a matrix \\
 		\hline
 		
 		& & \\
 		\multicolumn{3}{l}{\textbf{{\large Arrows}}} \\ [10pt]
 		\hline
 		\textbf{Name} & \textbf{Symbol} & \textbf{Comment} \\
 		\hline
 		Maps to 				& $\mapsto$			& \\
 		\hline
 		
 		& & \\
 		\multicolumn{3}{l}{\textbf{{\large Hebrew}}} \\ [10pt]
 		\hline
 		\textbf{Name} & \textbf{Symbol} & \textbf{Comment} \\
 		\hline
 		Aleph					& $\aleph$			& Carnality of infinite sets that can be well ordered \\
 		\hline
 		
 		& & \\
 		\multicolumn{3}{l}{\textbf{{\large Other}}} \\ [10pt]
 		\hline
 		\textbf{Name} & \textbf{Symbol} & \textbf{Comment} \\
 		\hline
 		Real part 				& $\Re$				& Real part of a number \\
 		Imaginary part 			& $\Im$				& Imaginary part of a number \\
 		\hline
	\end{tabularx}

	\newpage
	\section*{Book Constitution}
	\subsection*{Intents and Purpose}
	The goal of this book is to organize mathematical knowledge of topics related to the study of physics or the author's interest. It is meant to be used as a source of for future reference, not as a textbook for students new to the topics. It is a notebook of a student, thus should be treated as one and not as a textbook. At most, it could be used as a study guide along side a textbook. Definitely not as the main source for acquiring knowledge. 
	
	\subsection*{Layout and Organization}
	The book is split into parts each containing a field of study mathematics, or a topic large enough to justify giving it its own part. Each part contains chapters that focuses on a particular topic required to understand the field, with sections dedicated to describing a particular knowledge required for the topic. 
	
	As axioms, definitions, theorems, corollary, and proofs are integral and abundant to the study of mathematics, each will have a unique style. Each environment and its styles are displayed as follows: 
	
	\begin{axiom}[Axiom name]{Example Axiom}
		Axioms are the ``ground rules" of the set.
	\end{axiom}

	\begin{theorem}[Theorem name or citation]{Example Theorem}
		An important logical result from the axioms, with proof.
	\end{theorem}

	\begin{conjecture}[Name of conjecture or citation]{Example Conjecture}
		A hypothesis, without proof.
	\end{conjecture}
	
	\begin{corollary}{Example Corollary}
		An implication as a result of a theorem.
	\end{corollary}

	\begin{lemma}{Example Lemma}
		Small theorems that build up to a larger theorem. 
	\end{lemma}
	
	\begin{proposition}{Example Proposition}
		Example proposition.
	\end{proposition}
	
	\begin{proof}
		Logical deductions that results in a theorem.
		\textcolor{Grey}{Proofs I've written will be in grey, which may or may not be correct.}
	\end{proof}

%	\theoremstyle{break}
	\begin{definition}[Word]{Example Definition}
		The definition of a word.
	\end{definition}	

	\begin{example}
		An example.
	\end{example}

	\begin{remark}{Remark}
		A comment by the author in the textbooks used.
	\end{remark}
	
	\begin{observation}{Example Observation}
		A remark by me.
	\end{observation}

	\begin{question}{Example Question}
		A question from me for a mystery to be answered later. 
	\end{question}

	
	\section*{}
	\tableofcontents
	
	\mainmatter
	\part{Logic} \label{Logic Part}
	
	\chapter{Proofs}
	
	
	\part{Numbers} \label{Numbers Part}
	\begin{partintro}
		content...
	\end{partintro}
	
	\chapter{Natural $\mathbb{N}$} \label{Natural Chapter - Numbers}
	
	\chapter{Integers $\mathbb{Z}$} \label{Integers Chapter - Numbers}
	
	\chapter{Rationals $\mathbb{Q}$} \label{Rationals Chapter - Numebers}
	
	\chapter{Constructible} \label{Constructible Chapter - Numbers}
	
	\chapter{Algebraic $\mathbb{A}$} \label{Algebraic Chapter - Numbers}
	
	\chapter{Reals $\mathbb{R}$} \label{Reals Chapter - Numbers}
	
	\chapter{Complex $\mathbb{C}$} \label{Complex Chapter - Numbers}
	

	
	\part{Real Analysis} \label{Real Analysis Part}
	\begin{partintro}
		\begin{itemize}
			\item[1.] Kenneth A. Ross - Elementary Analysis (2nd Ed.) \cite{Ross.K-Elementary-Analysis-2013}
		\end{itemize}
	\end{partintro}

	\chapter{Sequences} \label{Sequences Chapter - Real Analysis}
	
	\section{Limits} \label{Limits Section - Real Analysis}
	
	\subsection{Limit Theorems} \label{Limit Theorems Subsection - Real Analysis}
	
	\section{Monotone and Cauchy Sequences} \label{Monotone and Cauchy Sequences Section - Real Analysis}
	
	\section{Subsequences} \label{Subsequences Section - Real Analysis}
	
	\section{$\lim \operatorname{sup}$ and $\lim \operatorname{inf}$} \label{lim sup and lim inf Section - Real Analysis}
	
	\section{Series} \label{Series Section - Real Analysis}
	
	\section{Alternating Series and Integral Tests} \label{Alternating Series and Integral Tests Section - Real Analysis}
	
	\chapter{Continuity} \label{Continuity Chapter - Real Analysis}
	
	\section{Continuous Functions} \label{Continuous Functions Section - Real Analysis}
	
	\subsection{Properties} \label{Properties of Continuous Functions Subsection - Real Analysis}
	
	\section{Uniform Continuity} \label{Uniform Continuity Section - Real Analysis}
	
	\section{Limits of Functions} \label{Limits of Functions Section - Real Analysis}

	\chapter{Metric Spaces}	
	
	\part{Complex Analysis} \label{Complex Analysis Part}
	\begin{partintro}
	\noindent Primary:
		\begin{itemize}
			\item[1.] Brown and Churchill - Complex Variables and Applications \cite{Brown.J;Churchill.R-Complex-Variables-2014}
		\end{itemize}
		Supplement: 
		\begin{itemize}
			\item[1.] A. David Wunsch - Complex Variables with Applications \cite{Wunsh.A-Complex-Variables-2005}
		\end{itemize}
	\end{partintro}
	
	\chapter{Basics} \label{Basics Chapter - Complex}
	\section{Complex Numbers} \label{Complex Numbers Section - Complex}
	$$\mathbb{C} = \{x + iy \mid x, y \in \mathbb{R}, i = \sqrt{-1}\}$$
	Complex numbers are elements of the complex field $(\mathbb{C})$, therefore, they obey all the properties of a field. 
	
	We will denote complex numbers by $z = x + iy$ with $x, y \in \mathbb{R}$, and refer the real part as $\Re(z) = \operatorname{Re}(z) = x$ and imaginary part as $\Im(z) = \operatorname{Im}(z) = y$. Complex numbers can also be defined as an ordered pair $z = (x, y)$ which is interpreted as points in the complex plane. $(x, 0)$ are points on the real axis while $(0 , y)$ are points in the imaginary axis. This expression is often called a Couple, and was presented in 1833 by mathematician William Rowan Hamilton (1805 - 1865).
	
	\begin{center}
		\begin{tikzpicture}
		\begin{axis}[
			small,
			%			title={},
			xlabel={$\Re$},
			ylabel={$\Im$},
			xmin=0, xmax=10,
			ymin=0, ymax=10,
			xtick={100},
			ytick={100},
			axis lines = left,
%			legend pos=north west,
%			ymajorgrids=true,
%			grid style=dashed,
			]
			
			\addplot[
			only marks,
			nodes near coords,
			point meta = explicit symbolic, 
			color=Black,
			mark=*,
			]
			coordinates {
				(7, 5) [$z = x + iy$]
			};
%			\legend{$z = x + iy$}
		\end{axis}
	\end{tikzpicture}
	\end{center}
	Like numbers in $\mathbb{R}$, numbers in $\mathbb{C}$ obey the commutative, distributive, and associative laws. We add and multiply complex numbers in the usual way: 
	\begin{align*}
		z_1 + z_2 &= (x_1 + iy_1) + (x_2 + iy_2) & z_1 z_2 &= (x_1 + iy_1) (x_2 + iy_2) \\
			&= (x_1 + x_2) + i(y_1 + y_2) & &=(x_1 x_2 - y_1 y_2) + i(x_1 y_2 + x_2 y_1)
	\end{align*}
	$\forall z \in \mathbb{C}$, there is an unique additive inverse $(-z)$ and $\forall z \in \mathbb{C}\setminus\{0\}$, there is an unique multiplicative inverse $(z^{-1})$ such that 
	\begin{align*}
		&z + (-z) = 0  & &zz^{-1} = 1 \\
		&\implies -z = -x - iy & &\implies (x_1 x_2 - y_1 y_2) = 1 \land (x_1 y_2 + x_2 y_1) = 0 \\
		& & &\implies z^{-1} = \frac{x_1}{x_1^2 + y_1^2} - i \frac{y_1}{x_1^2 + y_1^2}
	\end{align*}
	The existence and uniqueness of the inverses can be easily proven. 
	
	The addition of complex numbers may also be interpreted as akin to vector addition. 
	\begin{center}
		\begin{tikzpicture}
			\begin{axis}[
				small,
				%			title={},
				xlabel={$\Re$},
				ylabel={$\Im$},
				xmin=0, xmax=10,
				ymin=0, ymax=10,
				xtick={100},
				ytick={100},
				axis lines = left,
				%			legend pos=north west,
				%			ymajorgrids=true,
				%			grid style=dashed,
				]
				
				\addplot[
				only marks,
				nodes near coords,
				point meta = explicit symbolic, 
				color=Black,
				mark=*,
				]
				coordinates {
					(7, 5) [$z_1$]
					(1, 4) [$z_2$]
					(8, 9) [$z_1 + z_2$]
				};
				%			\legend{$z = x + iy$}
				\draw[line width=1pt,Grey,-stealth](0,0)--(70,50);
				\draw[line width=1pt,Grey,-stealth](0,0)--(10,40);
				\draw[line width=1pt,Grey,-stealth](70,50)--(80,90);
				\draw[line width=1pt,Grey,-stealth](10,40)--(80,90);
				\draw[line width=1pt,black,-stealth](0,0)--(80,90);
			\end{axis}
	\end{tikzpicture}
	\end{center}
	
	\section{Triangle Inequality} \label{Triangle Inequality Section - Complex}
	It is not analysis without a section dedicated to the triangle inequality. For any given number $z_1, z_2 \in \mathbb{C}$ it makes no sense to write an inequality $z_1 = a_1 + ib_1 <  a_2 + ib_2 = z_2$. Thus, we need have a different notion of size. 
	
	\begin{definition}[Modulus]
		The modulus of a complex number is a function $\mathbb{C} \rightarrow \mathbb{R}_{>0}$:
		$$\abs{z} = \sqrt{x^2 + y^2} = \sqrt{z \bar{z}}$$
	\end{definition}
	It is obvious why the definition is not $\abs{z} = \sqrt{x^2 + (iy)^2}$ as problems arise when $x = y$. The modulus is the distance of $z$ from $(0, 0)$. $\bar{z}$ is the complex conjugate of $z$, which is explored in \cref{Complex Conjugate Section - Complex}
	
	\begin{theorem}[Triangle Inequality]
		$\forall z_1, z_2 \in \mathbb{C} [\abs{z_1 + z_2} \leq \abs{z_1} + \abs{z_2}]$
		\label{Triange Inequality - Complex}
	\end{theorem}
	
	From the theorem, we can derive a similar inequality: 
	\begin{align*}
		\abs{z_1} = \abs{z_1 + z_2 - z_2} \leq \abs{z_1 + z_2} + \abs{-z_2}
		 &\implies \abs{z_1} - \abs{z_2} \leq \abs{z_1 + z_2}
	\end{align*}
	
	An important property of polynomials is observed when \cref{Triange Inequality - Complex} is applied to polynomials.
	
	\begin{corollary}
		Consider the polynomial $P(z)$ where $a_n \in \mathbb{C}$, $n \in \mathbb{N}$, $a_0 \neq 0$, and $z \in \mathbb{C}$.
		$$P(z) = a_0 + a_1 z + a_2 z^2 + \ldots + a_n z^n$$
		Then $\forall z, \exists R \in \mathbb{R}_{>0}, \abs{z} < R$ such that
		$$\abs{\frac{1}{P(z)}} < \frac{2}{\abs{a_n} R^n}$$
		\label{Complex Poly Reciprocal Bounded}
	\end{corollary}
	\begin{proof}
		Consider 
		\begin{align*}
			w &= \frac{P(z)}{z_n} - a_n = \frac{a_0}{z^n} + \frac{a_1}{z^{n-1}} + \ldots + \frac{a_{n-1}}{z} 
				& z \neq 0 \\
			&\implies wz^n = a_0 + a_1 z + \ldots + a_{n-1}z^{n-1} \\
			&\implies \abs{w}\abs{z}^n \leq \abs{a_0} + \abs{a_1}\abs{z} + \ldots + \abs{a_{n-1}}\abs{z}^{n-1} \\
			&\implies \abs{w} \leq \frac{\abs{a_0}}{\abs{z}^n} + \frac{\abs{a_1}}{\abs{z}^{n-1}} + \ldots + \frac{\abs{a_{n-1}}}{\abs{z}} 
				& \\
			&\implies \abs{w} < n\frac{\abs{a_n}}{2n} = \frac{\abs{a_n}}{2} 
				& \exists \text{ sufficiently large } R < \abs{z}  \text{ s.t.}\\
			& 	& \forall m, \ 0 \leq m \leq n-1, \ \frac{\abs{a_m}}{\abs{z}^{n-m}} < \frac{\abs{a_n}}{2n} \\
			&\implies \abs{a_n + w} \geq \abs{\abs{a_n} - \abs{w}} > \frac{\abs{a_n}}{2}
				& R < \abs{z} \\
			&\implies \abs{P_n(z)} = \abs{a_n + w} \abs{z}^n > \frac{\abs{a_n}}{2}\abs{z}^n > \frac{\abs{a_n}}{2} R^n 
				& R < \abs{z} \\
			&\implies \abs{\frac{1}{P(z)}} < \frac{2}{\abs{a_n} R^n}
		\end{align*}
	\end{proof}
	This tells us that if $z$ is a solution to a polynomial $P(z)$, then the reciprocal of the polynomial $1/P(z)$ is bounded above by $R = \abs{z}$. (i.e. It is bounded by a circle of radius $\abs{z}$.)
	
	\section{Polar and Exponential Form} \label{Polar and Exponential Form Section - Complex}
	\begin{definition}[Argument of $z$]
		Consider any $z \in \mathbb{C}$ where $z \neq 0$.
		Let $\theta$ be the angle in radians between $z$ and the real axis .
		Then $\forall n \in \mathbb{N}$, $-\pi < \theta \leq \pi$, the argument of $z$:
		$$ \operatorname{arg}(z) = \theta + 2 n \pi$$
		\label{Argument - Complex}
	\end{definition}
	
	We know $\forall n \in \mathbb{N}$, $\theta + 2 \pi n = \theta$. This leads us to the definition of the principal argument of $z$.
	\begin{definition}[Principal Argument of $z$]
		Consider any $z \in \mathbb{C}$ where $z \neq 0$.
		Let $\theta$ be the angle in radians between $z$ and the real axis.
		Then for $-pi < \theta \leq \pi$, the principal argument of $z$:
		$$\operatorname{Arg}(z) = \theta$$
		\label{Principal Argument - Complex}
	\end{definition}
	It is clear that $\operatorname{arg}(z) = \operatorname{Arg}(z) + 2n \pi$. It is common for the principal argument to be defined $-\pi < \theta \leq \pi$, although other definitions use $0 \leq \theta < 2\pi$.
	
	\begin{definition}[Polar Form of $z$]
		Consider $z \in \mathbb{C}$. Let $r = \abs{z}$, and $\theta = \operatorname{arg}(z)$. 
		Then $\forall z \in \mathbb{C}, z \neq 0$:
		$$z = x + iy = r(\cos(\theta) + i \sin(\theta))$$
		\label{Polar Form of z - Complex}
	\end{definition}

	Notice that all three definitions require that $z \neq 0$ as $\theta$ is undefined at $z = 0$.
	
	\begin{theorem}[Euler's Formula]
		$$e^{i \theta} = \cos(\theta) + i \sin(\theta)$$
		\label{Euler's Formula - Complex}
	\end{theorem}
	Combining \cref{Polar Form of z - Complex} with \cref{Euler's Formula - Complex}, we obtain the Exponential Form of $z$: 
	
	\begin{definition}[Exponential Form of $z$]
	Consider any $z \in \mathbb{C}$, and let $r = \abs{z}$ and $\theta = \operatorname{Arg}(z)$. Then the exponential form of $z$:
		$$z = r e^{i \theta}$$
		\label{Exponential Form of z - Complex}
	\end{definition}
	Note: $\theta = \tan^{-1}(y/x)$ and $r = \sqrt{x^2 + y^2}$.
	\begin{center}
		\begin{tikzpicture}
			\begin{axis}[
				small,
				%			title={},
				xlabel={$\Re$},
				ylabel={$\Im$},
				xmin=0, xmax=10,
				ymin=0, ymax=10,
				xtick={100},
				ytick={100},
				axis lines = left,
				%			legend pos=north west,
				%			ymajorgrids=true,
				%			grid style=dashed,
				]
				
				\addplot[
				only marks,
				nodes near coords,
				point meta = explicit symbolic, 
				color=Black,
				mark=*,
				]
				coordinates {
					(7, 5) [$z = x + iy = re^{i\theta}$]
				};
				%			\legend{$z = x + iy$}
				\draw[thick]  (0,0) to ["$r=\abs{z}$"] (70,50);
				% angle
				\draw[draw=blue] (0,0) ++(35:50) arc (35:0:50)
				node[midway,above right,inner sep=2pt,font={\footnotesize}]{$\theta$};
			\end{axis}
		\end{tikzpicture}
	\end{center}
	
	\subsection{Properties of Polar and Exponential Form} \label{Properties of Polar and Exponential Form Subsection - Complex}
	It would be easier to work with the exponential form of $z$ then convert it to the polar form later. The exponential form of a complex number is part of the exponential family of functions, thus possess all the properties of the family. Consider any complex number $z_1 = r_1 e^{i\theta_1}$ and $z_2 = r_2 e^{i\theta_2}$.
	\begin{align*}
		z_1  z_2 &= r_1 r_2 e^{i(\theta_1 + \theta_2)} 
			& z^n &= r^n e^{i n \theta} \qquad \forall n \in \mathbb{Z}
	\end{align*}
	A special case arrives for integer exponential of $z$ on the unit circle.
	\begin{theorem}[de Moivre's Formula]
		Consider any $z = e^{i \theta} \in \mathbb{C}$ on the unit circle, and let $n \in \mathbb{Z}$.
		\begin{align*}
			\forall z\in \mathbb{C} \ \forall n \in \mathbb{Z}
			\left[\abs{z} = 1 \implies (\cos(\theta) + i\sin(\theta))^n = \cos(n \theta) + i\sin(n \theta)\right]
		\end{align*}
		\label{de Moivre's Formula Theorem - Complex}
	\end{theorem}
	\begin{proof}
		Consider $z = e^{i \theta}$ and let $n \in \mathbb{Z}$. 
		\begin{align*}
			z^n = (e^{i \theta})^n = e^{in\theta} = \cos(n\theta) + i\sin(n\theta)
		\end{align*}
	\end{proof}

	The proof hints that \cref{de Moivre's Formula Theorem - Complex} can be generalized to $\forall n \in \mathbb{R}$, which we will see shortly in \cref{Roots of z Section - Complex}. Using \cref{de Moivre's Formula Theorem - Complex}, we can obtain the double angle identities.
	
	\begin{corollary}[Double Angle Identities]
		\begin{align*}
			\cos(2 \theta) &= \cos^2(\theta) - \sin^2(\theta) 
				& \sin(2\theta) &= 2\sin(\theta)\cos(\theta)
		\end{align*}
	\end{corollary}
	\begin{proof}
		Consider any $z$ on the unit circle, that is $z=e^{i\theta}$.
		\begin{align*}
			&(\cos(\theta) + i \sin(\theta))^2 = \cos(2\theta) + i \sin(2\theta)
				&\text{\Cref{de Moivre's Formula Theorem - Complex}} \\
			&\implies \cos^2(\theta) - \sin^2(\theta) + i2\sin(\theta)\cos(\theta) = cos(2\theta) + i\sin(2\theta) 
		\end{align*}
		Equating the real and imaginary parts yield the desired results. 
	\end{proof}

	\subsection{Properties of Arguments} \label{Properties of Argumetns Subsection - Complex}
	Recall from \cref{Properties of Polar and Exponential Form Subsection - Complex}: 
	\begin{align*}
		z_1  z_2 &= r_1 r_2 e^{i(\theta_1 + \theta_2)} 
		& z^n &= r^n e^{i n \theta} \qquad \forall n \in \mathbb{Z}
	\end{align*}
	The arguments for the arguments of products of any $z_1, z_2 \in \mathbb{C}$ follows immediately from the properties of the exponential.
	\begin{corollary}[Arguments of Products]
		\begin{align*}
			\arg(z_1 z_2) &= \arg(z_1) + \arg(z_2)
				& \operatorname{Arg}(z_1 z_2) = \operatorname{Arg}(z_1) + \operatorname{Arg}(z_2) \\
			\arg(z^n) &= n\arg(z)  
				& \operatorname{Arg}(z^n) = n\operatorname{Arg}(z)
		\end{align*}
		\label{Arguments of Products Corollary - Complex}
	\end{corollary}
	\begin{proof}
		\begin{align*}
			z_1  z_2 &= r_1 r_2 e^{i(\theta_1 + \theta_2)} & \\
			&\implies \arg(z_1 z_2) = \arg(z_1) + 2n_1 \pi + \arg(z_2) + 2n_2\pi
				& n_1, n_2 \in \mathbb{Z} \\
			&\implies \arg(z_1 z_2) = \arg(z_1) + \arg(z_2) & \\
			&\implies \operatorname{Arg}(z_1 z_2) = \operatorname{Arg}(z_1) = \operatorname{Arg}(z_2)  \\
			\\
			z^n &= r^n e^{i n\theta} & \\
			&\implies \arg(z^n) = n\arg(z) + 2n\pi & n \in \mathbb{Z} \\
			&\implies \arg(z^n) = n\arg(z)  \\
			&\implies \operatorname{z^n} = n\operatorname{Arg}(z)
		\end{align*}
	\end{proof}
	It is clear that: 
	\begin{align*}
		\arg\left(\frac{z_1}{z_2}\right) &= \arg(z_1) - \arg(z_2)
			& \operatorname{Arg}\left(\frac{z_1}{z_2}\right) &= \operatorname{Arg}(z_1) - \operatorname{Arg}(z_2)
	\end{align*}

	\section{Roots of $z$} \label{Roots of z Sections - Complex}
	In \cref{Exponential Form of z - Complex}, you might be wondering why $z^n = r^n e^{i n \theta}$ is not for $n \in \mathbb{R}$. That is because there is more things to consider, which we will explore in this section. Recall that $z = re^(i \theta) = re^{i (\theta + 2n \pi)}$ for $n \in \mathbb{Z}$. 
	
	\begin{definition}[Exponential of $z$]
		Consider any $z \in \mathbb{C}$ and any $x \in \mathbb{R}$
		$$z^x = \left(r e^{i(\theta + 2 n \pi)} \right)^x = r^x e^{i x(\theta + 2n \pi)}$$
		\label{Exponential of z Definiion - Complex}
	\end{definition}

	For $x \nin \mathbb{Z}$, it is clear that $z^x =  r^x e^{ix(\theta + 2 n \pi)} \neq r^x e^{ix\theta}$, since $2nx\pi = 0 \iff nx \in \mathbb{Z}$. In order to define the roots of $z$ we must need a more general and proper definition of $z$.
	
	\begin{definition}[Roots of $z_0$]
		\label{Roots of z Definition - Complex}
		Consider any $z_0 \in \mathbb{C}$ and any $m \in \mathbb{N}$.
		$$z_0^{\frac{1}{m}} = r_0^\frac{1}{m} e^{i\left(\frac{\theta_0 + 2n \pi}{m}\right)} = r_0^\frac{1}{m} e^{i \left(\frac{\theta_0}{m} + \frac{2n \pi}{m}\right)}$$
	\end{definition}  
	
	Taking the $m$-th root of $z_0 \in C$ scales $\theta_0$ by $1/m$, and provides solutions at equally spaced by $2\pi / m$ on a circle of radius $r^{1/m}$. That is, the roots lie on the vertices of a regular n-sided polygon inscribed in a circle of radius $\abs{z}^{1/m}$. 
	
	\begin{example}
		Consider $z_0 = 32 e^{i(5/6)\pi}$, then $z_0^{(1/5)} = 3e^{i(\pi/6) + i(2/5) n \pi}$ for $n \in \mathbb{Z}$. The radius went from $35$ to $35^{(1/5)} = 2$, and five roots appear equally spaced with distance of $(2/5)\pi$ on a circle with radius $2$. Before and after graphs are as follows, note graph on right is zoomed in: 
		\begin{figure}[H]
			\centering
			\begin{tikzpicture}[remember picture, baseline=(current bounding box.center)]
				\begin{axis}[
					width=8.9cm,
					unit vector ratio=1 1 1,
					xlabel={$\Re$},
					ylabel={$\Im$},
					xmin=-30, xmax=30,
					ymin=-30, ymax=30,
%					xtick={-20,-10,10,20},
%					ytick={-20,-10,10,20},
					xtick=100,
					ytick=100,
					axis lines = middle,
					]
					
					\addplot[
					only marks,
					nodes near coords,
					point meta = explicit symbolic, 
					color=Black,
					mark=*,
					]
					coordinates {
						(-27.7128, 16) [$z_0 = 32e^{i(5/6)\pi}$]
					};
				\end{axis}
%				\path (current bounding box.north east) -- 
%				(current bounding box.south east) coordinate[midway] (2BL);
			\end{tikzpicture}
			{$\Large \xrightarrow{\mathmakebox[2cm] {f(z) = z^{1/5}}}$}
%			\hspace*{2.8cm}
			\begin{tikzpicture}[remember picture, baseline=(current bounding box.center)]
				\begin{axis}[
%					small,
					width=8.9cm,
					unit vector ratio=1 1 1,
					xlabel={$\Re$},
					ylabel={$\Im$},
					xmin=-3.1, xmax=3.1,
					ymin=-3.1, ymax=3.1,
%					xtick distance=2,
%					ytick distance=2,
					xtick=10,
					ytick=10,
					axis lines = middle,
					axis line style = Black,
					]
					\addplot[
					only marks,
					nodes near coords,
					point meta = explicit symbolic, 
					color=Black,
					mark=*,
					]
					coordinates {
						(1.73205, 1) [$2e^{i\pi/6}$]
						(-0.4158, 1.9563) [$2e^{i17\pi/30}$]
						(-1.9890, 0.20906) [$2e^{i29\pi/30}$]
						(-0.8135, -1.8271) [$2e^{-i19\pi/30}$]
						(1.4863, -1.3383) [$2e^{-i7\pi/30}$]
					};
					\draw[line width=1pt,LightGrey](axis cs: 0, 0) circle [radius=200];
					\draw[line width=1pt,LightGrey](axis cs: 0,0)--(axis cs: 1.73205, 1);
					\draw[line width=1pt,LightGrey](axis cs:0,0)--(axis cs:-0.4158, 1.9563);
					\draw[line width=1pt,LightGrey](axis cs:0,0)--(axis cs:-1.9890, 0.20906);
					\draw[line width=1pt,LightGrey](axis cs:0,0)--(axis cs:-0.8135, -1.8271);
					\draw[line width=1pt,LightGrey](axis cs:0,0)--(axis cs:1.4863, -1.3383);
				\end{axis}
%				\path (current bounding box.north west) -- 
%				(current bounding box.south west) coordinate[midway] (2BR);
			\end{tikzpicture}
%			\tikz[overlay,remember picture]{\draw[->, very thick] 
%			($(2BL)+(0.5, 0)$) -- ($(2BR)+(-0.5,0)$)
%			node[midway,above,text width=2cm]{$f(z) = z^\frac{1}{5}$};} 
		\end{figure}
	\end{example} 
	
	We can see that the roots of $z_0$ form a set:
	\begin{definition}[Set of roots of $z_0$]
		\label{Set of roots of z - Complex}
		Consider the $m$-th root of any $z_0 \in \mathbb{C}$. Let: 
		\begin{align*}
			z_0 &= r_0 e^{i\theta_0}
			&c_0 &= r_0^{1/m} e^{i\theta_0/m} 
			&\omega_n &= e^{\frac{i2\pi}{m}} & \ m \in \mathbb{N}
		\end{align*}
		Then the set of roots of $z_0$:
		\begin{align*}
			z_0^{1/m} = \left\{c_k = c_0 \omega_m^k \mid k\in \mathbb{N}, \ 0 \leq k < m\right\}
		\end{align*}
	\end{definition}
	$c_0$ is the principal root. The root corresponding to the principal argument of $z$.
	\begin{definition}[Principal Root]
	Consider the $m$-th root of any $z_0 \in \mathbb{C}$. The principal root of $z_0$ is defined as:
		$$c_0 = r_0^{\frac{1}{m}} e^{i\frac{\theta_0}{m}}$$
	\end{definition}
	
	\begin{example}
		Recall from the previous example: $z_0 = 32 e^{i(5/6)\pi}$. This gives us
		\begin{align*}
			c_0 &= 32^{1/5} e^{i\pi/6}= 2 e^{i\pi/6}
				&\omega_5 &= e^{i2\pi/5}
		\end{align*}
		Then
		\begin{align*}
			c_0 &= c_0 \omega_5^0 = 2 e^{i\pi/6} \\
			c_1 &= c_0 \omega_5^1 = 2 e^{i\pi/6} e^{i2\pi/5} = 2 e^{i17\pi/30} \\
			c_2 &= c_0 \omega_5^1 = 2 e^{i\pi/6} e^{i4\pi/5} = 2 e^{i29\pi/30} \\
			c_3 &= c_0 \omega_5^1 = 2 e^{i\pi/6} e^{i6\pi/5} = 2 e^{i41\pi/30} = 2 e^{-i19\pi/30}\\
			c_4 &= c_0 \omega_5^1 = 2 e^{i\pi/6} e^{i8\pi/5} = 2 e^{i53\pi/30} = 2 e^{-i7\pi/30}\\
		\end{align*}
		\begin{figure}[H]
			\centering
			\begin{tikzpicture}[remember picture]
				\begin{axis}[
					%					small,
					width=10cm,
					unit vector ratio=1 1 1,
					xlabel={$\Re$},
					ylabel={$\Im$},
					xmin=-3.1, xmax=3.1,
					ymin=-3.1, ymax=3.1,
					%					xtick distance=2,
					%					ytick distance=2,
					xtick=10,
					ytick=10,
					axis lines = middle,
					axis line style = Black,
					]
					\addplot[
					only marks,
					nodes near coords,
					point meta = explicit symbolic, 
					color=Black,
					mark=*,
					]
					coordinates {
						(1.73205, 1) [$c_0 = 2e^{i\pi/6}$]
						(-0.4158, 1.9563) [$c_1 = 2e^{i17\pi/30}$]
						(-1.9890, 0.20906) [$c_2 = 2e^{i29\pi/30}$]
						(-0.8135, -1.8271) [$c_3 = 2e^{-i19\pi/30}$]
						(1.4863, -1.3383) [$c_4 = 2e^{-i7\pi/30}$]
					};
					\draw[line width=1pt,LightGrey](axis cs: 0, 0) circle [radius=200];
					\draw[line width=1pt,LightGrey](axis cs: 0,0)--(axis cs: 1.73205, 1);
					\draw[line width=1pt,LightGrey](axis cs:0,0)--(axis cs:-0.4158, 1.9563);
					\draw[line width=1pt,LightGrey](axis cs:0,0)--(axis cs:-1.9890, 0.20906);
					\draw[line width=1pt,LightGrey](axis cs:0,0)--(axis cs:-0.8135, -1.8271);
					\draw[line width=1pt,LightGrey](axis cs:0,0)--(axis cs:1.4863, -1.3383);
				\end{axis}
				\path (current bounding box.north west) -- 
				(current bounding box.south west) coordinate[midway] (2BR);
			\end{tikzpicture}
		\end{figure}
	\end{example}
	
	
	\section{Complex Conjugate} \label{Complex Conjugate Section - Complex}
	\begin{definition}[Complex Conjugate]
		The complex conjugate of $z \in \mathbb{C}$ is denoted $\bar{z}$.
		$$\bar{z} = x - iy = r(\cos(\theta) - i\sin(\theta)) = re^{-i\theta}$$
		\label{Complex Conjugate}
	\end{definition}
	Graphically, it is the reflection of $z$ across the real axis.
	\begin{center}
		\begin{tikzpicture}
			\begin{axis}[
				small,
				xlabel={$\Re$},
				ylabel={$\Im$},
				xmin=0, xmax=10,
				ymin=-10, ymax=10,
				xtick={100},
				ytick={100},
				axis lines = middle,
				]
				\addplot[
				only marks,
				nodes near coords,
				point meta = explicit symbolic, 
				color=Black,
				mark=*,
				]
				coordinates {
					(7, 5) [$z = x + iy$]
					(7, -5) [$\bar{z} = x - iy$]
				};
			\end{axis}
		\end{tikzpicture}
	\end{center}
	It is then easy to see
	\begin{align*}
		\operatorname{Re}(z) &= \frac{z + \bar{z}}{2} & \operatorname{Im}(z) &= \frac{z - \bar{z}}{2i} & \abs{z}^2 = z \bar{z}
	\end{align*}
	As $\operatorname{Re}(z) = x = r \cos(\theta)$ and $\operatorname{Im}(z) = y = r \sin(\theta)$ and using \cref{Exponential Form of z - Complex}, we can obtain the complex forms of sine and cosine: 
	\begin{definition}[Complex Sine and Cosine]
		\begin{align*}
			\cos(\theta) &= \frac{e^{i \theta} + e^{-i \theta}}{2} 
			&\sin(\theta) &= \frac{e^{i \theta} - e^{-i \theta}}{2i}
		\end{align*}
		\label{Trig Identities - Complex}
	\end{definition}


	\section{Operations as Transformations} \label{Operations as Transformations Section - Complex}
	
	Consider any $z \in \mathbb{C}$. A function $f: \mathbb{C} \rightarrow \mathbb{C}$ can be viewed as transformations of the complex plane. 
	
	\begin{example}[Addition as translation]
		Consider any $z_0 \in \mathbb{C}$, $z_0 = a + ib$ for $a, b \in \mathbb{R}$. Addition by $z_0$ can be seen as a shift in the complex plane by $a + bi$. (i.e. It takes the origin and shifts it by $z_0$.)
		\begin{figure}[H]
			\centering
			\begin{tikzpicture}[remember picture, baseline=(current bounding box.center)]
				\begin{axis}[
					width=8.9cm,
					unit vector ratio=1 1 1,
					xlabel={$\Re$},
					ylabel={$\Im$},
					xmin=-10, xmax=10,
					ymin=-10, ymax=10,
					axis lines = middle,
					xticklabels={}, yticklabels={},
					grid=both,
					axis lines=middle,
					minor tick num=4,
					minor tick style={draw=none},
					]
					
					\addplot[
					only marks,
					nodes near coords,
					point meta = explicit symbolic, 
					color=Blue,
					mark=*,
					]
					coordinates {
						(0,0)
					};
				\end{axis}
			\end{tikzpicture}
			{$\Large \xrightarrow{\mathmakebox[2cm] {f(z_0) = z + z_0}}$}
			\begin{tikzpicture}[remember picture, baseline=(current bounding box.center)]
				\begin{axis}[
					%					small,
					width=8.9cm,
					unit vector ratio=1 1 1,
					xlabel={$\Re$},
					ylabel={$\Im$},
					xmin=-10, xmax=10,
					ymin=-10, ymax=10,
					xticklabels={}, yticklabels={},
					grid=both,
					axis lines=middle,
					minor tick num=4,
					minor tick style={draw=none},
					]
					\addplot[
					only marks,
					nodes near coords,
					point meta = explicit symbolic, 
					color=Blue,
					mark=*,
					]
					coordinates {
						(4, 3) [$z_0=a+bi$]
					};
					\draw[->, very thick](axis cs:0,0)--(axis cs:4, 3);
				\end{axis}
			\end{tikzpicture}
		\end{figure}
	\end{example}

	\begin{example}[Multiplication as scaling and rotation]
		Consider any $z_0 \in \mathbb{C}$, $z_0 = re^{i\theta}$. Multiplication by $z_0$ scales the entire complex plane by $r$ and rotates it by $\theta$. (Imagine rotating and stretching out a net.)
		\begin{figure}[H]
			\centering
			\begin{tikzpicture}[remember picture, baseline=(current bounding box.center)]
				\begin{axis}[
					width=8.9cm,
					unit vector ratio=1 1 1,
					xlabel={$\Re$},
					ylabel={$\Im$},
					xmin=-10, xmax=10,
					ymin=-10, ymax=10,
					axis lines = middle,
					xticklabels={}, yticklabels={},
					grid=both,
					axis lines=middle,
					minor tick num=4,
					minor tick style={draw=none},
					]
					
					\addplot[
					only marks,
					nodes near coords,
					point meta = explicit symbolic, 
					color=Blue,
					mark=*,
					]
					coordinates {
						(1,0) [$1 = e^{0}$]
					};
				\draw[->, very thick](axis cs:0,0)--(axis cs:1, 0);
				\end{axis}
			\end{tikzpicture}
			{$\Large \xrightarrow{\mathmakebox[2cm] {f(z_0) = z \cdot z_0}}$}
			\begin{tikzpicture}[remember picture, baseline=(current bounding box.center)]
				\begin{axis}[
					width=8.9cm,
					unit vector ratio=1 1 1,
					xlabel={$\Re$},
					ylabel={$\Im$},
					xmin=-10, xmax=10,
					ymin=-10, ymax=10,
					xticklabels={}, yticklabels={},
%					ytick distance={3},
%					xtick distance={4},
%					grid=both,
					axis lines=middle,
					minor tick num=0,
					minor tick style={draw=none},
					]
					\addplot[
					only marks,
					nodes near coords,
					point meta = explicit symbolic, 
					color=Blue,
					mark=*,
					]
					coordinates {
						(4, 3) [$z_0=re^{i\theta}$]
					};
%					\draw[->,thick] (axis cs:0,0)--(axis cs:4,3);
					\draw[->,thick,draw=Blue]  (axis cs: 0,0) to ["$r$"] (axis cs: 4,3);
					% angle
					\draw[->, draw=Blue] (axis cs:0,0)++(0:30) arc (0:35:30)
					node[midway,above right,inner sep=2pt,font={\footnotesize}]{$\theta$};
				\end{axis}
%				\begin{axis}[
%					anchor=center, % Shift the axis so its origin is at (0,0)
%					rotate around={36.87:(current axis.origin)}, % Rotate around the origin
%					width=8.9cm,
%					unit vector ratio=1 1 1,
%					xmin=-10, xmax=10,
%					ymin=-10, ymax=10,
%					xticklabels={}, yticklabels={},
%					grid=both,
%					axis lines=middle,
%					minor tick num=0,
%					minor tick style={draw=none},
%					]
%				\end{axis}
%				\pgftransformrotate{36.87}
%				\pgfpathgrid[stepx=50,stepy=50]{\pgfpoint{0}{-50}}{\pgfpoint{200}{150}}
%				\pgfusepath{stroke}
			\end{tikzpicture}
		\end{figure}
	\end{example}

%	\begin{example}[Logarithm as compression (Speculation)]
%		content...
%	\end{example}
	
	\section{Complex Analysis Definitions} \label{Complex Analysis Definitions Section - Complex}
	
	\begin{definition}[Neighbourhood]
		A neighbourhood of a point $z_0$ is the set of all points $z$ with distance less than $\epsilon$. 
		$$\{z : \abs{z-z_0} < \epsilon \}$$
		i.e. It is the set of all points that lie within a circle centred at $z_0$ with radius $\epsilon$. Points on the circumference not included. 
	\end{definition}
	\begin{figure}[H]
		\centering
		\begin{tikzpicture}[remember picture]
			\begin{axis}[
				small,
				unit vector ratio=1 1 1,
				xlabel={$\Re$},
				ylabel={$\Im$},
				xmin=0, xmax=10,
				ymin=0, ymax=8,
				%					xtick distance=2,
				%					ytick distance=2,
				xtick=100,
				ytick=100,
				axis lines = middle,
				axis line style = Black,
				]
				\addplot[
				only marks,
				nodes near coords,
				point meta = explicit symbolic, 
				color=Black,
				mark=*,
				]
				coordinates {
					(7, 4) [$z_0$]
				};
				\draw[dashed,line width=1pt,Grey](axis cs: 7, 4) circle (2);
				\draw[-,draw=Grey]  (axis cs: 7,4) to ["$\epsilon$"] (axis cs: 5,4);
			\end{axis}
		\end{tikzpicture}
	\end{figure}

	\begin{definition}[Deleted Neighbourhood]
		A deleted neighbourhood is the set of all points $z$ with distance less than $\epsilon$ from a point $z_0$, not including $z_0$. That is, it is a neighbourhood of $z_0$ without $z_0$.
		$$\{z : \abs{z-z_0} < \epsilon, \ z \neq z_0 \}$$
	\end{definition}
	\begin{figure}[H]
		\centering
		\begin{tikzpicture}[remember picture]
			\begin{axis}[
				small,
				unit vector ratio=1 1 1,
				xlabel={$\Re$},
				ylabel={$\Im$},
				xmin=0, xmax=10,
				ymin=0, ymax=8,
				%					xtick distance=2,
				%					ytick distance=2,
				xtick=100,
				ytick=100,
				axis lines = middle,
				axis line style = Black,
				]
				\addplot[
				only marks,
				nodes near coords,
				point meta = explicit symbolic, 
				color=Black,
				mark=o,
				]
				coordinates {
					(7, 4) [$z_0$]
				};
				\draw[dashed,line width=1pt,Grey](axis cs: 7, 4) circle (2);
				\draw[-,draw=Grey]  (axis cs: 7,4) to ["$\epsilon$"] (axis cs: 5,4);;
			\end{axis}
		\end{tikzpicture}
	\end{figure}

	\begin{definition}[Interior Point]
		Let $S$ be a set. A point $z_0$ is an interior point of $S$ if $\exists \epsilon$ such that $\forall z$, $\abs{z-z_0} < \epsilon \implies z \in S$. That is, $z_0$ is an interior point of $S$ if it has a neighbourhood where all points in the neighbourhood is an element of $S$.
	\end{definition}
	\begin{figure}[H]
		\centering
		\begin{tikzpicture}[remember picture]
			\begin{axis}[
				small,
				unit vector ratio=1 1 1,
				xlabel={$\Re$},
				ylabel={$\Im$},
				xmin=0, xmax=10,
				ymin=0, ymax=8,
				%					xtick distance=2,
				%					ytick distance=2,
				xtick=100,
				ytick=100,
				axis lines = middle,
				axis line style = Black,
				]
				\addplot[
				only marks,
				nodes near coords,
				point meta = explicit symbolic, 
				color=Black,
				mark=*,
				]
				coordinates {
					(7, 4) [$z_0$]
				};
				\draw[dashed,line width=1pt,Grey](axis cs: 7, 4) circle (1);
				\draw[line width=1pt, Black](axis cs: 5.2, 3.5) circle (3);
				\draw[-,draw=Grey]  (axis cs: 7,4) to ["$\epsilon$"] (axis cs: 6,4);
				\draw[]  (axis cs: 5,2) to ["$S$"] (axis cs: 5,2);
			\end{axis}
		\end{tikzpicture}
	\end{figure}

	\begin{definition}[Exterior Point]
		Let $S$ be a set. A point $z_0$ is an exterior point of $S$ if $\exists \epsilon$ such that $\forall z$, $\abs{z - z_0}<\epsilon \implies z \notin S$. That is, $z_0$is an exterior point of $S$ if it has neighbourhood that does not contain any element of $S$. 
	\end{definition}
	\begin{figure}[H]
		\centering
		\begin{tikzpicture}[remember picture]
			\begin{axis}[
				small,
				unit vector ratio=1 1 1,
				xlabel={$\Re$},
				ylabel={$\Im$},
				xmin=0, xmax=10,
				ymin=0, ymax=8,
				%					xtick distance=2,
				%					ytick distance=2,
				xtick=100,
				ytick=100,
				axis lines = middle,
				axis line style = Black,
				]
				\addplot[
				only marks,
				nodes near coords,
				point meta = explicit symbolic, 
				color=Black,
				mark=*,
				]
				coordinates {
					(9, 5) [$z_0$]
				};
				\draw[dashed,line width=1pt,Grey](axis cs: 9, 5) circle (1);
				\draw[line width=1pt, Black](axis cs: 5.2, 3.5) circle (3);
				\draw[-,draw=Grey]  (axis cs: 9,5) to ["$\epsilon$"] (axis cs: 8,5);
				\draw[]  (axis cs: 5,2) to ["$S$"] (axis cs: 5,2);
			\end{axis}
		\end{tikzpicture}
	\end{figure}

	\begin{definition}[Boundary Point]
		Let $S$ be a set. A point $z_0$ is a boundary point of $S$ if $\forall \epsilon$, $\exists z \in S, z' \notin S$, such that $\abs{z - z_0} \epsilon$ and $\abs{z' - z_0} < \epsilon$. That is, for all neighbourhoods of $z_0$ there exists a point that is in $S$ and a point not in $S$.
	\end{definition}
	\begin{figure}[H]
		\centering
		\begin{tikzpicture}[remember picture]
			\begin{axis}[
				small,
				unit vector ratio=1 1 1,
				xlabel={$\Re$},
				ylabel={$\Im$},
				xmin=0, xmax=10,
				ymin=0, ymax=8,
				%					xtick distance=2,
				%					ytick distance=2,
				xtick=100,
				ytick=100,
				axis lines = middle,
				axis line style = Black,
				]
				\addplot[
				only marks,
				nodes near coords,
				point meta = explicit symbolic, 
				color=Black,
				mark=*,
				]
				coordinates {
					(5.2, 6.5) [$z_0$]
				};
				\draw[dashed,line width=1pt,Grey](axis cs: 5.2, 6.5) circle (1);
				\draw[line width=1pt, Black](axis cs: 5.2, 3.5) circle (3);
				\draw[-,draw=Grey]  (axis cs: 5.2,6.5) to ["$\epsilon$"] (axis cs: 4.2,6.5);
				\draw[]  (axis cs: 5,2) to ["$S$"] (axis cs: 5,2);
			\end{axis}
		\end{tikzpicture}
	\end{figure}

	Note: A boundary point of $S$ may or may not be in $S$.

	\begin{definition}[Boundary of a Set]
		A boundary of a set $S$ is the set of all boundary points of $S$. The set containing all boundary points of $S$.
		$$\{z_0 : \forall \epsilon \exists z\in S, z'\notin S(\abs{z-z_0}<\epsilon \land \abs{z'-z_0}<\epsilon )\}$$ 
	\end{definition}

	\begin{definition}[Open Set]
		A set that does not contain any boundary points. 
	\end{definition}

	\begin{theorem}
		Set $S$ is open $\iff$ $\forall s \in S$, $s$ is an interior point of $S$ 
	\end{theorem}
	\begin{proof}
		\textcolor{Grey}{
		\underline{$\implies$}:
		Suppose $S$ is open $\nRightarrow$ $\forall s \in S$, $s$ is an interior point of $S$, for contradiction. That is, $\exists s \in S$ that is either a boundary point or an exterior point. $s \in S$ implies $s$ is not an exterior point of $S$, so $s$ has to be a boundary point of $S$. This contradicts that $S$ is an open set. 
		$$S \text{ is open } \implies \forall s \in S (s \text{ is an interioir point of }S)$$
		\underline{$\impliedby$:}
		\begin{align*}
			&\forall s \in S (s \text{ is an interior point of S}) \\
			&\implies \forall s' \forall \epsilon (\abs{s'-s} < \epsilon \implies s' \in S)\\
			&\implies S \text{ does not contain boundary points} \implies S \text{ is open}
		\end{align*}
		}
	\end{proof}

	A set can be neither open or closed. Consider the set $S = \{z : 0 < \abs{z} \leq 1\}$. $S$ is not closed since it does not contain the boundary point $0$, and it is not open since it contains boundary points where $\abs{z} = 1$. The set $\mathbb{C}$ is both open and closed since it has no boundary points. 
	
	\begin{definition}[Closed Set]
		A set that contains all of its boundary points. 
	\end{definition}

	\begin{definition}[Closure of a Set]
		Let $S$ be a set. The closure of S is a closed set containing all points of $S$ and all boundary points of $S$. 
	\end{definition}

	\begin{definition}[Connected Set]
		An opens set $S$ is connected if $\forall z_1, z_2 \in S$, $z_1$ and $z_2$ can be connected by a polygonal line lying within $S$.
	\end{definition}
	\begin{figure}[H]
		\centering
		\begin{tikzpicture}[remember picture]
			\begin{axis}[
				small,
				unit vector ratio=1 1 1,
				xlabel={$\Re$},
				ylabel={$\Im$},
				xmin=0, xmax=10,
				ymin=0, ymax=8,
				%					xtick distance=2,
				%					ytick distance=2,
				xtick=100,
				ytick=100,
				axis lines = middle,
				axis line style = Black,
				]
				\addplot[
				only marks,
				nodes near coords,
				point meta = explicit symbolic, 
				color=Black,
				mark=*,
				]
				coordinates {
					(3, 3) [$z_1$]
					(6, 6) [$z_2$]
				};
				\draw[line width=1pt, Black](axis cs: 5, 4) circle (1);
				\draw[line width=1pt, Black](axis cs: 5, 4) circle (3.5);
				\draw[-,draw=Grey]  (axis cs: 3,3) -- (axis cs: 4,5);
				\draw[-,draw=Grey]  (axis cs: 4,5) -- (axis cs: 6,6);
				\draw[]  (axis cs: 5,2) to ["$S$"] (axis cs: 5,2);
			\end{axis}
		\end{tikzpicture}
	\end{figure}

	\begin{definition}[Polygonal Line]
		A finite set of line segments joined end to end. 
	\end{definition}

	\begin{definition}[Domain]
		A nonempty connected set. 
	\end{definition}
	Note: All neighbourhoods are domains. 

	\begin{definition}[Region]
		A domain with none, some, or all of its boundary points. 
	\end{definition}
	
	\begin{definition}[Closed Region]
		A domain with all of its boundary points. 
	\end{definition}
	
	\begin{definition}[Bounded Set/Region]
		A set $S$ is bounded if $\exists R = \abs{z} > 0$ such that $\forall s \in S$, $\abs{s} < R$. That is, $S$ is bounded if $\forall s \in S$, $s$ is contained in some circle of radius $R$ centred at the origin. 
	\end{definition}

	\begin{definition}[Closed Region]
		A bounded and closed region.
	\end{definition}

	\begin{definition}[Accumulation/Limit Point]
		A point $z_0$ is a accumulation point of a set $S$ if all deleted neighbourhood of $z_0$ contains an element of $S$. 
		$$\forall \epsilon \exists s \in S (s \neq z_0 \land \abs{z-s} < \epsilon)$$
	\end{definition}
	Note: Unlike a boundary point, an accumulation point does not require that all neighbourhood of $z_0$ contain an element not in $S$.
	
	\begin{theorem}
		Set $S$ is closed $\iff$ $\forall$ accumulation points $z_0$ of $S$, $z_0 \in S$
	\end{theorem}
	\begin{proof}
		\underline{$\implies$:}
		Let $S$ is closed and $z_0$ is an accumulation point of a set $S$ where $z_0 \notin S$ for contradiction. If $\exists z_0 \notin S$, then $z_0$ is a boundary point of $S$. Contradicts closed set contains all boundary points. 
		
		\textcolor{Grey}{
		\underline{$\impliedby$:}
		Suppose all accumulation points of $S$ are elements of $S$ but $S$ is not closed for contradiction. Then $S$ does not contain one or more boundary points. Suppose $z_0$ is a boundary point of $S$ that is not in $S$. Then $\forall \epsilon \ \exists s\in S$ where $\abs{s-z_0} < \epsilon$, so by considering the deleted neighbourhood of $z_0$, this makes $z_0$ an accumulation point of $S$. This contradicts that all accumulation points of $S$ is in $S$. 		
		}
	\end{proof}

	\chapter{Analytic Functions} \label{Analytic Functions Chapter - Complex}
	\section{Functions as mappings} \label{Functions as Mappings Section - Complex}
	A function $f: S \rightarrow S'$ is a function that maps elements from $S$ to elements on $S'$. The value of $f$ at $z$ is denoted $f(z)$ and the set $S$ is the domain of $f$ while $S'$ is the image of $f$. Recall \cref{Operations as Transformations Section - Complex}, a function can likewise be viewed as a transformation or mapping, that maps $z \in \operatorname{dom}(f) = S$ to values $z' \in \operatorname{img}(f) = S'$.
	
	\begin{definition}[Range]
		Let $f$ be a function with domain $S$ and image $S'$. The range of $f$ is the entire image of $S$.
	\end{definition}

	Note: Image is a subset of range, and can be a single point or a set of points.

	\begin{definition}[Inverse Range]
		The set of all points $s \in S$ with the value $f(s) = s'$ for some $s' \in S'$.
		$$\{s : f(s) = s', \ s' \in S'\}$$
	\end{definition}
	
	Note: The domain of a function is often a domain, but it does not need to be a domain. 
	
	We will consider functions $f: S \rightarrow S'$ where both $S, S' \subseteq \mathbb{C}$. For such functions we can break it into a two real valued functions: 
	\begin{align*}
		f(z) &= u(x, y) + i v(x, y)
			& \operatorname{dom}(u) \subseteq \mathbb{R}, \operatorname{dom}(v) \subseteq \mathbb{R} \\
			&= u(r, \theta) + i v(r, \theta)
	\end{align*}
	Recall that a real-valued function is a function with a domain that is a subset of $\mathbb{R}$ (\cref{Real-Valued Function Definition - Real Analysis}).
	If $\forall z$, $v(x, y) = 0$, then $f$ is called a real-valued function of a complex variable. 
	
	\begin{definition}[Polynomial]
		\label{Polynomial Definition - Complex}
		Let $a_i \in \mathbb{C}$, $0 \leq i \leq n$ where $i, n \in \mathbb{N}\cup\{0\}$. If $a_n \neq 0$, then a polynomial of degree n is
		$$P(z) = a_0 + a_1 z + a_2 z^2 + \ldots + a_n z^n = \sum_{i=0}^n a_i z^i$$
	\end{definition}

	\begin{definition}[Rational Functions]
		\label{Rational Functions Definition - Complex}
		Let $P(z)$ and $Q(z)$ are polynomials, then rational functions are quotients:
		$$\frac{P(z)}{Q(z)}$$
		Defined for all $z$ where $Q(z) \neq 0$.
	\end{definition}
	
	\begin{definition}[Multiple-Valued Function]
		\label{Multiple-Valued Function Definition - Complex}
		Let $f$ be a function and $z \in \operatorname{dom}(f)$. $f$ is a multiple-valued function if it assigns more than one value to a point $z$.
	\end{definition}
	``When multiple-valued functions are studied, usually just one of the possible values assigned at each point is taken, in a systematic manner and a (single-valued) function is constructed from the multiple-valued one'' - Brown and Churchill \cite{Brown.J;Churchill.R-Complex-Variables-2014}
	
	What this means that for $z \in \mathbb{C}$ a function $f$ assigns $u(z)$ and $v(z)$ to to $z$. By taking just $u$ or $v$, we create a single-valued function from a multiple-valued function. 
	
	\begin{example}[$f(z) = z^2$]
		\begin{align*}
			f(z) &= z^2  = x^2 - y^2 + i2xy \\
				&\implies u(x,y) = x^2 - y^2 \qquad v(x,y) = 2xy
		\end{align*}
		By setting $u = x^2 - y^2 = c_1$ where $c_1 \in \mathbb{R}_{>0}$ we can see that
		\begin{align*}
			u &= x^2 - y^2 = c_1 & v &= 2xy = \pm 2y \sqrt{y^2 + c_1}
		\end{align*}
		This tells us that in the complex plane of $u$ and $v$, if we fix $u$ to a constant $c_1$ and move along $v = \pm 2 y \sqrt{y^2 + c_1}$ by incrementing $y$ we draw out two hyperbolas in the complex plane of $x$ and $y$.  This means that the function $f(z) = z^2$ takes points on hyperbolas the complex plane of $x$ and $y$ and translates them onto a vertical line in the complex plane of $u$ and $v$ where $u$ is a constant.
		\begin{figure}[H]
			\centering
			\begin{tikzpicture}[remember picture, baseline=(current bounding box.center)]
				\begin{axis}[
					width=8.9cm,
					unit vector ratio=1 1 1,
					xlabel={$x$},
					ylabel={$y$},
					xmin=-10, xmax=10,
					ymin=-10, ymax=10,
					axis lines = middle,
					xticklabels={}, yticklabels={},
					grid=none,
					major tick style={draw=none},
					minor tick style={draw=none},
					legend pos = south west,
					]
					
					\addplot [domain=2.23:10, samples=1000, color=blue,]
					{(x^2 - 5)^(0.5)};
					\addlegendentry{$x^2 - y^2 = c_1$}
					
					\addplot [domain=2.23:10, samples=1000, color=blue,]
					{-(x^2 - 5)^(0.5)};
					
					\addplot [domain=-10:-2.23, samples=1000, color=blue,]
					{(x^2 - 5)^(0.5)};
					
					\addplot [domain=-10:-2.23, samples=1000, color=blue,]
					{-(x^2 - 5)^(0.5)};
					
					\draw[->, draw=Blue] (axis cs:6.02,-5.6) -- (axis cs:5.5,-5);
					\draw[->, draw=Blue] (axis cs:-6.02,5.6) -- (axis cs:-5.5,5);
				\end{axis}
%				\path (current bounding box.north east) -- 
%				(current bounding box.south east) coordinate[midway] (2BL);
			\end{tikzpicture}
			{$\Large \xrightarrow{\mathmakebox[2cm] {f(z)=z^2}}$}
%			\hspace*{2.8cm}
			\begin{tikzpicture}[remember picture, baseline=(current bounding box.center)]
				\begin{axis}[
					width=8.9cm,
					unit vector ratio=1 1 1,
					xlabel={$u$},
					ylabel={$v$},
					xmin=-10, xmax=10,
					ymin=-10, ymax=10,
					axis lines = middle,
					xticklabels={}, yticklabels={},
					grid=none,
					major tick style={draw=none},
					minor tick style={draw=none},
					legend pos = south west,
					]
					
					\addplot [mark=none, color=Blue] coordinates {(5, -10) (5, 10)};
					\addlegendentry{$u = c_1, \ v = \pm 2y \sqrt{y^2 + c_1}$}
					
					\draw[->, draw=Blue] (axis cs:5,-10) -- (axis cs:5,-5);
%					node[midway,above right,inner sep=2pt,font={\footnotesize}]{$\theta$};
				\end{axis}
%				\path (current bounding box.north west) -- 
%				(current bounding box.south west) coordinate[midway] (2BR);
			\end{tikzpicture}
%			\tikz[overlay,remember picture]{\draw[->, very thick] 
%				($(2BL)+(0.5, 0)$) -- ($(2BR)+(-0.5,0)$)
%				node[midway,above,text width=2.5cm]{$f(z) = z^2$};}
		\end{figure}
		Likewise if we set $v = c_2$ where $c_2 \in \mathbb{R}_{>0}$, we get:
		\begin{align*}
			u &= x^2 - \frac{c_2^2}{4x^2} 
			& v &= 2xy = c_2
		\end{align*}
		Taking the limits: 
		\begin{align}
			\lim_{x\rightarrow 0^+} u &= -\infty 
				& \lim_{x \rightarrow \infty, x>0} u &= \infty \\
			\lim_{x\rightarrow -\infty, x<0} u &= \infty 
				& \lim_{x \rightarrow 0^-} u &= -\infty 
		\end{align}
		Equation 11.1 tells us as $x$ goes from $0$ to $\infty$, $u$ moves from $-\infty$ to $\infty$, which corresponds to the hyperbola in the first quadrant of the $x$$y$ complex plane. Similarly for equations 11.2.
		\begin{figure}[H]
			\centering
			\begin{tikzpicture}[remember picture, baseline=(current bounding box.center)]
				\begin{axis}[
					width=8.9cm,
					unit vector ratio=1 1 1,
					xlabel={$x$},
					ylabel={$y$},
					xmin=-10, xmax=10,
					ymin=-10, ymax=10,
					axis lines = middle,
					xticklabels={}, yticklabels={},
					grid=none,
					major tick style={draw=none},
					minor tick style={draw=none},
					legend pos = south west,
					]
					\addplot [domain=0.01:10, samples=50, color=Red,]
					{5/(2*x)};
					
					\addplot [domain=-10:-0.01, samples=50, color=Red,]
					{5/(2*x)};
					\addlegendentry{$2xy = c_2$};
					
					\draw[->, draw=Red] (axis cs:0.485,5.1) -- (axis cs:0.51,5);
					\draw[->, draw=Red] (axis cs:-0.485,-5.1) -- (axis cs:-0.51,-5);
				\end{axis}
			\end{tikzpicture}
			{$\Large \xrightarrow{\mathmakebox[2cm] {f(z)=z^2}}$}
			\begin{tikzpicture}[remember picture, baseline=(current bounding box.center)]
				\begin{axis}[
					width=8.9cm,
					unit vector ratio=1 1 1,
					xlabel={$u$},
					ylabel={$v$},
					xmin=-10, xmax=10,
					ymin=-10, ymax=10,
					axis lines = middle,
					xticklabels={}, yticklabels={},
					grid=none,
					major tick style={draw=none},
					minor tick style={draw=none},
					legend pos = south west,
					]
					
					\addplot [domain=-10:10, samples=10, color=Red,]
					{5};
					\addlegendentry{$u = x^2 - c_2^2/(4x^2), \ v=c_2$}
					
					\draw[->, draw=Red] (axis cs:-10,5) -- (axis cs:-5,5);
					%					node[midway,above right,inner sep=2pt,font={\footnotesize}]{$\theta$};
				\end{axis}
			\end{tikzpicture}
		\end{figure}
		If we look at $f$ using the polar representation, we get $f(z) = r^2 e^{i2\theta}$. This tells us $\forall r \geq 0$, $r \mapsto r^2 = \rho \geq 0$, and $\forall \theta$, $\theta \mapsto \phi = 2\theta$. It is worth noting that mapping of points between $0 \leq 0 < 2\pi$ is not one-to-one, since points in $0 \leq \theta < \pi$ and points in $\pi \leq \theta < 2\pi$ both get mapped to $0 \leq \phi < 2\pi$. 
	\end{example}

	\section{Limits} \label{Limits Section - Complex}
	\begin{definition}[Limit]
		\label{Limit Definition - Complex}
		Let $z, z_0, w_0 \in \mathbb{C}$ and $f$ be a function. We say $f(z)$ has limit $w_0$ as $z$ approaches $z_0$ if: 
		\begin{align*}
			\forall \epsilon \exists \delta [0<\abs{z-z_0} < \delta \implies \abs{f(z) - w_0} < \epsilon]
		\end{align*} 
		We then denote: $\lim_{z \rightarrow z_0} f(z) = w_0$
	\end{definition}
	This tells us that $\lim_{z \rightarrow z_0} f(z) = w_0$ if some deleted neighbourhood $\abs{z-z_0} < \delta$ corresponds to a neighbourhood $\abs{f(z) - w_0} < \epsilon$. Note that the mapping of all points $z$ in $\abs{z-z_0} < \delta$ to $\abs{f(z) - w_0} < \epsilon$ need not be subjective. It just needs to be mapped less than distance $\epsilon$ from $w_0$.
	
	Note: \Cref{Limit Definition - Complex} allows us to verify if a limit exists, but it is not a method for determining a limit. 
	
	\begin{figure}[H]
		\centering
		\begin{tikzpicture}[remember picture, baseline=(current bounding box.center)]
			\begin{axis}[
				width=8cm,
				unit vector ratio=1 1 1,
				xlabel={$x$},
				ylabel={$y$},
				xmin=-10, xmax=10,
				ymin=-1, ymax=10,
				ticks = none,
				axis lines = middle,
				axis line style = Black,
				]
				\addplot[
				only marks,
				nodes near coords,
				point meta = explicit symbolic, 
				color=Black,
				mark=o,
				]
				coordinates {
					(8, 8) [$z_0$]
				};
				\draw[dashed, line width=1pt,Grey](axis cs: 8, 8) circle (2);
				\draw[-,draw=Black]  (axis cs: 8,8) to ["$\delta$"] (axis cs: 6,8);
			\end{axis}
			\path (current bounding box.north east) -- 
			(current bounding box.south east) coordinate[midway] (2BL);
		\end{tikzpicture}
		{$\Large \xrightarrow{\mathmakebox[2cm]f}$}
%		\hspace*{2.8cm}
		\begin{tikzpicture}[remember picture, baseline=(current bounding box.center)]
			\begin{axis}[
				width=8cm,
				unit vector ratio=1 1 1,
				xlabel={$v$},
				ylabel={$u$},
				xmin=-10, xmax=10,
				ymin=-1, ymax=10,
				ticks = none,
				axis lines = middle,
				axis line style = Black,
				]
				\addplot[
				only marks,
				nodes near coords,
				point meta = explicit symbolic, 
				color=Black,
				mark=*,
				]
				coordinates {
					(-4, 5) [$w_0$]
				};
				\draw[dashed, line width=1pt,Grey](axis cs: -4, 5) circle (3);
				\draw[-,draw=Black]  (axis cs: -4,5) to ["$\epsilon$"] (axis cs: -7,5);
			\end{axis}
			\path (current bounding box.north west) -- 
			(current bounding box.south west) coordinate[midway] (2BR);
		\end{tikzpicture}
%		\tikz[overlay,remember picture]{\draw[->, very thick] 
%			($(2BL)+(0.5, 0)$) -- ($(2BR)+(-0.5,0)$)
%			node[midway,above]{$f$};} 
		\label{Limit Definition Figure - Complex}
	\end{figure}
	
	\begin{theorem}[Uniqueness of Limits]
		\label{Uniqueness of Limits Theorem - Complex}
		Suppose the limit of $f$ at $z_0$ exists, then it is unique.
	\end{theorem}
	\begin{proof}
		Suppose two limits of $f$ at $z_0$ exists for contradiction.
		\begin{align*}
			&[\lim_{z \rightarrow z_0} f(z) = w_0] \land [\lim_{z \rightarrow z_0} f(z) = w_1] \\
			&\implies [0 < \abs{z - z_0} < \delta_0 \implies \abs{f(z) - w_0} < \epsilon] 
				\land [0 < \abs{z - z_0} < \delta_0 \implies \abs{f(z) - w_0} < \epsilon] \\
			\\
			&w_1 - w_0 = [f(z) - w_0] + [w_1 - f(z)]\\
			&\implies \abs{w_1 - w_0} = \abs{[f(z) - w_0] + [w_1 - f(z)]} 
				\leq \abs{f(z) - w_0} + \abs{f(z) - w_1} 
		\end{align*}
		Now choosing $\delta = \min\{\delta_1, \delta_2\}$, we get:
		$$\abs{w_1 - w_0} < \epsilon + \epsilon = 2 \epsilon$$
		Choosing $\epsilon$ to be arbitrary small, we end up with: 
		$$w_1 - w_0 = 0 \implies w_1 = w_0$$
	\end{proof}

	\Cref{Limit Definition - Complex} requires that $f$ be defined at all points in the deleted neighbourhood of $z_0$. That is, $z_0$ is interior to the region which $f$ is defined. We can extend the definition by agreeing that $0<\abs{z-z_0} < \delta \implies \abs{f(z) - w_0} < \epsilon$ also holds for $z$ that lie in the region where $f$ is defined and the deleted neighbourhood of $z_0$.  That is $f(z_0)$ need not be defined for a limit at $z_0$ to exist.
	
	\begin{example}
		Show $(f(z) = iz/2) \land (\abs{z} < 1) \implies \lim_{z\rightarrow 1} f(z) = i/2$.
		
		We can see that we have restricted the domain of $f$ to the region $\abs{z} < 1$, this puts $z = 1$ right at the boundary of the domain of definition of $f$.
		\begin{align*}
			\abs{z} < 1 \implies& \abs{f(z) - \frac{i}{2}} = \abs{\frac{iz}{2} - \frac{i}{2}} = \frac{\abs{z - 1}}{2} \\
			\implies& \forall z \forall \epsilon \exists \delta \left[ 0 < \abs{z - 1} < \delta = 2\epsilon \implies \abs{f(z) - \frac{i}{2}} < \epsilon \right] \\
			\implies& \lim_{z \rightarrow 1} f(z) = \frac{i}{2}
		\end{align*}
	\end{example}
	
	This highlights the fact that if the limit exists, then $z$ is allowed to approach $z_0$ from any arbitrary direction. 
	
	\begin{example}
		Limit of $f(z) = z /\bar{z}$ does not exist at $z = 0$
		
		Consider $\lim_{z \rightarrow 0} f(z)$. Let us approach the limit from the $x$-axis and the $y$-axis. 
		\begin{align*}
			\lim_{z = (x, 0) \rightarrow 0} f(z) &= \frac{x + i0}{x - i0} = 1
			& \lim_{z = (0, y) \rightarrow 0} f(z) &= \frac{0 + iy}{0 - iy} = -1
		\end{align*} 
		We end up with two different limits. As limits are unique, we conclude that $\lim_{z\rightarrow0} f(z)$ does not exist. 
	\end{example}

	\subsection{Limit Theorems} \label{Limit Theorems Subsection - Complex}
	
	\begin{theorem}
		\label{Limit of f=u+iv Theorem - Complex}
		Consider $f(z) = u(x, y) + iv(x,y)$. Let $z_0 = x_0 + iy_0$ and $w_0 = u_0 + iv_0$.
		\begin{align*}
			\left[\lim_{(x,y) \rightarrow (x_0, y_0)} u(x,y) = u_0 \right] \land
			\left[\lim_{(x,y) \rightarrow (x_0, y_0)} v(x,y) = v_0 \right]
			\iff
			\lim_{z \rightarrow z_0} f(z) = w_0
		\end{align*}
	\end{theorem}
	\begin{proof}
		\underline{$\implies$:}
		
		By definition: 
		\begin{align}
			\label{Limit of f=u+iv Theorem Proof Eqn 1 - Complex}
			&\left[\lim_{(x,y) \rightarrow (x_0, y_0)} u(x,y) = u_0 \right] \land
			 \left[\lim_{(x,y) \rightarrow (x_0, y_0)} v(x,y) = v_0 \right] \\ \nonumber
			&\implies \forall \epsilon \exists \delta_1, \delta_2
			 \left[\left(0< \sqrt{(x-x_0)^2 + (y-y_0)^2} < \delta_1 \implies \abs{u-u_0} < \frac{\epsilon}{2}\right) \right.\\ \nonumber
			&\qquad \qquad \qquad \land \left.
			 \left(0< \sqrt{(x-x_0)^2 + (y-y_0)^2} < \delta_2 \implies \abs{v-v_0} < \frac{\epsilon}{2}\right)
			 \right]
		\end{align}
		Triangle inequality for the distance between points:
		\begin{align*}
			\abs{(u+iv) - (u_0 - iv_0)} &= \abs{(u-u_0) + i(v-v_0)} \leq \abs{u-u_0} + \abs{v-v_0} \\
			\sqrt{(x-x_0)^2 + (y-y_0)^2} &= \abs{(x-x_0) + i(v-v_0)} = \abs{(x+iy) - (x_0 - iy_0)}
		\end{align*}
		Let $\delta = \min\{\delta_1, \delta_2\}$, it follows from 
		\cref{Limit of f=u+iv Theorem Proof Eqn 1 - Complex}: 
		$$0 < \abs{(x+iy) - (x_0 + iy_0)} < \delta 
			\implies \abs{(u+iv)-(u_0 - iv_0)} < \frac{\epsilon}{2} + \frac{\epsilon}{2} = \epsilon$$
		Thus, $\lim_{z \rightarrow z_0} f(z) = w_0$.
		
		\underline{$\impliedby$:}
		
		Suppose $\lim_{z \rightarrow z_0} f(z) = w_0$.
		\begin{align}
			\label{Limit of f=u+iv Theorem Proof Eqn 2 - Complex}
			&\lim_{z \rightarrow z_0} f(z) = w_0 \\ \nonumber
			&\implies \forall \epsilon \exists \delta>0 [\abs{(x+iy)-(x_0-iy_0)}<\delta \implies \abs{(u+iv) - (u_0+iv_0)}<\epsilon]
		\end{align}
		By the triangle inequality: 
		\begin{align*}
			\abs{u-u_0} &\leq \abs{(u-u_0) + i(v-v_0)} = \abs{(u+iv) - (u_0 + iv_0)} \\
			\abs{v-v_0} &\leq \abs{(u-u_0) + i(v-v_0)} = \abs{(u+iv) - (u_0 + iv_0)} 
		\end{align*}
		$$\abs{(x+iy) - (x_0 + iy_0)} = \abs{(x-x_0) + i(y-y_0)} = \sqrt{(x-x_0)^2 + (y-y_0)^2}$$
		Thus, it follows from the inequalities in \cref{Limit of f=u+iv Theorem Proof Eqn 2 - Complex}:
		\begin{align*}
			&0 < \sqrt{(x-x_0)^2 + (y-y_0)^2} < \delta  \\
			&\implies [\abs{u-u_0} < \epsilon ]\land [\abs{v-v_0} < \epsilon] \\
			&\implies \left[\lim_{(x,y) \rightarrow (x_0, y_0)} u(x,y) = u_0 \right] \land
			\left[\lim_{(x,y) \rightarrow (x_0, y_0)} v(x,y) = v_0 \right]
		\end{align*}
	\end{proof}

	\begin{theorem}
		Suppose
		\begin{align*}
		\left[\lim_{z\rightarrow z_0} f(z) = w_0 \right] \land \left[\lim_{z\rightarrow z_0} F(z) = W_0 \right]
		\end{align*}
		Then
		\begin{align*}
%			\label{Limit of f(z)+F(z) Theorem Equation - Complex}
			\lim_{z \rightarrow z_0} [f(z) + F(z)] &= w_0 + W_0 & \\
%			\label{Limit of f(z)F(z) Theorem Equation - Complex} 
			\lim_{z \rightarrow z_0} [f(z)F(z)] &= w_0 W_0  & \\
%			\label{Limit of f(z)/F(z) Theorem Equation - Complex}
			\lim_{z \rightarrow z_0} \frac{f(z)}{F(z)} &= \frac{w_0}{W_0} & W_0 \neq 0
		\end{align*}
		\label{Limit of f(z)+F(z) f(z)F(z) and f(z)/F(z) Theorem - Complex}
	\end{theorem}
	\begin{proof}
		Let: 
		\begin{align*}
			f(z) &= u(x,y) + iv(x,y) & F(z) &= U(x,y) + iV(x,y) 
		\end{align*}
		\begin{align*}
			z_0 &= x_0 + iy_0 	& w_0 &= u_0 + iv_0 & W_0 &= U_0 + iV_0
		\end{align*}
		\underline{$\lim_{z \rightarrow z_0} [f(z) + F(z)] = w_0 + W_0$}
		
		\textcolor{Grey}{
		From \Cref{Limit of f=u+iv Theorem - Complex}:
		\begin{align*}
			&f(z) + F(z) = (u + U) + i(v + V) \\
			&\implies \lim_{(x,y) \rightarrow (x_0, y_0)} f(z)F(Z) = (u_0 + U_0) + i(v_0 + V_0) = w_0 + W_0
		\end{align*}
		}
		
		\underline{$\lim_{z \rightarrow z_0} [f(z)F(z)] = w_0 W_0$}
		
		From \Cref{Limit of f=u+iv Theorem - Complex}:
		\begin{align*}
			&f(z)F(z) = (uU - vV) + i(vU + uV) \\
			&\implies \lim_{(x,y) \rightarrow (x_0, y_0)} f(z)F(Z) = (u_0 U_0 - v_0 V_0) + i(v_0 U_0 + u_0 V_0) = w_0 W_0
		\end{align*}
		
		\underline{$\lim_{z \rightarrow z_0} \frac{f(z)}{F(z)} = \frac{w_0}{W_0}$ if $W_0 \neq 0$}
		
		\textcolor{Grey}{
		From \Cref{Limit of f=u+iv Theorem - Complex}:
		\begin{align*}
			&\frac{f(z)}{F(z)} = \frac{u+iv}{U+iV} 
			\implies \lim_{(x, y) \rightarrow (x_0, y_0)} \frac{f(z)}{F(z_0)} = \frac{u_0 + v_0}{U_0 + iV_0} = \frac{w_0}{W_0}
		\end{align*}
		}
	\end{proof}

	\begin{corollary}
		\label{Limits of f=u+iv Theorem Corollary - Complex}
		Let $c$ be a constant, $z, z_0 \in \mathbb{C}$, and $P(z)$ be a polynomial. Then
		\begin{align*}
			\lim_{z \rightarrow z_0} c &= c & \lim_{z \rightarrow z_0} z &= z_0 &
			\lim_{z \rightarrow z_0} z^n &= z_0^n & n \in \mathbb{N}
		\end{align*}
		\begin{align*}
			\lim_{z \rightarrow z_0} P(z) = P(z_0)
		\end{align*}
	\end{corollary}

	\begin{observation}
		It is surprisingly quick that Brown and Churchill went from $\epsilon - \delta$ proofs straight to proving with limits. This is different to the approach in Sequences of Limits Theorem for Sequences Section by Kennith A. Ross. \cite{Ross.K-Elementary-Analysis-2013}. (\Cref{Limit Theorems Subsection - Real Analysis})
	\end{observation}
	\begin{question}
		It might be possible use a series approach to prove limit theorems for $z \in \mathbb{C}$ by having separate series for $x$ and $y$ (real and imaginary components of $z$), or a series in the form of $s_n = (x_n, y_n)$. Which would be the proper approach?
	\end{question}
	
	\subsection{Limits of Points at Infinity} \label{Limits of Points at Infinity Subsection - Complex}
	
	\begin{definition}[Extended Complex Plane]
		The complex plane union with the points at infinity: 
		$$\mathbb{C} \cup \{\pm \infty, \pm i \infty\}$$
	\end{definition}
	
	\begin{definition}[Riemann Sphere]
		A unit sphere centred at the origin of the complex plane, which is consequently bisected by the complex plane.
		\label{Riemann Sphere Definition - Complex}
	\end{definition}

	\begin{definition}[Stereographic Projection]
		Consider the Riemann Sphere. Let $N$ be the northern point of the sphere (the point on the sphere above the origin of the complex plane) and $z$ be any point in the complex plane. Let $l$ be a line that goes through $N$ and $z$, then $l$ will intersect the Riemann Sphere. Let $P$ be the point where $l$ intersects the Riemann Sphere. If we let $N$ correspond to the points at infinity, then there is a one-to-one correspondence between points on the sphere and the points on the extended complex plane. This correspondence is called the Stereographic Projection. (\Cref{Riemann Sphere and Stereographic Projection Figure - Complex})
		\label{Stereographic Projection Definition - Complex}
	\end{definition}

	\begin{figure}[H]
		\centering
		\begin{tikzpicture}
			\begin{axis}[
%				title = Riemann Sphere with Stereographic Projection,
				axis equal,
				width=20cm,
				axis lines = center,
				xlabel = {$x$},
				ylabel = {$y$},
				zlabel = {$h$},
				xmin=-3, xmax=3,
				ymin=-3, ymax=3,
				zmin=-2, zmax=2,
				ticks = none,
				axis lines = middle,
				axis line style = Black,
%				enlargelimits=0.3,
				view/h=45,
				scale uniformly strategy=units only,
				]
				\addplot3[
				opacity = 0.2,
				surf,
				z buffer = sort,
				samples = 50,
				domain = 0:360,
				y domain = 0:180,
				]
				({cos(x)*sin(y)}, {sin(x)*sin(y)}, {cos(y)});
				
				\addplot3[
				samples = 50,
				domain = 0:360,
				y domain = 0:180,
				color = Black,
				]
				({cos(x)}, {sin(x)}, {0});
				
				\addplot3[
				only marks,
				nodes near coords,
				point meta = explicit symbolic, 
				color=Black,
				mark=*,
				]
				coordinates {
					(1.5, 2, 0) [$z$]
					(0, 0, 1) [$N$]
					(0.413793, 0.551724, 0.724138) [$P$]
				};
				\draw[thick,draw=Black] (axis cs: 0,0,1)--(axis cs: 1.5,2,0);
				\draw[dashed,draw=Grey] (axis cs: 1.5,0,0)--(axis cs: 1.5,2,0);
				\draw[dashed,draw=Grey] (axis cs: 0,2,0)--(axis cs: 1.5,2,0);
				\draw[dashed,draw=Grey] (axis cs: (0.413793, 0, 0)--(axis cs: (0.413793, 0.551724, 0);
				\draw[dashed,draw=Grey] (axis cs: (0, 0.551724, 0)--(axis cs: (0.413793, 0.551724, 0);
				\draw[dashed,draw=Grey] (axis cs: (0.413793, 0.551724, 0)--(axis cs: (0.413793, 0.551724, 0.724138);
			\end{axis}
		\end{tikzpicture}
	    \caption{Riemann Sphere and Stereographic Projection}
		\label{Riemann Sphere and Stereographic Projection Figure - Complex}
	\end{figure}
	
	The region outside the unit circle enveloped by the Riemann sphere corresponds to the upper hemisphere of the Riemann sphere, with the point $N$ deleted. $N$ corresponds to the points at infinity, since $l$ will be parallel to the complex plane. 
	
	\begin{definition}[Neighbourhood of $\infty$]
		The set: $\{\abs{z} > 1/\epsilon : \epsilon \in \mathbb{R}_{>0} \}$
	\end{definition}
	Note that since $\epsilon$ is a small positive number, $\abs{z} > 1/\epsilon$ corresponds to points far away from the unit circle, hence $P$ is close to $N$. 
	
	\textcolor{red}{Note: When referring to any point $z$, it is referring to a point in the finite plane. Points at infinity will be specifically mentioned.}
	
	\begin{theorem}
		\label{Properties of Limits at Infinity Theorem - Complex}
		Let $z_0, w_0 \in \mathbb{C}$, then
		\begin{align*}
			\lim_{z \rightarrow z_0} \frac{1}{f(z)} = 0
			&\implies \lim_{z \rightarrow z_0} f(z) = \infty \\
			\lim_{z \rightarrow 0} f\left(\frac{1}{z}\right) = w_0
			&\implies \lim_{z \rightarrow \infty} f(z) = w_0 \\
			\lim_{z \rightarrow 0} \frac{1}{f(1/z)} = 0 
			&\implies \lim_{z \rightarrow \infty} f(z) = \infty
		\end{align*}
	\end{theorem}
	\begin{proof}
		\underline{$\lim_{z \rightarrow z_0} \frac{1}{f(z)} = 0 
		\implies \lim_{z \rightarrow z_0} f(z) = \infty$}
		\begin{align*}
			\lim_{z \rightarrow z_0} \frac{1}{f(z)} = 0 
			&\implies \forall \epsilon \exists \delta>0 \left[\abs{z-z_0} < \delta \implies \abs{\frac{1}{f{z}} - 0} < \epsilon \right] \\
			&\implies \forall \epsilon \exists \delta>0 \left[\abs{z-z_0} < \delta \implies \abs{f(z)} > \frac{1}{\epsilon} \right]  \\
			&\implies \lim_{z \rightarrow z_0} f(z) = \infty 
		\end{align*}
	
		\underline{$\lim_{z \rightarrow 0} f\left(\frac{1}{z}\right) = w_0
		\implies \lim_{z \rightarrow \infty} f(z) = w_0$}
		\begin{align*}
			\lim_{z \rightarrow 0} f\left(\frac{1}{z}\right) = w_0
			&\implies \forall \epsilon \exists \delta>0 \left[ \abs{z-0} < \delta \implies \abs{f\left(\frac{1}{z}\right) - w_0} < \epsilon \right] \\
			&\implies \forall \epsilon \exists \delta>0 \left[ \abs{z} > \frac{1}{\delta} \implies \abs{f(z) - w_0} < \epsilon \right] \\
			&\implies \lim_{z \rightarrow \infty} f(z) = w_0
		\end{align*}
	
		\underline{$\lim_{z \rightarrow 0} \frac{1}{f(1/z)} = 0 
		\implies \lim_{z \rightarrow \infty} f(z) = \infty$}
		\begin{align*}
			\lim_{z \rightarrow 0} \frac{1}{f(1/z)} = 0 
			&\implies \forall \epsilon \exists \delta > 0 \left[ \abs{z - 0} < \delta \implies \abs{\frac{1}{f(1/z)} - 0} < \epsilon \right] \\
			&\implies \forall \epsilon \exists \delta > 0 \left[ \abs{z} > \frac{1}{\delta} \implies \abs{f(z)} > \frac{1}{\epsilon} \right] \\
			&\implies \lim_{z \rightarrow \infty} f(z) = \infty 
		\end{align*}
	\end{proof}
	Note: As $\delta$ goes to $0$, $1/\delta$ goes to $\infty$, hence $\abs{z}$ goes to $\infty$ if $\abs{z} > 1/\delta$. 
	\begin{observation}
		As expected, \cref{Properties of Limits at Infinity Theorem - Complex} is consistent if $z \in \mathbb{R}$. (Check: \Cref{Limit Theorems Subsection - Real Analysis}).
	\end{observation}
	
	\section{Continuity} \label{Continuinty Section - Complex}
	\begin{definition}[Continuous]
		\label{Continuous Function Definition - Complex}
		Let $f$ be a function. We say $f$ is continuous at all point $z_0 \in \mathbb{C}$ if it satisfies the following: 
		\begin{align*}
			\lim_{z \rightarrow z_0} f(z) \text{ exists } \land f(z_0) \text{ exists } \land
			\lim_{z\rightarrow z_0} f(z) &= f(z_0)
		\end{align*} 
	\end{definition}
	Note: 
	\begin{align*}
		&\lim_{z \rightarrow z_0} f(z) = f(z_0) \implies \lim_{z \rightarrow z_0} f(z) \text{ exists } \land f(z_0) \text{ exists } \\
		&\forall \epsilon \exists \delta>0 \left[\abs{z-z_0} < \delta \implies \abs{f(z) - f(z_0)} < \epsilon \right] \iff \lim_{z \rightarrow z_0} f(z) = f(z_0)
	\end{align*}
	
	\begin{definition}[Continuous at a Region]
		Let $f$ be a function, $R \subset \mathbb{C}$ be a region, and $z \in R$: 
		\begin{center}
			$f$ is continuous in $R \iff$ $\forall z \in R$($f$ is continuous)
		\end{center}
	\end{definition}
	
	\begin{theorem}
		Let $f(z)$ and $g(z)$ be continuous functions at $z_0 \in \mathbb{C}$. Then the following are also continuous at $z_0$:
		\begin{align*}
			f(z_0) + g(z_0) \qquad f(z_0)g(z_0) \qquad \frac{f(z_0)}{g(z_0)} \qquad g(z_0) \neq 0
		\end{align*}
		\label{Continuity of f(z)+g(z) f(z)g(z) and f(z)/g(z) Theorem - Complex}
	\end{theorem}
	\begin{proof}
		Consequence of \cref{Limit of f(z)+F(z) f(z)F(z) and f(z)/F(z) Theorem - Complex}.
	\end{proof}

	\begin{corollary}
		Let $P(z)$ be a polynomial, then $P(z)$ is continuous $\forall z \in \mathbb{C}$. That is $P(z)$ is continuous in the entire plane of $\mathbb{C}$.
		\label{Continuity of f(z)+g(z) f(z)g(z) and f(z)/g(z) Corollary - Complex}
	\end{corollary}
	\begin{proof}
		Consequence of \cref{Limits of f=u+iv Theorem Corollary - Complex}.
	\end{proof}
	
	\begin{observation}
		Both \cref{Continuity of f(z)+g(z) f(z)g(z) and f(z)/g(z) Theorem - Complex} and \cref{Continuity of f(z)+g(z) f(z)g(z) and f(z)/g(z) Corollary - Complex} rely on \cref{Continuous Function Definition - Complex}, which state for a function $f$ and point $z_0 \in \mathbb{C}$:
		$$\lim_{z \rightarrow z_0} f(z) \text{ exists } \implies f(z) \text{ is continuous at } z_0$$
		This is why the proofs cite the results of \cref{Limit of f(z)+F(z) f(z)F(z) and f(z)/F(z) Theorem - Complex} and \cref{Limits of f=u+iv Theorem Corollary - Complex}.
	\end{observation}

	\begin{theorem}
		Let $f(z)$ and $g(z)$ be functions.
		$$f(z) \text{ and } g(z) \text{ continuous} \implies g(f(z)) \text{ continuous}$$
		\label{Continuity of Composition of Functions Theorem - Complex}
	\end{theorem}
	\begin{proof}
		Let $f(z) = w$ be defined in the neighbourhood $\forall z[\abs{z-z-0}<\delta]$, and $g(w) = W$ where $\operatorname{dom} (g) = \operatorname{img}(f)$. Suppose that $f$ is continuous at $z_0$ and $g$ is continuous at $f(z_0)$.
		\begin{align*}
			f \text{ continuous at } z_0 
			&\iff \forall \gamma \exists \delta>0 \left[ \abs{z-z_0} < \delta \implies \abs{f(z) - f(z_0)}<\gamma \right] \\
			&\implies \forall \epsilon \exists \gamma>0 \left[ \abs{f(z) - f(z_0)} < \gamma \implies \abs{g(f(z)) - g(f(z_0))} < \epsilon \right]
		\end{align*} 
		We can always find a small enough $\delta$ for $\gamma$ to satisfy $\abs{g(f(z)) - g(f(z_0))} < \epsilon$.
	\end{proof}

	\begin{figure}[H]
		\centering
		\begin{tikzpicture}[remember picture, baseline=(current bounding box.center)]
			\begin{axis}[
				width=6cm,
				unit vector ratio=1 1 1,
				xlabel={$x$},
				ylabel={$y$},
				xmin=-1, xmax=10,
				ymin=-1, ymax=10,
				ticks = none,
				axis lines = middle,
				axis line style = Black,
				]
				\addplot[
				only marks,
				nodes near coords,
				point meta = explicit symbolic, 
				color=Black,
				mark=*,
				]
				coordinates {
					(8, 8) [$z_0$]
				};
				\draw[dashed, line width=1pt,Grey](axis cs: 8, 8) circle (2);
				\draw[-,draw=Black]  (axis cs: 8,8) to ["$\delta$"] (axis cs: 6,8);
			\end{axis}
		\end{tikzpicture}
		{$\Large \xrightarrow{\mathmakebox[1cm]f}$}
		\begin{tikzpicture}[remember picture, baseline=(current bounding box.center)]
			\begin{axis}[
				width=6cm,
				unit vector ratio=1 1 1,
				xlabel={$v$},
				ylabel={$u$},
				xmin=-10, xmax=2,
				ymin=-1, ymax=10,
				ticks = none,
				axis lines = middle,
				axis line style = Black,
				]
				\addplot[
				only marks,
				nodes near coords,
				point meta = explicit symbolic, 
				color=Black,
				mark=*,
				]
				coordinates {
					(-4, 2) [$w_0$]
				};
				\draw[dashed, line width=1pt,Grey](axis cs: -4, 2) circle (3);
				\draw[-,draw=Black]  (axis cs: -4,2) to ["$\gamma$"] (axis cs: -7,2);
			\end{axis}
		\end{tikzpicture}
		{$\Large \xrightarrow{\mathmakebox[1cm]{g(f)}}$}
		\begin{tikzpicture}[remember picture, baseline=(current bounding box.center)]
			\begin{axis}[
				width=6cm,
				unit vector ratio=1 1 1,
				xlabel={$V$},
				ylabel={$U$},
				xmin=-1, xmax=10,
				ymin=-1, ymax=10,
				ticks = none,
				axis lines = middle,
				axis line style = Black,
				]
				\addplot[
				only marks,
				nodes near coords,
				point meta = explicit symbolic, 
				color=Black,
				mark=*,
				]
				coordinates {
					(5, 5) [$w_0$]
				};
				\draw[dashed, line width=1pt,Grey](axis cs: 5, 5) circle (4);
				\draw[-,draw=Black]  (axis cs: 5,5) to ["$\epsilon$"] (axis cs: 1,5);
			\end{axis}
		\end{tikzpicture}
		\label{Limit Composition of Functions Figure - Complex}
	\end{figure}
	
	\begin{theorem}
		\label{f(z) neq 0 Neighbourhood Theorem - Complex}
		Let $f(z)$ be a function and $f(z_0) \neq 0$.
		\begin{align*}
			f(z_0) \neq 0 
			\implies \exists \epsilon \forall z [\abs{f(z) - f(z_0)}<\epsilon \implies f(z) \neq 0]
		\end{align*}
		That is, if $f(z_0) \neq 0$ then it has a neighbourhood where $f(z) \neq 0$.
	\end{theorem}
	\begin{proof}
		Suppose $f(z)$ is continuous and non-zero at $z_0$, and let $\epsilon = \abs{f(z_0)}/2$:
		\begin{align*}
			&\exists z [f(z) = 0]  \land 
			\forall \epsilon \exists \delta>0 \left[ \abs{z-z_0}<\delta \implies \abs{f(z) - f(z_0)}  < \frac{\abs{f(z_0)}}{2} \right] & \\
			&\implies \abs{f(z_0)} < \frac{\abs{f(z_0)}}{2} & \text{Contradiction!}
		\end{align*}
	\end{proof}

	\begin{theorem}
		Let $f(z) = u(x,y) + iv(x,y)$ be a function, and $z = x + iy$, $z \in \mathbb{C}$. 
		\begin{align*}
			f \text{ continuous at } z_0 \iff \left[ u \text{ continuous at } z_0 \right] \land \left[ v \text{ continuous at } z_0 \right]
		\end{align*}
		\label{Continuity of f=u+iv Theorem - Complex}
	\end{theorem}
	\begin{proof}
		Direct consequence of \cref{Limit of f=u+iv Theorem - Complex}
	\end{proof}
	
	\begin{theorem}
		Let $f$ be continuous in a closed and bounded region $R$, then
		\begin{align*}
			\forall z \in R, \exists M \in \mathbb{R}_{>0}  \left[\abs{f(z) \leq M}\right] 
			\land \abs{\{z : f(z) = M \}} \geq 1
		\end{align*}
		That is, for $\forall z \in R$, $\abs{f(z)} \leq M$ and there is at least one point $z$ where $\abs{f(z)}= M$. $f(z)$ is bounded in $R$.
	\end{theorem}
	\begin{proof}
		Let $f(z) = u(x,y) + iv(x,y)$ be continuous, then 
		\begin{align*}
			\left( \abs{f(z)} = \sqrt{[u(x,y)]^2 + [v(x,y)]^2} \text{ is continuous in } R \right) \land \left( \exists M \in \mathbb{R}_{>0} [\abs{f(z)}<M] \right)
		\end{align*}
	\end{proof}

	\subsection{Exercises}
	
	\begin{example}
		Prove: $$ \lim_{z \rightarrow z_0} f(z) = w_0 \implies \lim_{z \rightarrow z_0} \abs{f(z)} = \abs{w_0}$$
		Note: $\abs{\abs{f(z_0)} - \abs{w_0}} \leq \abs{f(z) - w_0}$
	\end{example}
	\begin{proof}
		\textcolor{Grey}{Use definition of limit, then plug and chug.}
	\end{proof}
	
	\begin{example}
		Prove: Limits involving points at infinity are unique.
	\end{example}
	\begin{proof}
		\textcolor{Grey}{
		Suppose that limit of the point at infinity is not unique, that is there is two neighbourhoods of infinity. Using he definition of the limit, we will arrive at a contradiction where the two neighbourhoods are the same. 
		}
	\end{proof}
	
	\begin{example}
		Prove:
		\begin{align*}
			S \text{ is unbounded } \iff \forall \epsilon \exists z \left[ z \in S : \abs{z} > \frac{1}{\epsilon} \right]
		\end{align*}
		That is, $S$ is unbounded $\iff$ every neighbourhood of the point at infinity contains at least one point in $S$
	\end{example}
	\begin{proof}
		\textcolor{Grey}{
		Proof Sketch: 
		Recall the Riemann Sphere. (\Cref{Riemann Sphere Definition - Complex}). The set $\abs{z} > 1/\epsilon$ corresponds to the points close to $N$, which is the neighbourhood of the point at infinity. If we let $\gamma = 2\epsilon$, $\exists z$ where $\abs{z} > 1/\gamma$ holds. This along with $z \in \mathbb{C}$ (which is $S$ in our case), implies the direction $\impliedby$ is true. That is, we can still find elements in $S$ as we shrink the circle around $N$. 
		}
		
		\textcolor{Grey}{
		$S$ is unbounded implies that for all circle with radius $R$ centred at the origin there is at least one element of $s \in S$ where $\abs{s} > R$. Suppose for contradiction that there is a neighbourhood of the point at infinity that does not contain any points in $S$. We will arrive at a contradiction, where there is $M \in \mathbb{R}_{>0}$ such that $\forall s \in S [\abs{s} < M]$. Thus $S$ is bounded, a contradiction. This implies that the direction $\implies$ is true. 
		}
	\end{proof}

	\section{Differentiation} \label{Differentiation - Complex}
	
	\begin{definition}[Derivative]
		\label{Derivative Definition - Complex}
		Let $f$ be a function where $\abs{z - z_0} < \epsilon$ and $z \in \operatorname{dom}(f)$. Then the derivative of $f$ at point $z_0$:
		\begin{align*}
			f'(z_0) = \lim_{z \rightarrow z_0} \frac{f(z) - f(z_0)}{z - z_0}
		\end{align*}
	\end{definition}

	\begin{definition}[Differentiable]
		\label{Differentiable Definition - Complex}
		A function $f$ is differentiable at $z_0 \in \mathbb{C}$ if $f'(z_0)$ exists.
	\end{definition}

	If we let $\Delta z = z - z_0$ where $z \neq z_0$: 
	\begin{align*}
		f'(z_0) = \lim_{\Delta z \rightarrow 0} \frac{f(z_0 + \Delta z) - f(z_0)}{\Delta z}
	\end{align*}
	
	\begin{figure}[H]
		\centering
		\begin{tikzpicture}[remember picture]
			\begin{axis}[
				small,
				unit vector ratio=1 1 1,
				xlabel={$\Re$},
				ylabel={$\Im$},
				xmin=0, xmax=20,
				ymin=0, ymax=16,
				%					xtick distance=2,
				%					ytick distance=2,
				xtick=100,
				ytick=100,
				axis lines = middle,
				axis line style = Black,
				]
				\addplot[
				only marks,
				nodes near coords,
				point meta = explicit symbolic, 
				color=Black,
				mark=*,
				]
				coordinates {
					(12, 10) [$z_0$]
					(14, 7)
				};
				\draw[dashed,line width=1pt,Grey](axis cs: 12, 10) circle (5);
				\draw[-,draw=Grey]  (axis cs: 12,10)--(axis cs: 7,10)
				node[midway,above] {$\epsilon$};
				\draw[->,draw=Black]  (axis cs: 12,10) to ["$\Delta z$"] (axis cs: 14,7);
				\draw[->, thick, draw=Black]  (axis cs: 0,0)--(axis cs: 12,10)
				node[midway,sloped,above] {$z_0$};
				\draw[->, thick, draw=Black]  (axis cs: 0,0)--(axis cs: 14,7) node[midway,sloped,below]  {$z_0 + \Delta z$};
			\end{axis}
		\end{tikzpicture}
	\end{figure}
	
	There's another notation by letting $\Delta w = f(z + \Delta z) - f(z)$:
	\begin{align*}
		f'(z) = \dv{w}{z}  = \lim_{\Delta z \rightarrow 0} \frac{\Delta w}{\Delta z}
	\end{align*}
	\begin{observation}
		The definition of a derivative in \cref{Derivative Definition - Complex} looks similar to that of a derivative for the real numbers:
		\begin{align*}
			F'(x_0) = \lim_{x \rightarrow x_0} \frac{f(x) - f(x_0)}{x - x_0}
		\end{align*}
		However, the existence of $f'(z)$ possesses a much stronger requirement than the existence of $F'(z)$. That is, let $f(z) = u(x,y) + iv(x,y)$. The existence of $f'(z)$ at point $z_0$ requires the existence of both $u'(x,y)$ and $v'(x,y)$.
		\begin{align*}
			f'(z_0) &= \lim_{(x,y) \rightarrow (x_0, y_0)} \frac{f(z) - f(z_0)}{z - z_0} = \lim_{(x,y) \rightarrow (x_0, y_0)} \frac{u(z) - u(z_0)}{z - z_0} + i \frac{v(z) - v(z_0)}{z - z_0}
		\end{align*}
		and that 
		\begin{align*}
			\lim_{(x,y_0) \rightarrow (x_0, y_0)} \frac{u(x, y_0) - u(x_0, y_0)}{x - x_0} + i \frac{v(x_0, y_0) - v(x_0, y_0)}{x-x_0} \\ 
			= 
			\lim_{(x_0,y) \rightarrow (x_0, y_0)} \frac{u(x_0, y) - u(x_0, y_0)}{x - x_0} + i \frac{v(x_0,y) - v(x_0,y_0)}{x-x_0}
		\end{align*}
		That is
		\begin{align*}
			\lim_{(\Delta x, 0) \rightarrow (0,0)} \frac{u(x_0+\Delta x, y_0) - u(x_0,y_0)}{\Delta x}
			&= \lim_{(0, \Delta y) \rightarrow (0,0)} \frac{u(x_0, y_0 +\Delta y) - u(x_0,y_0)}{\Delta y} \\
			\lim_{(\Delta x, 0) \rightarrow (0,0)} \frac{v(x+\Delta x, y_0) - v(x_0,y_0)}{\Delta x}
			&= \lim_{(0, \Delta y) \rightarrow (0,0)} \frac{v(x_0, y+\Delta y) - v(x_0,y_0)}{\Delta y}
		\end{align*}
		This tells us that the existence of a derivative for a real valued function $F(x)$ does not imply the existence of a derivative for a similar function $f(z)$ in the complex plane, which we will see later. (i.e. Take $f(z) = \abs{z}^2$ and $F(x) = \abs{x}^2$.) We are dealing with a two-dimensional limit instead of a one dimensional limit.
	\end{observation}

	\begin{question}
		Under what conditions will differentiability in\(\mathbb{C}\) imply differentiability in \(\mathbb{R}\), and vice versa?
	\end{question}
	
	\begin{example}
		Let $f(z) = \bar{z}$:
		\begin{align*}
			\frac{\Delta w}{\Delta z} = \frac{\overline{z + \Delta z} - \bar{z}}{\Delta z} = \frac{\bar{z} + \overline{\Delta z} - \bar{z}}{\Delta z} = \frac{\overline{\Delta z}}{\Delta z}
		\end{align*}
	Consider $\Delta z = (\Delta x, \Delta y) \rightarrow (0,0)$. If we move on the real axis, that is $(\Delta x, 0)$:
	\begin{align*}
		\overline{\Delta z} = \overline{\Delta x + i0} = \Delta x - i0  = \Delta x + i0 = \Delta z 
		\implies \frac{\Delta w}{\Delta z} = \frac{\overline{\Delta z}}{\Delta z} = \frac{\Delta z}{\Delta z} = 1
	\end{align*}
	If we move on the imaginary axis, that is $(0, \Delta y)$:
	\begin{align*}
		\overline{\Delta z} = \overline{0 + i\Delta y} = 0 - i \Delta y = - \Delta z
		\implies \frac{\Delta w}{\Delta z} = \frac{\overline{\Delta z}}{\Delta y} = \frac{-\Delta z}{\Delta z} =  -1
	\end{align*}
	Limits are unique, so the limit of $\dd w / \dd z$ does not exist anywhere. 
	\end{example}

	\begin{example}
		Consider $f(z) = \abs{z}^2$:
		\begin{align*}
			\frac{\Delta w}{\Delta z} &= \frac{\abs{z + \Delta z}^2 - \abs{z}^2}{\Delta z}
				= \frac{(z + \Delta z)(\overline{z + \Delta z}) - z \bar{z}}{\Delta z} \\
				&= \frac{(z + \Delta z)(\bar{z} + \overline{\Delta z}) - z \bar{z}}{\Delta z} 
				= \frac{z\bar{z} + \Delta z \bar{z} + \overline{\Delta z} z + \overline{\Delta z} \Delta z - z \bar{z}}{\Delta z}
				= \bar{z} + \overline{\Delta z} + z\frac{\overline{\Delta z}}{\Delta z}
		\end{align*}
		As in the previous example, as $(\Delta x, \Delta y) \rightarrow (0,0)$:
		\begin{align*}
			\overline{\Delta z} &= \Delta z  & &\text{From the real axis} \\
			\overline{\Delta z} &= -\Delta z & &\text{From the imaginary axis}
		\end{align*}
		Thus
		\begin{align*}
			\frac{\Delta w}{\Delta z} &= \bar{z} + \Delta z + z & \Delta z &= (\Delta x, 0) \\
			\frac{\Delta w}{\Delta z} &= \bar{z} - \Delta z - z & \Delta z &= (0, \Delta y)
		\end{align*}
		Therefore, by uniqueness of limits as $\Delta z \rightarrow 0$:
		\begin{align*}
			\lim_{\Delta z \rightarrow 0} (\bar{z} + \Delta z + z) = \lim_{\Delta z \rightarrow 0} (\bar{z} - \Delta z - z )
			\implies  z = - z   \implies z = 0
		\end{align*}
		Hence, $\dd w /\dd z$ does not exist for $z \neq 0$.
		We can also see that:
		\begin{align*}
			\frac{\Delta w}{\Delta z}
			&= \bar{z} + \overline{\Delta z} + z\frac{\overline{\Delta z}}{\Delta z}
			= \overline{\Delta z}  & z = 0
		\end{align*}
		Thus, $\dd w /\dd z $ only exists at $z = 0$: 
		\[ \eval{\dv{w}{z}}_{z=0} = 0 \]
	\end{example}
	
	\begin{remark}
		The following are facts:
		\begin{itemize}
			\item[(1)] A function $f(z)$ can be differentiable at a point $z_0$, but nowhere else in the neighbourhood of $z_0$.
			\item[(2)] \(f(z) = \abs{z}^2 \implies u(x,y) = x^2 + y^2 \land v(x,y) = 0 \), hence $u(x,y)$ and $v(x,y)$ can have continuous partial derivatives of all orders at a point $z_0$, even though $f$ may not be differentiable at $z_0$.
			\item[(3)]
			\( f(z) \text{ differentiable at } z_0 \implies f(z) \text{ continuous at } z_0 \)
			\begin{proof}
				Assume \( f'(z_0) \) exists: 
				\begin{align*}
					&\lim_{z \rightarrow z_0} [f(z) - f(z_0)] = \lim_{z \rightarrow z_0} \frac{f(z) - f(z_0)}{z - z_0} \lim_{z \rightarrow z_0} (z - z_0) = f'(z_0) \cdot 0 = 0\\
					&\implies \lim_{z \rightarrow z_0} f(z) = f(z_0)
				\end{align*}
			So, $f$ is differentiable at \(z_0 \implies f \) is continuous at \(z_0\).
			\end{proof}
			Note: Continuity of a function  at \(z_0 \in \mathbb{C}\) $\nRightarrow$ existence of derivative at point \(z_0\). 
			
			Ex: \(f(z) = \abs{z}^2 \) is continuous everywhere in $\mathbb{C}$ for \(z_0 \neq 0\), but \(f(z_0) \) does not exist at \(z_0\). 
		\end{itemize}
	\end{remark}

	\subsection{Differentiation Rules} \label{Differentiation Rules Subsection - Complex}
	Definition of derivative in \(\mathbb{C}\) (\cref{Derivative Definition - Complex}) is the same of that in \(\mathbb{R}\), so rules remain the same.
	
	Let \(c \in \mathbb{C}\) be a constant and functions \(f\) and \(g\) be differentiable at point \(z\). Then
	\begin{align*}
		\dv{z} c &= 0 & \dv{z} z &= 1 & \dv{z} [cf(z)] &= cf'(z) & \dv{z} z^n &= nz^{n-1} & n \in \mathbb{Z}\setminus\{0\}
	\end{align*}
	Let functions \(f\) and \(g\) be differentiable at point \(z\). Then
	\begin{align*}
		\dv{z} [f(z) + g(z)] &= f'(z) + g'(z) & \dv{z}[f(z)g(z)] &= f(z)g'(z) + f'(z)g(z) 
	\end{align*}
	\begin{align*}
		\dv{z}\left[\frac{f(z)}{g(z)}\right] = \frac{g(z)f'(z) - f(z)g'(z)}{[g(z)]^2}
	\end{align*}
	
	\begin{proof}
		Deriving: \( \dv{z}[f(z)g(z)] = f(z)g'(z) + f'(z)g(z) \)
		
		Let \(w = f(z) g(z)\): 
		\begin{align*}
			\Delta w &= f(z + \Delta z) g(z + \Delta z ) - f(z) g(z)\\
				&=f(z)[g(z + \Delta z) - g(z)] + [f(z + \Delta z) - f(z)] g(z + \Delta z)
		\end{align*}
		Thus
		\begin{align*}
			\frac{\Delta w}{\Delta z}
			= f(z)\frac{g(z + \Delta z) - g(z)}{\Delta z} + \frac{f(z + \Delta z) - f(z)}{\Delta z} g(z + \Delta z)
		\end{align*}
		Hence
		\begin{align*}
			\dv{w}{z} =	\lim_{\Delta z \rightarrow 0} \frac{\Delta w}{\Delta z} = f(z)g'(z) + f'(z)g(z)
		\end{align*}
	\end{proof}
	
	\begin{theorem}[Chain Rule for Composite Functions]
		\label{Chain Rule for Composite Functions Theorem - Complex}
		Let function $f$ be differentiable at $z_0$ and function $g$ be differentiable at \(f(z_0)\). Then \(F(z) = g[f(z)] \text{ is differentiable at } z_0\).
		\begin{align*}
			F'(z_0) = g'[f(z_0)]f'(z_0) 
		\end{align*}
	\end{theorem}
	\begin{proof}
		Suppose $f$ is differentiable at \(z_0\). Let \(w_0 = f(z_0)\) and assume that \(g'(w_0)\) exists. Then 
		\[ \forall w \exists \epsilon [\abs{w - w_0} < \epsilon \implies \Phi(w_0) = 0 ] \]
		Where 
		\begin{align*}
			\Phi(w) &= \frac{g(w) - g(w_0)}{w - w_0} - g'(w_0) & w \neq w_0
		\end{align*}
		Note: \(\lim_{w \rightarrow w_0} \Phi(w) = 0 \), so \(\Phi\) is continuous at \(w_0\).
		Then 
		\begin{align*}
			g(w) - g(w_0) &= [g'(w_0) + \Phi(w)](w - w_0)	&	 \abs{w-w_0}<\epsilon
		\end{align*}
		Note: This is valid for \(w = w_0\). 
		\begin{align*}
			f'(z_0) \text{ exists } &\implies f \text{ continuous at } z_0 \\
			&\implies \forall \epsilon \exists \delta>0 [ \abs{z - z_0} < \delta \implies \abs{w - w_0}< \epsilon]
		\end{align*}
		Hence, we can replace \(w\) by \(f(z)\) when \(\abs{z - z_0} < \delta\). Subbing \(w = f(z)\) and \(w_0 = f(z_0)\):
		\begin{align*}
			\frac{g[f(z)] - g[f(z_0)]}{z - z_0} &= \{g'[f(z_0)] + \Phi[f(z)]\} \frac{f(z) - f(z_0)}{z - z_0} 	& 0 < \abs{z-z_0} < \delta, \ z\neq z_0
		\end{align*}
		Then
		\begin{align*}
			(f \text{ continous at } z_0) \land (\Phi \text{ continuoust at } w_0 = f(z_0))
			\implies \Phi[f(z)] \text{ continuous at } z_0
		\end{align*}
		\[\Phi(w_0) = 0 \implies \lim_{z \rightarrow z_0} \Phi[f(z) = 0] \]
		Thus
		\begin{align*}
			\lim_{z \rightarrow z_0} \frac{g[f(z)] - g[f(z_0)]}{z - z_0} 
			&= \lim_{z \rightarrow z_0} \{g'[f(z_0)] + \Phi[f(z)]\} \frac{f(z) - f(z_0)}{z - z_0} \\
			&= g'[f(z_0)] f'(z_0)
		\end{align*}
		We then get
		\begin{align*}
			F'(z_0) = g'[f(z_0)]f'(z_0) 
		\end{align*}
	\end{proof}
	Alternatively, if we let \(w = f(z) \) and \(W = F(z)\), then the Chain Rule becomes:
	\[ \dv{W}{z} = \dv{W}{w} \dv{w}{z }\]
	Note: Although this looks like a fraction, it is not a fraction and should not be treated as such! (Logical inconsistency when infinitesimals when viewed as ratios.)
	
	
	\subsection{Exercises}
	
	\begin{example}
		Show that \(f'(z)\) does not exist for all points \(z \in \mathbb{C}\) when:
		\begin{itemize}
			\item[(a)] \(f(z) = \Re{z}\)
			\item[(b)] \(f(z) = \Im{z}\)
		\end{itemize}
		\begin{proof}
			{\color{Grey}
			Let \(f(z) = u(x,y) + iv(x,y)\), \(\Delta w = f(x+\Delta x, y + \Delta y) - f(x,y)\).
			
			\underline{\(f(z) = \Re{z}\)}
			
			Recall \(\Re{z} = x + i0\).
			\begin{align*}
				\frac{\Delta w}{\Delta z}
					&= \frac{\Re{z + \Delta z} - \Re{z}}{\Delta z} 
					 = \frac{x + \Delta x - x }{\Delta z} 
					 = \frac{\Delta x}{\Delta x + \Delta y}
			\end{align*}
			Now as \((\Delta x, 0) \rightarrow (0,0)\):
			\begin{align*}
				\lim_{(\Delta x, 0) \rightarrow (0,0)} \frac{\Delta w}{\Delta z} = \lim_{(\Delta x, 0) \rightarrow (0,0)} \frac{\Delta x}{\Delta x} = 1
			\end{align*} 
			Now as \((0, \Delta y) \rightarrow (0,0)\):
			\begin{align*}
				\lim_{(0, \Delta y) \rightarrow (0,0)} \frac{\Delta w}{\Delta z} = \lim_{(0,\Delta y) \rightarrow (0,0)} \frac{0}{\Delta y} = 0
			\end{align*}
			Limits are unique, but this isn't the case, so we conclude that \(f'(z)\) when \(f(z) = \Re{z}\) does not exist.
			
			\underline{\(f(z) = \Im{z}\)}
			
			Recall \(\Im{z} = 0 + iy\).
			\begin{align*}
				\frac{\Delta w}{\Delta z}
					&= \frac{\Im{z + \Delta z} - \Im{z} }{\Delta z}
					 = \frac{y + \Delta y - y}{\Delta z}
					 = \frac{\Delta y}{\Delta x + \Delta y}
			\end{align*}
			Now as \((\Delta x, 0) \rightarrow (0,0)\):
			\begin{align*}
				\lim_{(\Delta x, 0) \rightarrow (0,0)}  \frac{\Delta w}{\Delta z} = \lim_{(\Delta x, 0) \rightarrow (0,0)}  \frac{0}{\Delta x} = 0
			\end{align*} 
			Now as \((0, \Delta y) \rightarrow (0,0)\):
			\begin{align*}
				\lim_{(0,\Delta y) \rightarrow (0,0)} \frac{\Delta w}{\Delta z} = \lim_{(0,\Delta y) \rightarrow (0,0)} \frac{\Delta y}{\Delta y} = 1
			\end{align*}
			Limits are unique, but this isn't the case, so we conclude that \(f'(z)\) when \(f(z) = \Im{z}\) does not exist.
			}
		\end{proof}
	\end{example}

	
	\section{Cauchy-Riemann Equations} \label{Cauchy-Riemann Equations Section - Complex}
%	\subsection{Cartesian and Polar Forms of the Cauchy-Riemann Equations} \label{Cauchy-Riemann Equations (Cartesian and Polar) Subsection - Complex}
	\begin{theorem}[Cauchy-Riemann Equations (Cartesian)]
		\label{Cauchy-Riemann Equations (Cartesian) Theorem - Complex}
		Let \( f(z) = u(x, y) + iv(x,y) \).
		If \(f'(z)\) exists at a point \(z_0 = x_0 + iy_0\), then \(u'(x_0, y_0)\) and \(v'(x_0, y_0)\) exists and satisfy Cauchy-Riemann equations:
		\begin{align*}
			u_x &= v_y	&	u_y = -v_x
		\end{align*}
		Also, as a result of evaluating \(f'(z)\) from the horizontal and vertical direction:
		\begin{align*}
			f'(z_0) &= \eval{[u_x + iv_x]}_{(x_0, y_0)} 
					 = \eval{[v_y - i u_y]}_{(x_0, y_0)}
		\end{align*}
	\end{theorem}
	\begin{proof}
		Let \(f(z) = u(x,y) + iv(x,y)\), and suppose $f'(z)$ exists at $z_0$. Then 
		\begin{align*}
			z_0 &= x_0 + iy_0	&	\Delta z &= \Delta x + i\Delta y
			&	\Delta w &= f(z_0 + \Delta z) - f(z_0)
		\end{align*}
		So that
		\begin{align*}
			\Delta w = [u(x_0 + \Delta x, y_0 + \Delta y) + iv(x_0 + \Delta x, y_0 + \Delta y)] - [u(x_0, y_0) + iv(x_0, y_0)]
		\end{align*}
		Therefore
		\begin{align*}
			\frac{\Delta w}{\Delta z}
			&= \frac{u(x_0 + \Delta x, y_0 + \Delta y) - u(x_0, y_0)}{\Delta x + i\Delta y} + i\frac{v(x_0 + \Delta x, y_0 + \Delta y) - v(x_0, y_0)}{\Delta x + i\Delta y}
		\end{align*}
		Note: This equation remains valid as \((\Delta x, \Delta y) \rightarrow (0,0)\).
		
		\textbf{Horizontal Approach:}
		
		Let \((\Delta x, 0) \rightarrow (0,0)\) in the horizontal direction, then 
		\begin{align*}
			f'(z_0) 
			=& \lim_{\Delta x \rightarrow 0} \frac{u(x_0 + \Delta x, y_0) - u(x_0, y_0)}{\Delta x} + i \lim_{\Delta x \rightarrow 0} \frac{v(x_0 + \Delta x, y_0) - v(x_0, y_0)}{\Delta x} \\
			\implies& f'(z_0) = u_x(x_0, y_0) + iv_x(x_0, y_0)
		\end{align*}
		
		\textbf{Vertical Approach:}
		
		Let \((0, \Delta y) \rightarrow (0,0)\) in the vertical direction, then
		\begin{align*}
			f'(z_0) 
			=& \lim_{\Delta y \rightarrow 0} \frac{u(x_0, y_0 + \Delta y) - u(x_0, y_0)}{i \Delta y} + i \lim_{\Delta x \rightarrow 0} \frac{v(x_0, y_0+\Delta y) - v(x_0, y_0)}{i \Delta y} \\
			=& - i \lim_{\Delta y \rightarrow 0} \frac{u(x_0, y_0 + \Delta y) - u(x_0, y_0)}{ \Delta y} + \lim_{\Delta x \rightarrow 0} \frac{v(x_0, y_0+\Delta y) - v(x_0, y_0)}{\Delta y} \\
			\implies& f'(z_0) = v_y(x_0, y_0) - iu_y(x_0, y_0)
		\end{align*}
		
		\textbf{Putting it together:}
		
		For \(f'(z)\) to exists at \(z_0\), \(f(z_0)\) from the horizontal approach must equal that of the vertical approach. By equating the real and imaginary parts: 
		\begin{align*}
			&u_x(x_0, y_0) + iv_x(x_0, y_0) = v_y(x_0, y_0) - iu_y(x_0, y_0)\\
			&\implies (u_x = v_y) \land (u_y = - v_x)
		\end{align*}
		
	\end{proof}
	
	\begin{theorem}[Cauchy-Riemann Equations (Polar)]
		\label{Cauchy-Riemann Equations (Polar) Theorem - Complex}
		Let \(f(z) = u(r, \theta) + iv(r,\theta)\) be defined in some neighbourhood \(\epsilon\) of \(z_0 = r_0 e^{i\theta_0}\), $z_0 \neq 0$. If the first order partials derivatives of \(u\) and \(v\) with respect to \(r\) and \(\theta\) exists and are continuous at \(z_0\), and satisfies the polar form of the Cauchy-Riemann equations: 
		\begin{align*}
			ru_r &= v_\theta	&	u_\theta &= -r v_r
		\end{align*}
		Then \(f'(z_0)\) exists:
		\begin{align*}
			f'(z_0) = \eval{e^{-i\theta}(u_r + iv_r)}_{(r_0, \theta_0)}
					= \eval{\frac{-i}{z_0}(u_\theta + iv_\theta )}_{(r_0, \theta_0)}
		\end{align*}
	\end{theorem}
	\begin{proof}
		Let \(f(z) = u(r,\theta) + iv(r,\theta)\). Suppose that the first order partial derivatives of \(u\) and \(v\) exists in some neighbourhood \(\epsilon \) of \(z_0\) and is continuous at \(z_0\). By differentiating \(u\) with respect to \(x\) and \(y\):
		\begin{align*}
			\pdv{u}{r} &= \pdv{u}{x} \pdv{x}{r} + \pdv{u}{y}\pdv{y}{r} &
			\pdv{u}{\theta} &= \pdv{u}{x} \pdv{x}{\theta} + \pdv{u}{y}\pdv{y}{\theta} 
		\end{align*}
		Likewise for \(v\). As \(x = r \cos\theta \) and \(y = r\sin \theta\):
		\begin{align*}
			u_r &= u_x \cos \theta + u_y \sin \theta 
				&	u_\theta &= -u_x r \sin \theta + u_y r \cos \theta \\
			v_r &= v_x \cos \theta + v_y \sin \theta 
			&	v_\theta &= -v_x r \sin \theta + v_y r \cos \theta \\
		\end{align*}
		From \cref{Cauchy-Riemann Equations (Cartesian) Theorem - Complex} we have: 
		\begin{align*}
			u_x &= v_y	&	u_y = -v_x
		\end{align*}
		Subbing the Cauchy-Riemann equations into \(v_r\) and \(v_\theta\):
		\begin{align*}
			u_r &= u_x \cos \theta + u_y \sin \theta 
			&	u_\theta &= -u_x r \sin \theta + u_y r \cos \theta \\
			v_r &= -u_y \cos \theta + u_x \sin \theta 
			&	v_\theta &= u_y r \sin \theta + u_x r \cos \theta \\
		\end{align*}
		We can see that: 
		\begin{align*}
			ru_r &= v_\theta 	&	u_\theta &= -rv_r
		\end{align*}
		Which are the Cauchy Riemann equations in polar form. Let's verify it without relying on the Cauchy-Riemann equations in Cartesian form: 
		
		{\color{Grey}
			Recall:
			\begin{align*}
				u_r &= u_x \cos \theta + u_y \sin \theta 
				&	u_\theta &= -u_x r \sin \theta + u_y r \cos \theta \\
				v_r &= v_x \cos \theta + v_y \sin \theta 
				&	v_\theta &= -v_x r \sin \theta + v_y r \cos \theta \\
			\end{align*}
			Writing \(u_r\) and \(v_r\) in matrix notation: 
			\begin{align*}
				\begin{bmatrix}
					u_r \\ u_\theta
				\end{bmatrix}
				&= 
				\begin{bmatrix}
					\cos\theta & \sin\theta \\
					-r\sin\theta & r\cos\theta
				\end{bmatrix}
				\begin{bmatrix}
					u_x \\ u_y
				\end{bmatrix}
			\end{align*}
			Solving for \(u_x\) and \(u_y\):
			\begin{align*}
				\begin{bmatrix}
					u_x \\ u_y
				\end{bmatrix}
				&=
					\begin{bmatrix}
						\cos\theta & \sin\theta \\
						-r\sin\theta & r\cos\theta
					\end{bmatrix}^{-1}
					\begin{bmatrix}
						u_r \\ u_\theta
					\end{bmatrix} & \\
				&= \frac{1}{r \cos^2\theta + r \sin^2\theta}
					\begin{bmatrix}
						r\cos\theta & -\sin\theta \\
						-r\sin\theta & \cos\theta
					\end{bmatrix}
					\begin{bmatrix}
						u_r \\ u_\theta
					\end{bmatrix} 
					& \begin{bmatrix}
						a & b \\
						c & d \\
					  \end{bmatrix}^{-1}
				  	  &=
				  	  \frac{1}{ad - cb}
				  	  \begin{bmatrix}
				  	  	d & -b \\
				  	  	-c & a
				  	  \end{bmatrix} \\
				&= \frac{1}{r}
					\begin{bmatrix}
						r\cos\theta & -\sin\theta \\
						r\sin\theta & \cos\theta
					\end{bmatrix}
					\begin{bmatrix}
						u_r \\ u_\theta
					\end{bmatrix} & \\
			\end{align*}
		
			It is clear that for \(u_x\) and \(u_y\), and likewise for \(v_x\) and \(v_y\):
			\begin{align}
				\label{Cauchy-Riemann Equations (Polar) Theorem Proof Eqn 1 - Complex}
				u_x &= u_r \cos\theta - \frac{1}{r} u_\theta \sin\theta & 
				u_y &= u_r \sin\theta + \frac{1}{r} u_\theta \cos\theta \\
				\label{Cauchy-Riemann Equations (Polar) Theorem Proof Eqn 2 - Complex}
				v_x &= v_r \cos\theta - \frac{1}{r} v_\theta \sin\theta & 
				v_y &= v_r \sin\theta + \frac{1}{r} v_\theta \cos\theta 
			\end{align}
		
			Using the Cauchy-Riemann equations \(u_x = v_y\) and \(u_y = -v_x\), we see:
			\begin{align*}
				u_r \cos\theta - \frac{1}{r}u_\theta \sin\theta 
					&= v_r \sin\theta + \frac{1}{r} v_\theta \cos\theta \\
				u_r \sin\theta + \frac{1}{r}u_\theta \cos\theta 
					&= -v_r \cos\theta + \frac{1}{r} v_\theta \sin\theta
			\end{align*}
		
			Clearly, the equations are equal only if 
			\begin{align*}
				ru_r &= v_\theta & u_\theta = -rv_r
			\end{align*}
			Which are the polar forms of the Cauchy-Riemann equations.	
		}
		
		Show \(f'(z_0) = e^{-i\theta}(u_r + iv_r)\):
		
		{\color{Grey}
			Recall from \cref{Cauchy-Riemann Equations (Cartesian) Theorem - Complex}:
			\begin{align*}
				f'(z_0) =  u_x + iv_y
			\end{align*}
			Using \cref{Cauchy-Riemann Equations (Polar) Theorem Proof Eqn 1 - Complex} and \cref{Cauchy-Riemann Equations (Polar) Theorem Proof Eqn 2 - Complex} from before and substituting them into \(f'(z_0)\):
			\begin{align*}
				f'(z_0) &= \eval{\left(u_r \cos\theta - \frac{1}{r}u_\theta \sin\theta + iv_r \cos\theta - \frac{i}{r} v_\theta \sin\theta \right)}_{(r_0, \theta_0)} \\
				&= \eval{\left(u_r \cos\theta + v_r \sin\theta + iv_r \cos\theta -iu_r \sin\theta \right)}_{(r_0, \theta_0)} \\
				&= \eval{\left[u_r(\cos\theta - i\sin\theta) + v_r(\sin\theta + i\cos\theta)\right]}_{(r_0,\theta_0)} \\
				&= \eval{\left[u_r(\cos\theta - i\sin\theta) + i v_r(\cos\theta - i\sin\theta)\right]}_{(r_0,\theta_0)} \\
				&= \eval{\left[\left(\frac{e^{i\theta} + e^{-i \theta}}{2} - \frac{e^{i\theta} - e^{-i \theta}}{2} \right)(u_r+iv_r)\right]}_{(r_0, \theta_0)} \\
				&= \eval{e^{-i\theta}(u_r + iv_r)}_{(r_0,\theta_0)} \\
				&= \eval{\frac{-i}{re^{i\theta}} (u_\theta + iv_\theta)}_{(r_0, \theta_0)}
				 = \eval{\frac{-i}{z_0} (u_\theta + iv_\theta)}_{(r_0, \theta_0)}
					& (ru_r = v_\theta) \land (u_\theta = -rv_r)
			\end{align*}
			Thus
			\begin{align*}
				f'(z_0) &= \eval{e^{-i\theta}(u_r + iv_r)}_{(r_0,\theta_0)} 
						 = \eval{\frac{-i}{z_0} (u_\theta + iv_\theta)}_{(r_0, \theta_0)}
			\end{align*}
		}
		
		
		
	\end{proof}

	\begin{question}
		When comparing the Cartesian form to the polar form of the Cauchy-Riemann equations:
		\begin{align*}
			&f'(z_0) \text{ exists} \implies \forall z_0[(u_x = v_y)\land(u_y = -v_x)] \\
			&(z_0 \neq 0) \land \forall z_0 [(ru_r = v_\theta) \land (u_\theta = -r v_r)]  
				\implies f'(z_0) \text{ exists }
		\end{align*}
		Should both be $\iff$ instead of $\implies$? 
		No, satisfying Cauchy-Riemann equations does not guarantee differentiability at a point as we will see in \cref{f(z) satisfy Cauchy-Riemann but f'(z_0) does not exist Example - Complex}. However, satisfying certain conditions allows allows differentiability to exist (\cref{Cauchy-Riemann Differentiablity Conditions Theorem - Complex}). 
	\end{question}

	\begin{example}[Solving the \(f'(z)\) using the partial derivative with respect to one variable] 
		Recall in \cref{Cauchy-Riemann Equations (Cartesian) Theorem - Complex}: 
		\begin{align*}
			f'(z_0) &= \eval{[u_x + iv_x]}_{(x_0, y_0)} 
			= \eval{[v_y - i u_y]}_{(x_0, y_0)}
		\end{align*}
		This implies we can solve \(\dd{f(z)} /\dd{z}\) by taking the partial of \(f(z)\) with respect to \(x\) or \(y\). Consider \(f(z) = z^2\):
		\begin{align*}
			f(z) = z^2 = x^2 - y^2 + i2xy
		\end{align*}
		We then have:
		\begin{align*}
			u(x,y) &= x^2 - y^2	& v(x,y) &= 2xy
		\end{align*}
		Hence
		\begin{align*}
			u_x &= 2x = v_y	&
			u_y &= -2y = -v_x	
		\end{align*}
		Thus
		\begin{align*}
			f'(z) = 2x + i2y = 2(x+iy) = 2z 
		\end{align*}
	\end{example}
	
	\begin{example}[Using Cauchy-Riemann equations to find where \(f(z)\) is not differentiable]
		Using the contrapositive of \(f'(z_0) \text{ exists} \implies \exists u' \exists v' [(u_x = v_y)\land(u_y = -v_x)]\):
		\begin{align*}
			\exists z_0 [(u_x \neq v_y)\lor(u_y \neq -v_x)] \implies f(z) \text{ not differentiable at } z_0 
		\end{align*}
		Consider \(f(z) = \abs{z}^2\):
		\begin{align*}
			u(x,y) &= x^2 + y^2	&
			v(x,y) = 0
		\end{align*}
		By Cauchy-Riemann: 
		\begin{align*}
			2x &= 0 & 2y &= 0
		\end{align*}
		Therefore, \(f'(z)\) only exists at \((0,0)\) and does not exist elsewhere. 
	\end{example}	

	Note: \Cref{Cauchy-Riemann Equations (Cartesian) Theorem - Complex} does not guarantee the existence of \(f'(z)\) at \(z_0\).
	
	\begin{example}[\(f(z)\) satisfy Cauchy-Riemann equations at (0,0), but \(f'(0)\) does not exist]
	\label{f(z) satisfy Cauchy-Riemann but f'(z_0) does not exist Example - Complex}
		Consider
		\begin{align*}
			f(z) = 
			\begin{cases}
				\bar{z}^2 / z & z \neq 0 \\
				0			  & z = 0
			\end{cases}
		\end{align*}
		Then
		\begin{align*}
			u(x,y) &= \frac{x^3 - 3xy^2}{x^2 + y^2} &
			v(x,y) &= \frac{y^3 - 3x^2y}{x^2 + y^2} & (x,y) \neq (0,0)
		\end{align*}
		Checking differentiability at \((0,0)\), note \(u(0,0) = 0\) and \(v(0,0) = 0\):
		\begin{align*}
			u_x(0,0) 
			&= \lim_{\Delta x \rightarrow 0} \frac{u(0+\Delta x, 0) - u(0,0)}{\Delta x}
			 = \lim_{\Delta x \rightarrow 0} \frac{\Delta x}{\Delta x} = 1 \\
			v_y(0,0) 
			&= \lim_{\Delta y \rightarrow 0} \frac{v(0, 0+\Delta y) - v(0,0)}{\Delta y}
			 = \lim_{\Delta y \rightarrow 0} \frac{\Delta y}{\Delta y} = 1 \\
			u_y(0,0) 
			 &= \lim_{\Delta y \rightarrow 0} \frac{u(0, 0+\Delta y) - u(0,0)}{\Delta y}
			  = \lim_{\Delta y \rightarrow 0} \frac{0/(\Delta y)^2}{\Delta y} = 0 \\
			v_x(0,0) 
			 &= \lim_{\Delta x \rightarrow 0} \frac{v(0+\Delta x, 0) - v(0,0)}{\Delta x}
			  = \lim_{\Delta x \rightarrow 0} \frac{0/(\Delta x)^2}{\Delta x} = 0
		\end{align*}
		We can see that the Cauchy-Riemann equations are satisfied: 
		\begin{align*}
			u_x &= v_y = 1	&	u_y &= -v_x = 0
		\end{align*}
		However, \(f'(0)\) does not exist: (Brown and Churchill - Complex Variables and Applications, Section 20, Exercise 9 \cite{Brown.J;Churchill.R-Complex-Variables-2014})
		
		Let \(\Delta w = f(z + \Delta z) - f(z)\). We need to show for all nonzero points on the real and imaginary axis, \(\Delta w/\Delta z = -1\), but for all nonzero points on the line \(\Delta x = \Delta y\), \(\Delta w / \Delta z = -1\). Hence, a contradiction, so \(f('0)\) does not exist.
		
		\begin{figure}[H]
			\centering
			\begin{tikzpicture}[remember picture]
				\begin{axis}[
					unit vector ratio=1 1 1,
					xlabel={$\Re$},
					ylabel={$\Im$},
					xmin=-1, xmax=12,
					ymin=-1, ymax=12,
					%					xtick distance=2,
					%					ytick distance=2,
					xtick=100,
					ytick=100,
					axis lines = middle,
					axis line style = Black,
					]
					\addplot[
					only marks,
					nodes near coords,
					point meta = explicit symbolic, 
					color=Black,
					mark=*,
					]
					coordinates {
						(7, 7) []
						(7, 0) [$(\Delta x, 0)$]
						(0, 7) [$(0, \Delta y)$]
						(0, 0) [$(0, 0)$]
					};
					\draw[->, very thick, draw=Black]  (axis cs: 7,0)--(axis cs: 5,0)
					node[midway,above] {};
					\draw[->, very thick, draw=Black]  (axis cs: 0,7)--(axis cs: 0,5);
					\draw[->, very thick, draw=Black]  (axis cs: 7,7)--(axis cs: 5,5)
					node[midway,sloped,above] {$\Delta y = \Delta x$};
					\draw[draw=Black]  (axis cs: 10,10)--(axis cs: 0,0);
				\end{axis}
			\end{tikzpicture}
		\end{figure}
		
		{\color{Grey}
			\begin{align*}
				\frac{\Delta w}{\Delta z}
					&=\frac{f(z+\Delta z) - f(z)}{\Delta z}
					 =\frac{u(x+\Delta x, y+\Delta y) + v(x+\Delta x, y+\Delta y)}{\Delta x + \Delta y} - \frac{u(x,y) + v(x,y)}{\Delta x + \Delta y}
			\end{align*}
		
			\textbf{Along the real axis:}
			
			Evaluating along \((\Delta x, 0) \rightarrow (0,0)\).
			\begin{align*}
				\lim_{(\Delta x, 0) \rightarrow (0,0)} \frac{\Delta w}{\Delta z}
					&= \frac{u(\Delta x, 0) + v(\Delta x, 0)}{\Delta x}
						- \frac{u(0,0) + v(0,0)}{\Delta x} \\
					&= \frac{1}{\Delta x} \left[ \frac{(\Delta x)^3}{(\Delta x)^2} + \frac{0}{(\Delta x)^2}\right] - 0
					 = \frac{\Delta x}{\Delta x} = 1
			\end{align*}
			
			\textbf{Along the imaginary axis:}
			
			Evaluating along \((0, \Delta y) \rightarrow (0,0)\).
			\begin{align*}
				\lim_{(0,\Delta y) \rightarrow (0,0)} \frac{\Delta w}{\Delta z}
					&= \frac{u(0,\Delta y) + v(0, \Delta y)}{\Delta y}
						- \frac{u(0,0) + v(0,0)}{\Delta y} \\
					&= \frac{1}{\Delta y} \left[ \frac{0}{(\Delta y)^2} + \frac{(\Delta y)^3}{(\Delta y)^2}\right] - 0 
					 = \frac{\Delta y}{\Delta y} = 1
			\end{align*}
		
			\textbf{Along the axis \(\Delta x = \Delta y\):}
			
			Evaluating along \((\Delta x, \Delta x) \rightarrow (0,0)\).
			\begin{align*}
				\lim_{(\Delta x,\Delta x) \rightarrow (0,0)} \frac{\Delta w}{\Delta z}
					&= \frac{u(\Delta x, \Delta x)}{\Delta x + \Delta x} - \frac{u(0,0) + v(0,0)}{\Delta x + \Delta x} \\
					&= \frac{1}{2\Delta x} \left[ \frac{(\Delta x)^3 - 3(\Delta x)^3}{2(\Delta x)^2} + \frac{(\Delta x)^3 - 3(\Delta x)^3}{2(\Delta x)^2}\right] \\
					&= \frac{1}{2\Delta x}  \left[ - \frac{2(\Delta x)^3}{2(\Delta x)^2} - \frac{2(\Delta x)^3}{2(\Delta x)^2} \right] 
					= \frac{1}{2\Delta x} \left[-\Delta x - \Delta x \right] = -\frac{2\Delta x}{2\Delta x} = -1
			\end{align*}
			As we can see, the limits are not unique regardless of the path we take to approach \((0,0)\), hence \(f'(0)\) does not exist. Therefore, an equation can satisfy the Cauchy-Riemann equations at \(0,0\), yet have a derivative that does not exist. The Cauchy-Riemann equations does not guarantee differentiability at \(z_0\).
		}
		
	
	\end{example}

	\begin{example}(Any branch of \(f(z) = z^{1/2}\) is differentiable everywhere in domain of definition)
		Let 
		\begin{align*}
			f(z) = z^{1/2} &= \sqrt{r}e^{i\theta}	&	r>0, \ \alpha<\theta<\alpha+2\pi
		\end{align*}
		Hence
		\begin{align*}
			u(r, \theta) &= \sqrt{r}\cos \left(\frac{\theta}{2}\right) &
			v(r, \theta) &= \sqrt{r}\sin \left(\frac{\theta}{2}\right)
		\end{align*}
		By Cauchy-Riemann: 
		\begin{align*}
			ru_r &= \frac{\sqrt{r}}{2} \cos \left(\frac{\theta}{2}\right) = v_\theta &
			u_\theta &= - \frac{\sqrt{r}}{2} \sin \left(\frac{\theta}{2}\right) = -r v_r
		\end{align*}
		Thus, the derivative exists wherever \(f(z)\) is defined. Also, by \cref{Cauchy-Riemann Equations (Polar) Theorem - Complex}:
		\begin{align*}
			f'(z) 
				&= \eval{e^{i\theta} (u_r + iv_r)}_{(r_0, \theta_0)} \\
				&= e^{-i\theta} \left[\frac{1}{2\sqrt{r}}\cos\left(\frac{\theta}{2}\right) + i \frac{1}{2\sqrt{r}}\sin\left(\frac{\theta}{2}\right)\right] 
				= \frac{1}{2\sqrt{r}} e^{-i\theta} \left[\cos \left(\frac{\theta}{2}\right) + i \sin \left(\frac{\theta}{2}\right)\right] \\
				&= \frac{1}{2\sqrt{r}e^{i\theta/2}} = \frac{1}{2f(z)} = \frac{1}{2} z^{-1/2}
		\end{align*}
	\end{example}

	\subsection{Complex Form of the Cauchy-Riemann Equations}
	\label{Cauchy-Riemann Equations (Complex) Subsection - Complex}
	
	\begin{theorem}[Cauchy-Riemann Equation (Complex Form)]
		\label{Cauchy-Riemann Equations (Complex) Theorem - Complex}
		Let \(f(z) = u(x,y) + iv(x,y)\). If the first order partial derivatives of \(u\) and \(v\) with respect to \(x\) and \(y\) exists and satisfy the Cauchy-Riemann equations. Then
		\[\pdv{\bar{z}} f(z) = 0\]
	\end{theorem}
	\begin{proof}{\color{Grey}
		Recall:
		\begin{align*}
			x &= \frac{z + \bar{z}}{2} & y &= \frac{z - \bar{z}}{2i}
		\end{align*}
		Let \(F\) be a real valued function, that is \(x, y \in \mathbb{R}\). Then
		\begin{align*}
			\pdv{F}{\bar{z}} &= \pdv{F}{x} \pdv{x}{\bar{z}} + \pdv{F}{y} \pdv{y}{\bar{z}}
		\end{align*}
		Substituting \(\pdv{x}{\bar{z}} = 1/2\) and \(\pdv{y}{\bar{z}} = i/2\):
		\begin{align*}
			\pdv{F}{\bar{z}} &= \frac{1}{2} \left(\pdv{F}{x} + i \pdv{F}{y} \right)
		\end{align*}
		Define the operator:
		\begin{align*}
			\pdv{\bar{z}} &= \frac{1}{2} \left(\pdv{x} + i \pdv{y} \right)
		\end{align*}
		Then 
		\begin{align*}
			\pdv{f}{\bar{z}} &= \frac{1}{2} \left(\pdv{f}{x} + i \pdv{f}{y} \right)
				= \frac{1}{2} \left(\pdv{u}{x} + i \pdv{v}{x} + i\pdv{u}{y} -  \pdv{v}{y}  \right) \\
				&= \frac{1}{2} \left[ (u_x - v_y) + i (u_y + v_x)\right]
		\end{align*}
		We can see that if \(\pdv{f}{\bar{z}}\) satisfies the Cauchy-Riemann equations (\cref{Cauchy-Riemann Equations (Cartesian) Theorem - Complex}):
		\begin{align*}
			\pdv{\bar{z}} f(z) &= 0 & \pdv{x}f &= -i \pdv{f}{y} \implies i\pdv{x}f = \pdv{f}{y} 
		\end{align*}
		}
	\end{proof}
	
	\subsection{Conditions for Differentiability}
	\label{Conditions for Differentiablity Subsection - Complex}
	
	\begin{theorem}
		\label{Cauchy-Riemann Differentiablity Conditions Theorem - Complex}
		Let \(f(z) = u(x,y) + iv(x,y)\) be defined in some neighbourhood \(\epsilon\) of point \(z_0 = x_0 + iy_0\). Consider the first order partial derivatives of \(u\) and \(v\) with respect to \(x\) and \(y\). If they 
		\begin{itemize}
			\item[(1)] Exist for all \(z\), \(\abs{z-z_0}<\epsilon\).
			\item[(2)] Are continuous at \(z_0\).
			\item[(3)] Satisfies the Cauchy-Riemann equations at \(z_0\).
		\end{itemize}
		Then \(f'(z_0)\) exists:
		\[f'(z_0) = \eval{(u_x + iv_x)}_{(x_0, y_0)}\]
	\end{theorem}
	\begin{proof}
		Assume the first order partial derivatives of \(u\) and \(v\) with respect to \(x\) and \(y\) exists \(\forall z [\abs{z-z_0}<\epsilon]\), are continuous at \(z_0\), and satisfies the Cauchy-Riemann equations. 
		Let \(\Delta z = \Delta x + i\Delta y\), \(0<\abs{\Delta z}<\epsilon\), and \(\Delta w = f(z_0 + \Delta z) - f(z_0)\).
		We then have \[\Delta w = \Delta u + i\Delta v\]
		Where
		\begin{align*}
			\Delta u &= u(x_0 + \Delta x, y_0 + \Delta y) - u(x_0, y_0) \\
			\Delta v &= v(x_0 + \Delta x, y_0 + \Delta y) - v(x_0, y_0) 
		\end{align*}
		Since first order partials of \(u\) and \(v\) are continuous at \(z_0\):
		\begin{align*}
			\Delta u &= u_x(x_0, y_0)\Delta x + u_y(x_0, y_0)\Delta y + \epsilon_1\Delta x + \epsilon_2 \Delta y \\
			\Delta v &= v_x(x_0, y_0)\Delta x + v_y(x_0, y_0)\Delta y + \epsilon_3\Delta x + \epsilon_4 \Delta y
		\end{align*}
		\[(\epsilon_1, \epsilon_2, \epsilon_3, \epsilon_4) \rightarrow (0,0,0,0) \text{ as } (\Delta x, \Delta y) \rightarrow (0,0)\]
		Substituting \(\Delta u\) and \(\Delta v\) into \(\Delta w\):
		\begin{align*}
			\Delta w 
				=& u_x(x_0, y_0)\Delta x + u_y(x_0, y_0)\Delta y + \epsilon_1\Delta x + \epsilon_2 \Delta y \\
				&+ i[v_x(x_0, y_0)\Delta x + v_y(x_0, y_0)\Delta y + \epsilon_3\Delta x + \epsilon_4 \Delta y]
		\end{align*}
		Using the Cauchy-Riemann equations and dividing by \(\Delta z\):
		\begin{align*}
			\frac{\Delta w}{\Delta z}
			&= u_x(x_0, y_0) + iv_x(x_0, y_0) + (\epsilon_1 + i\epsilon_3) \frac{\Delta x}{\Delta z} + (\epsilon_2 + i\epsilon_4) \frac{\Delta y}{\Delta z}
		\end{align*}
		From the inequalities \(\abs{\Delta x} \leq \abs{\Delta z}\) and \(\abs{\Delta y} \leq \abs{\Delta z}\):
		\begin{align*}
			\abs{\frac{\Delta x}{\Delta z}} &\leq 1		&
			\abs{\frac{\Delta y}{\Delta z}} &\leq 1	
		\end{align*}
		So
		\begin{align*}
			\abs{(\epsilon_1 + i\epsilon_3 )\frac{\Delta x}{\Delta z}} 
				&\leq \abs{\epsilon_1 + i \epsilon_3} \leq \abs{\epsilon_1} + \abs{\epsilon_3} \\
			\abs{(\epsilon_2 + i\epsilon_4 )\frac{\Delta y}{\Delta z}} 
				&\leq \abs{\epsilon_2 + i \epsilon_4} \leq \abs{\epsilon_2} + \abs{\epsilon_4}
		\end{align*}
		Then \(\abs{\epsilon_2} + \abs{\epsilon_4} \rightarrow 0\) and \(\abs{\epsilon_1} + \abs{\epsilon_3} \rightarrow 0\) as \(\Delta z = \Delta x + i\Delta y \rightarrow 0\).
		\begin{align*}
			\implies& \frac{\Delta w}{\Delta z} = u_x(x_0, y_0) + iv_x(x_0, y_0) 
			\implies f'(z_0) \text{ exists}
		\end{align*}
	\end{proof}

	\begin{example}[All 3 conditions must be satisfied for \(f'(z_0)\) to exist]
		Do not use expression of \(f'(z)\) before existence of \(f'(z_0)\)  is established.
		Consider \(f(z) = x^3 + i(1-y)^3\).
		\begin{align*}
			u(x,y) &= x^3	& 	v(x,y) &= (1-y)^3
		\end{align*}
		Taking the partial derivatives:
		\begin{align*}
			u_x &= 3x^2	& 	v_x &= 0\\
			u_y &= 0	&	v_y &= -3(1-y)^2
		\end{align*}
		It would be foolish to ignore Cauchy-Riemann and directly use:
		\begin{align*}
			f'(z) = u_x + iv_x = 3x^2
		\end{align*}
		We can see that the Cauchy-Riemann equations are satisfied only if:
		\begin{align*}
			3x^2 = -3(1-y)^2 \implies  x^2 + (1-y)^2 = 0 \implies (x=0)\land(y=1)
		\end{align*}
		Therefore, \(f'(z)\) exists only if \(z = i\), and that \(f'(i) = 0\)
	\end{example}
	
	\section{Analytic Functions} \label{Analytic Functions - Complex}
	
	\begin{definition}[Analytic/Regular/Holomorphic]
		\label{Analytic Definition - Complex}
		Let $S$ be an open set, \(S \subset \mathbb{C}\). Let \(f\) be a function.
		\begin{align*}
			f \text{ is analytic in } S \iff \forall z \in S[ f'(z) \text{ exists }]
		\end{align*}
		We say \(f(z)\) is analytic at a point \(z_0\) if it is analytic in some neighbourhood of \(z_0\). If we say that \(f(z)\) is analytic in a closed set \(S'\) then we mean that it is analytic in an open set \(S\) where \(S' \subset S\).
	\end{definition}
	
	\begin{definition}[Entire]
		A function \(f(z)\) is entire if it is analytic at all points in the plane. 
	\end{definition}

	\begin{example}
		\begin{align*}
			\text{Derivative of polynomial exists everywhere} \implies \text{All polynomials are entire functions}
		\end{align*}
	\end{example}
	
	See \cref{Conditions for Differentiablity Subsection - Complex} for conditions for a function te be differentiable, hence analytic in a set \(S\).
	
	\begin{corollary}
		Let \(f(z)\) and \(g(z)\) be analytic in a domain \(D\). Then the following are analytic in \(D\):
		\begin{align*}
			&f(z)  + g(z) 		& \\
			&f(z)g(z)	 		& \\
			&\frac{f(z)}{g(z)} 	& g(z) \neq 0 \forall z \in D
		\end{align*}
		Likewise, if \(P(z)\) and \(Q(z)\) are polynomials, then \(P(z)/Q(z)\) is analytic if \(\forall z \in D[Q(z) \neq 0]\).
	\end{corollary}

	\begin{corollary}
		Let \(w\) be the image of \(D\) under \(f(z)\) and \(w\) be the domain of \(g\). Then \(g(f(z))\) is analytic in \(D\) and 
		\begin{align*}
			\dv{z} g[f(z)] = g'[f(z)]f'(z)
		\end{align*}
	\end{corollary}

	\begin{theorem}
		Let \(D\) be the domain of a function \(f(z)\).
		\begin{align*}
			\forall z \in D[f'(z) = 0] \implies f(z) \text{ is constant in } D
		\end{align*}
	\end{theorem}
	\begin{proof}
		Let \(f(z) = u(x,y) + iv(x,y)\) with domain \(D\), and \(P\), \(P'\), and \(Q\) be points in \(D\). Let \(\vec{U}\) be the unit vector on the line segment \(L\) connecting \(P\) and \(P'\), and \(s\) be the distance along \(L\).
		\begin{align*}
			f'(z) = 0 &\implies \forall z \in D [u_x = u_y = v_x = v_y = 0]
		\end{align*}
			\begin{figure}[H]
			\centering
			\begin{tikzpicture}[remember picture]
				\begin{axis}[
					unit vector ratio=1 1 1,
					xlabel={$x$},
					ylabel={$y$},
					xmin=0, xmax=20,
					ymin=0, ymax=12,
					%					xtick distance=2,
					%					ytick distance=2,
					xtick=100,
					ytick=100,
					axis lines = middle,
					axis line style = Black,
					]
					\addplot[
					only marks,
					nodes near coords,
					point meta = explicit symbolic, 
					color=Black,
					mark=*,
					]
					coordinates {
						(3, 3) [$P$]
						(12, 9) [$P'$]
						(16, 5) [$Q$]
					};
					\draw[->, draw=Black]  (axis cs: 6,5)--(axis cs: 11,8.333)
					node[midway,sloped,above] {$L$};
					\draw[->, very thick, draw=Black]  (axis cs: 4,3.667)--(axis cs: 6,5)
					node[midway,above] {$\vec{U}$};
					\draw[-,draw=Grey]  (axis cs: 3,3) -- (axis cs: 12,9) ;
					\draw[-,draw=Grey]  (axis cs: 12,9) -- (axis cs: 16,5);
					\draw[]  (axis cs: 7,2) to ["$D$"] (axis cs: 7,2);
					\draw[]  (axis cs: 11,8.333)--(axis cs: 11,8.333) node[midway, below] {$s$};
					\draw[dashed, draw=Grey] plot [smooth cycle] coordinates {(axis cs: 1,3) (axis cs: 10,1) (axis cs: 18,3) (axis cs: 12,11)};
				\end{axis}
			\end{tikzpicture}
		\end{figure}
		We know that the directional derivative:
		\begin{align*}
			\dv{u}{s} &= \grad{u} \cdot \vec{U} & \grad{u} &= u_x \hat{i} + u_y \hat{j}
		\end{align*}
		Previously, \(u_x = u_y = 0\), so for all points on \(L\):
		\begin{align*}
			u_x = u_y = 0 \implies \grad{u} = 0 \implies \dv{u}{s} = 0 
			\implies u \text{ constant on } L
		\end{align*}
		Now, that we have established that \(u\) is constant on any given line \(L\) in \(D\), we can see that since \(D\) is simply connected and there are finitely many lines connecting \(P\) and \(Q\), the values of \(u\) at \(P\) and \(Q\) must be equal and constant. Hence, \(\exists a \in \mathbb{R}\) such that \(u(x,y) = a \) in \(D\). Likewise, \(v(x,y) = b\) in \(D\). Thus
		\begin{align*}
			f(z) &= a + bi = c & c \text{ is constant}
		\end{align*}
	\end{proof}
	
	\begin{definition}[Singular Point]
		\label{Singular Point Definition - Complex}
		Let \(\epsilon\) be a neighbourhood of point \(z_0\), and \(f(z)\) be a function. \(z_0\) is a singular point if \(f'(z_0)\) does not exist, but \(f(z)\) is differentiable in all neighbourhoods of \(z_0\).
	\end{definition}
	
	\subsection{Examples} \label{Analytic Functions Examples Subsection - Complex}
	
	\begin{example}[Determining analyticity using Cauchy-Riemann equations]
		Consider \(f(z) = \sin(x)\cosh(y) + i\cos(x)\sinh(y) \).
		\begin{align*}
			u(x,y) &= \sin(x)\cosh(y)	& 	v(x,y) = \cos(x)\sinh(y)
		\end{align*}
		Cauchy-Riemann:
		\begin{align*}
			u_x &= \cos(x)\cosh(y) = v_y	&	u_y = \sin(x)\sinh(y) = -v_x
		\end{align*}
		Therefore, it is clear that \(f(z)\) is entire.
		\begin{align*}
			f'(z) = u_x + iv_x = \cos(x)\cosh(y) - i\sin(x)\sinh(y) 
		\end{align*}
		Another application of Cauchy-Riemann see that \(f'(z)\) is also entire.
	\end{example}

	\begin{example}[\(f(z)\) and \(\overline{f(z)}\) is analytic in \(D \implies f(z)\) is constant in \(D\)]
		Let 
		\begin{align*}
			f(z) &= u(x,y) + iv(x,y) & \overline{f(z)} &= u(x,y) - iv(x,y) = U(x,y) + iV(x,y)
		\end{align*}
		Because of \(f(z)\) and \(\overline{f(z)}\) is analytic in \(D\), the Cauchy-Riemann equations hold:
		\begin{align*}
			u_x &= v_y		&	u_y &= -v_x \\
			U_x &= V_y		&	U_y &= -V_x
		\end{align*}
		We can see that:
		\begin{align*}
			u_x &= -v_y	= v_y &	u_y &= v_x = -v_x
		\end{align*}
		Hence, \(u_x = 0\) and \(v_x = 0\), then we can conclude
		\begin{align*}
			f'(z) = 0 \implies f(z) \text{ is constant in } D
		\end{align*}
	\end{example}

	\begin{example}[\(f(z)\) is analytic in \(D\) and \(\abs{f(z)}\) is constant in \(D \implies f(z)\) is constant in \(D\)]
		Let \(\forall z \in D[\abs{f(z)} = c]\), where \(c\) is a constant. It is easy to see that \(c = 0 \implies \forall z \in D[f(z) = 0]\), so consider \(c \neq 0\). Then
		\begin{align*}
			f(z)\overline{f(z)} = c^2 \neq 0 \implies \forall z \in D[f(z) \neq 0]
		\end{align*}
		Thus
		\begin{align*}
			\overline{f(z)} = \frac{c^2}{f(z)} \qquad \forall z \in D
		\end{align*}
		Hence \(\overline{f(z)}\) is analytic everywhere in \(D\), so \(f(z)\) is constant in \(D\).
	\end{example}
	
	\section{Harmonic Functions} \label{Harmonic Functions - Complex}
	
	\begin{definition}[Laplace's Equation]
		\label{Laplace's Equation Definition - Complex}
		Let \(F(x,y)\) be a real-valued function. That is \(x,y \in \mathbb{R}\). Laplace's equation: 
		\begin{align*}
			\pdv[2]{x} F + \pdv[2]{y} F = 0
		\end{align*}
		In polar form:
		\begin{align*}
			r^2 u_{rr}(r, \theta) + ru_r(r, \theta) + u_{\theta \theta}(r, \theta) &= 0 \\
			r^2 v_{rr}(r, \theta) + rv_r(r, \theta) + v_{\theta \theta}(r, \theta) &= 0
		\end{align*}
		See \cref{Laplace's Equation (Polar) Exercise - Complex}
	\end{definition}

	\begin{definition}[Harmonic]
		A real-valued function \(F(x,y)\) is harmonic in the \(xy\)-plane if it satisfies Laplace's equation.
	\end{definition}

	\begin{theorem}
		Let \(D\) be the domain of a function \(f(z) = u(x,y) + iv(x,y)  \).
		\begin{align*}
			f(z) \text{ is analytic in } D \implies u(x,y) \land v(x,y) \text{ are harmonic in } D
		\end{align*}
	\end{theorem}
	\begin{proof}
		\(f\) is analytic in \(D\), so its component functions must satisfy the Cauchy-Riemann equations: 
		\begin{align*}
			(u_x = v_y)\land(u_y = -v_x) 
				&\implies (u_{xy} = v_{yy})\land(u_{yx} = -v_{xx})  \\
			(u_x = v_y)\land(u_y = -v_x) 
				&\implies (u_{xx} = v_{yx})\land(u_{yy} = -v_{xy}) 
		\end{align*}
		Now, we know from calculus that \(u_{xy} = u_{yx}\) and \(v_{yx} = v_{xy}\), so we conclude
		\begin{align*}
			u_{xx} + u_{yy} &= 0 & v_{xx} + v_{yy} = 0
		\end{align*}
	\end{proof}
	
	\begin{example}
		\label{Laplace's Equation (Polar) Exercise - Complex}
		Let \(f(z) = u(r, \theta) + iv(r, \theta)\) be analytic in domain \(D' = D \setminus \{0\}\). Show \(u(r, \theta)\) and \(v(r, \theta)\) satisfies the polar form of Laplace's equation.
		\begin{proof}{\color{Grey}
			We know from the Polar form of the Cauchy-Riemann equation:
			\begin{align*}
				ru_r &= v_\theta & u_\theta &= -rv_r
			\end{align*}
			Operating by \(r \pdv{r}\) and \(\pdv{\theta}\), we obtain:
			\begin{align*}
				\left(r \pdv{r}\right) r u_r &= ru_r + r^2 u_{rr} = r v_{\theta r} \\
				\left(r \pdv{r}\right) u_\theta &= ru_{\theta r} = \left(r \pdv{r}\right) -rv_r = -rv_r - r^2v_{rr} \\
				\left(\pdv{\theta}\right) ru_{r} &= r u_{\theta r} = v_{\theta \theta} \\
				\left(\pdv{\theta}\right) u_\theta &= u_{\theta \theta} = -r v_{r \theta}
			\end{align*}
			We can see that 
			\begin{align*}
				\begin{cases}
					ru_r + r^2 u_{rr} = -u_{\theta \theta} \\
					rv_r + r^2 v_{rr} = -v_{\theta \theta}
				\end{cases}
				\implies
				\begin{cases}
					r^2 u_{rr} + ru_r + u_{\theta \theta} = 0 \\
					r^2 v_{rr} + rv_r + v_{\theta \theta} = 0
				\end{cases}
			\end{align*}
			}
		\end{proof}
	\end{example}
	
	
	\section{Uniquely Determined Analytic Functions} \label{Uniquely Determined Analytic Functions Section - Complex}
	
	
	\chapter{Conformal Mapping} \label{Conformal Mapping Chapter - Complex}
	
	
	
	
	\part{Ordinary Differential Equations} \label{Ordinary Differential Equations Part}
	
	\part{Nonlinear Dynamics} \label{Nonlinear Dynamics Part}
	
	
	\part{Partial Differential Equations} \label{Partial Differential Equations Part}
	
	\paragraph{Calculus of Variations} \label{Calculus of Variations Part}
	
	\part{Integral Equations} \label{Integral Equations Part}
	
	
	\part{Linear Algebra} \label{Linear Algebra Part}
	
	\chapter{Markov Chains} \label{Markov Chains Chapter - Linear Algebra}
	
	
	\part{Tensors} \label{Tensors Part}
	
	
	\part{Riemann Geometry} \label{Reimann Geometry Part}
	
	
	\part{Abstract Algebra} \label{Abstract Algebra Part}
	
	\chapter{Groups} \label{Groups Chapter - Abstract Algebra}
	
	
	\chapter{Rings} \label{Rings Chapter - Abstract Algebra}
	
	\section{Ideals} \label{Ideals Section - Abstract Algebra}
	
	\chapter{Integral Domains} \label{Integral Domains Chapter - Abstract Algebra}
	
	\chapter{GCD Domains} \label{GCD Domains Chapter - Abstract Algebra}
	
	\chapter{Unique Factorization Domains} \label{Unique Factorization Domains Chapter - Abstract Algebra}
	
	\chapter{Principal Ideal Domains} \label{Principal Ideal Domains Chapter - Abstract Algebra}
	
	\chapter{Fields} \label{Fields Chapter - Abstract Algebra}
	
	
	\part{Galois Theory} \label{Galois Theory Part}
	
	\part{Lie Theory} \label{Lie Algebra Part}
	
	\chapter{Lie Groups}
	
	\chapter{Lie Algebra}
	
	\part{C-Star Algebra} \label{C-Star Algebra Part}
	
	\part{Set Theory} \label{Set Theory Part}
	
	\part{Model Theory} \label{Model Theory Part}
	
	\part{Statistics} \label{Statistics Part}
	\part{Tips and Tricks} \label{Tips and Tricks Part}
	
	\chapter{Integration Techniques} \label{Integration Techniques Chapter - Tips and Tricks}
	
	\section{DI Method (Integration Table)} \label{DI Method Section - Tips and Tricks}
	
	\section{Feynman Integration} \label{Feynman Integration Section - Tips and Tricks}
	
	\backmatter
	\part{Index} \label{Index Part}
	
	\part{Bibliography}
	\bibliographystyle{unsrt}
	\typeout{}
	\bibliography{Bibliography}
	

\end{document}
