\documentclass[12pt, english]{book}
\usepackage[letterpaper, portrait, margin=2.5cm]{geometry} %Set paper type, format, margin
\usepackage[svgnames]{xcolor}
\usepackage{%abstract, 
	amsmath, amssymb, array, babel, booktabs, caption, censor, fancyhdr, flafter, float, framed, gensymb, hyperref, inputenc, indentfirst, mathtools, MnSymbol, multicol, multirow, longtable, lscape, ltablex, polynom, pgfplots, placeins, rotating, scalerel, setspace, stackengine, tabularx, threeparttable, titlesec, titling, wasysym, wrapfig}
\usepackage{cleveref}
\usetikzlibrary{calc,angles,positioning,intersections,quotes,decorations.markings,backgrounds,patterns}
\usepgfplotslibrary{groupplots}
\usepackage{amsthm} %Do not use amsthm with ntheorem unlees ntheorem is using the amsthm option
\usepackage{thmtools}
%\usepackage[standard]{ntheorem}

%\renewcommand{\abstractname}{}    % clear the title
%\renewcommand{\absnamepos}{empty} % originally center 
\setlength{\parindent}{0em} %Set indent on paragraph
\setlength{\parskip}{0.5em} %Set spaces between paragraphs

\DeclarePairedDelimiter\abs{\lvert}{\rvert}%
\DeclarePairedDelimiter\norm{\lVert}{\rVert}%

\newcommand\showdiv[1]{\overline{\smash{\hstretch{.5}{)}\mkern-3.2mu\hstretch{.5}{)}}#1}}
\newcommand\ph[1]{\textcolor{white}{#1}}
\setstackgap{S}{1.5pt}

% Swap the definition of \abs* and \norm*, so that \abs
% and \norm resizes the size of the brackets, and the 
% starred version does not.

% Variables for costomizing theorem enviroment
% \newtheoremstyle{style name}{space above}{space below}{body font}{indent amount}{head font}{head punct}{after head space}{head spec}

%\newtheoremstyle{break}
%	{\topsep}{\topsep}%
%	{\itshape}{}%
%	{\bfseries}{}%
%	{\newline}{}%

%\theoremstyle{break}
%\newtheorem{axiom}{Axiom}[chapter]
%\newtheorem{thm}{Theorem}[section]
%\newtheorem{cor}{Corollary}[thm]
%\newtheorem{lem}[thm]{Lemma}
%\newtheorem{conj}[thm]{Conjecture}
%\newtheorem{definition}{Definition}[section]
%
%\theoremstyle{plain}
%\newtheorem*{rmk}{Remark}
%\newtheorem*{obs}{Observation}
%\newtheorem*{qtn}{Question}
%\newtheorem*{example}{Example}

%\colorlet{headcolour}{DarkOrange}
%\colorlet{rulecolour}{Orange}
%\renewenvironment{leftbar}{%
%	\def\FrameCommand{{\color{rulecolour}\vrule width 3pt} \hspace{10pt}}%
%	\MakeFramed {\advance\hsize-\width \FrameRestore}}%
%{\endMakeFramed}

\declaretheoremstyle[
	spaceabove=6pt, 		
	spacebelow=6pt,
	%	headfont=\normalfont\bfseries,
	notefont=\bfseries, notebraces={}{},
	bodyfont=\slshape,
	postheadspace=1em,
	headpunct=,
	headformat={\NAME~\NUMBER:\NOTE \hfill \smallskip\linebreak},%
	]{axiomstyle}

\declaretheoremstyle[
	spaceabove=6pt, 		
	spacebelow=6pt,
	headfont=\bfseries, %\color{blue},
	notefont=\bfseries, notebraces={}{},
	bodyfont=\slshape \color{DarkBlue},
	postheadspace=1em,
	headpunct=,
	headformat={\NAME~\NUMBER:\NOTE \hfill \smallskip\linebreak},%
	]{theoremstyle}
	
\declaretheoremstyle[
	spaceabove=6pt, 		
	spacebelow=6pt,
	headfont=\bfseries, %\color{Blue}%
	notefont=\bfseries,%
	notebraces={}{},%
	headpunct=,%
	bodyfont=\itshape\color{DarkGreen},%
	headformat={\NAME~\NUMBER:\NOTE \hfill \smallskip\linebreak},%
	%	preheadhook=\begin{leftbar},%
	%	postfoothook=\end{leftbar},%
	]{minorstyle}
	
\declaretheoremstyle[
	spaceabove=6pt, 		
	spacebelow=6pt,
	headfont=\bfseries,%\color{Blue},%
	notefont=\bfseries,%
	notebraces={}{},%
	headpunct=,%
	bodyfont=\slshape \color{OrangeRed},%
	headformat={\NAME~\NUMBER:\NOTE \hfill \smallskip\linebreak},%
%	preheadhook=\begin{leftbar},%
%	postfoothook=\end{leftbar},%
	]{defstyle}
	
\declaretheoremstyle[
	spaceabove=6pt, 		
	spacebelow=6pt,
	%	headfont=\normalfont\bfseries,
	bodyfont=\slshape,
%	postheadspace=1em,
	headpunct=,
%	headformat={\NAME~\NUMBER:\NOTE \hfill \smallskip\linebreak},%
	]{examplestyle}

\declaretheorem{axiom}[
	numberwithin=chapter, style=axiomstyle, parent=chapter,
	]
	
\declaretheorem{theorem, conjecture}[
	numberwithin=section, style=theoremstyle, parent=section,
	]
	
\declaretheorem{lemma, proposition, corollary}[
	style=minorstyle,
	parent=theorem,
%	numberwithin=section
	]

\declaretheorem{definition}[parent=section, style=defstyle]

\declaretheorem{remark, observation, question}[numbered=no]

\declaretheorem{example}[numberwithin=section, style=examplestyle]

% Indenting the Proof enviroment
\makeatletter
\renewenvironment{proof}[1][\proofname]{\par
	\pushQED{\qed}%
	\normalfont \topsep6\p@\@plus6\p@\relax
	\list{}{%
		\settowidth{\leftmargin}{\itshape\proofname:\hskip\labelsep}%
		\setlength{\labelwidth}{0pt}%
		\setlength{\itemindent}{-\leftmargin}%
	}%
	\item[\hskip\labelsep\itshape#1\@addpunct{:}]\ignorespaces
}{%
	\popQED\endlist\@endpefalse
}
\makeatother

% Letters after part
\makeatletter
\renewcommand{\@endpart}{\vfil\newpage}
\makeatother
\newenvironment{partintro}
{\vspace*{\fill}
	\section*{\centering Resources used in part \thepart}
	\begin{quotation}}
	{\end{quotation}\vspace*{\fill}\newpage}
\newcommand{\nopartintro}{%
	\vspace*{\fill}
	\thispagestyle{empty}
	\newpage
}


\begin{document}
	\title{The Book of Math (Notes)}
	\author{Kevin Kuo}
%	\renewcommand{\abstractname}{\vspace{-\baselineskip}}
	
	\pagestyle{fancy}
	\fancyhead{} % clear all header fields
	\fancyhead[LO]{  }
	\fancyhead[CO]{  }
	\fancyhead[RO]{  }
	\renewcommand{\headrulewidth}{0pt}
	
	\fancyfoot{} % clear all footer fields
	\fancyfoot[LO]{}
	\fancyfoot[CO]{\thepage}
	\fancyfoot[RO]{}
	\renewcommand{\footrulewidth}{0pt}
	\vspace{0cm}	
	
	\frontmatter
	
	\maketitle
	
	\newpage
	\section*{Forward and Disclaimer}
	These are math notes made by a student (with a physics major and math minor) based off text books. It may contain misconceptions and misinterpretations, thus should not be viewed in the same light of a text book. Use at your own risk and mental sanity.
	
	\section*{Symbols}
	\begin{tabularx}{\textwidth}{ l c p{0.6\linewidth}}
		\multicolumn{3}{l}{\textbf{{\large Logic}}} \\ [10pt]
		\hline
		\textbf{Name} & \textbf{Symbol} & \textbf{Comment} \\
		\hline
		Exists 					& $\exists$ 		& There exists at least one\\
		For all 				& $\forall$ 		& \\
		Not exists 				& $\nexists$ 		& There does not exist\\ 
		Exists one				& $\exists!$ 		& There only exists one and only one \\
		And 					& $\land$			& \\
		Or						& $\lor$			& Inclusive or \\
		Not 					& $\neg$			& \\
		Logically implies 		& $\implies$ 		& If \\
		Logically implied by 	& $\Longleftarrow$ 	& Only if \\  
		Logically equivalent 	& $\iff$ 			& If and only if \\
		Implies 				& $\longrightarrow$	& \\
		Implied by 				& $\longleftarrow$ 	& \\  
		Double Implication 		& $\longleftrightarrow$	& \\
		\hline	
		
		& & \\
		\multicolumn{3}{l}{\textbf{{\large Set Notation}}} \\ [10pt]
		\hline
		\textbf{Name} & \textbf{Symbol} & \textbf{Comment} \\
		\hline
 		Empty Set 				& $\emptyset$ 		& The set that is empty \\
 		Natural Numbers 		& $\mathbb{N}$		& Set of natural numbers not containing 0, equivalent to the set of positive integers \\
 		Integers 				& $\mathbb{Z}$		& Set of integers \\
 		Rational Numbers 		& $\mathbb{Q}$		& \\
 		Algebraic Numbers		& $\mathbb{A}$		& \\
 		Real Numbers 			& $\mathbb{R}$		& \\
 		Complex Numbers 		& $\mathbb{C}$		& \\
 		
 		In 						& $\in$ 			& \\
 		Not in 					& $\nin$			& \\
 		Owns 					& $\ni$				& Has an element \\
 		
 		Proper Subset 			& $\subset$			& Subset that is not itself \\
 		Subset 					& $\subseteq$		& \\
 		Superset 				& $\supset$ 		& Superset that is not itself\\
 		Proper Superset 		& $\supseteq$		& \\
 		Power set				& $\wp$				& \\
 		Union 					& $\cup$			& \\
 		Intersection			& $\cap$			& \\
 		Difference				& $\setminus$		& \\
 		\hline
 		
 		& & \\
 		\multicolumn{3}{l}{\textbf{{\large Relationships}}} \\ [10pt]
 		\hline
 		\textbf{Name} & \textbf{Symbol} & \textbf{Comment} \\
 		\hline
 		Defined 				& $\doteq$ 			& \\
 		Approximate 			& $\approx$			& \\
 		Equivalent				& $\equiv$	 		& Isomorphic (Group Theory) \\
 		Congruent 				& $\cong$			& Homomorphic (Group Theory) \\
 		Proportional 			& $\propto$			& \\
 		\hline
 		
 		& & \\
 		\multicolumn{3}{l}{\textbf{{\large Operators}}} \\ [10pt]
 		\hline
 		\textbf{Name} & \textbf{Symbol} & \textbf{Comment} \\
 		\hline
 		& $\oplus$ & \\
 		& $\otimes$ & \\
 		& $\odot$ & \\
 		& $\circ$ & Convolution \\
 		Dagger& $\dagger$ & Complex conjugate transpose of a matrix \\
 		\hline
 		
 		& & \\
 		\multicolumn{3}{l}{\textbf{{\large Arrows}}} \\ [10pt]
 		\hline
 		\textbf{Name} & \textbf{Symbol} & \textbf{Comment} \\
 		\hline
 		Maps to 				& $\mapsto$			& \\
 		\hline
 		
 		& & \\
 		\multicolumn{3}{l}{\textbf{{\large Hebrew}}} \\ [10pt]
 		\hline
 		\textbf{Name} & \textbf{Symbol} & \textbf{Comment} \\
 		\hline
 		Aleph					& $\aleph$			& Carnality of infinite sets that can be well ordered \\
 		\hline
 		
 		& & \\
 		\multicolumn{3}{l}{\textbf{{\large Other}}} \\ [10pt]
 		\hline
 		\textbf{Name} & \textbf{Symbol} & \textbf{Comment} \\
 		\hline
 		Real part 				& $\Re$				& Real part of a number \\
 		Imaginary part 			& $\Im$				& Imaginary part of a number \\
 		\hline
	\end{tabularx}

	\newpage
	\section*{Book Constitution}
	\subsection*{Intents and Purpose}
	The goal of this book is to organize mathematical knowledge of topics related to the study of physics or the author's interest. It is meant to be used as a source of for future reference, not as a textbook for students new to the topics. It is a notebook of a student, thus should be treated as one and not as a textbook. At most, it could be used as a study guide along side a textbook. Definitely not as the main source for acquiring knowledge. 
	
	\subsection*{Layout and Organization}
	The book is split into parts each containing a field of study mathematics, or a topic large enough to justify giving it its own part. Each part contains chapters that focuses on a particular topic required to understand the field, with sections dedicated to describing a particular knowledge required for the topic. 
	
	As axioms, definitions, theorems, corollary, and proofs are integral and abundant to the study of mathematics, each will have a unique style. Each environment and its styles are displayed as follows: 
	
	\begin{axiom}[Axiom name]{Example Axiom}
		Axioms are the ``ground rules" of the set.
	\end{axiom}

	\begin{theorem}[Theorem name or citation]{Example Theorem}
		An important logical result from the axioms, with proof.
	\end{theorem}

	\begin{conjecture}[Name of conjecture or citation]{Example Conjecture}
		A hypothesis, without proof.
	\end{conjecture}
	
	\begin{corollary}{Example Corollary}
		An implication as a result of a theorem.
	\end{corollary}

	\begin{lemma}{Example Lemma}
		Small theorems that build up to a larger theorem. 
	\end{lemma}
	
	\begin{proposition}{Example Proposition}
		Example proposition.
	\end{proposition}
	
	\begin{proof}
		Logical deductions that results in a theorem.
		\textcolor{Grey}{Proofs I've written will be in grey, which may or may not be correct.}
	\end{proof}

%	\theoremstyle{break}
	\begin{definition}[Word]{Example Definition}
		The definition of a word.
	\end{definition}	

	\begin{example}
		An example.
	\end{example}

	\begin{remark}{Remark}
		A comment by the author in the textbooks used.
	\end{remark}
	
	\begin{observation}{Example Observation}
		A remark by me.
	\end{observation}

	\begin{question}{Example Question}
		A question from me for a mystery to be answered later. 
	\end{question}

	
	\section*{}
	\tableofcontents
	
	\mainmatter
	\part{Logic} \label{Logic Part}
	
	\chapter{Proofs}
	
	
	\part{Numbers} \label{Numbers Part}
	\begin{partintro}
		content...
	\end{partintro}
	
	\chapter{Natural $\mathbb{N}$} \label{Natural Chapter - Numbers}
	
	\chapter{Integers $\mathbb{Z}$} \label{Integers Chapter - Numbers}
	
	\chapter{Rationals $\mathbb{Q}$} \label{Rationals Chapter - Numebers}
	
	\chapter{Constructible} \label{Constructible Chapter - Numbers}
	
	\chapter{Algebraic $\mathbb{A}$} \label{Algebraic Chapter - Numbers}
	
	\chapter{Reals $\mathbb{R}$} \label{Reals Chapter - Numbers}
	
	\chapter{Complex $\mathbb{C}$} \label{Complex Chapter - Numbers}
	

	
	\part{Real Analysis} \label{Real Analysis Part}
	\begin{partintro}
		\begin{itemize}
			\item[1.] Kenneth A. Ross - Elementary Analysis (2nd Ed.) \cite{Ross.K-Elementary-Analysis-2013}
		\end{itemize}
	\end{partintro}

	\chapter{Metric Spaces}	
	
	\part{Complex Analysis} \label{Complex Analysis Part}
	\begin{partintro}
	\noindent Primary:
			\begin{itemize}
				\item[1.] Brown and Churchill - Complex Variables and Applications \cite{Brown.J;Churchill.R-Complex-Variables-2014}
			\end{itemize}
			Supplement: 
			\begin{itemize}
				\item[1.] A. David Wunsch - Complex Variables with Applications \cite{Wunsh.A-Complex-Variables-2005}
			\end{itemize}
	\end{partintro}
	
	\chapter{Basics} \label{Basics Chapter - Complex}
	\section{Complex Numbers} \label{Complex Numbers Section - Complex}
	$$\mathbb{C} = \{x + iy \mid x, y \in \mathbb{R}, i = \sqrt{-1}\}$$
	Complex numbers are elements of the complex field $(\mathbb{C})$, therefore, they obey all the properties of a field. 
	
	We will denote complex numbers by $z = x + iy$ with $x, y \in \mathbb{R}$, and refer the real part as $\Re(z) = \operatorname{Re}(z) = x$ and imaginary part as $\Im(z) = \operatorname{Im}(z) = y$. Complex numbers can also be defined as an ordered pair $z = (x, y)$ which is interpreted as points in the complex plane. $(x, 0)$ are points on the real axis while $(0 , y)$ are points in the imaginary axis. 
	
	\begin{center}
		\begin{tikzpicture}
		\begin{axis}[
			small,
			%			title={},
			xlabel={$\Re$},
			ylabel={$\Im$},
			xmin=0, xmax=10,
			ymin=0, ymax=10,
			xtick={100},
			ytick={100},
			axis lines = left,
%			legend pos=north west,
%			ymajorgrids=true,
%			grid style=dashed,
			]
			
			\addplot[
			only marks,
			nodes near coords,
			point meta = explicit symbolic, 
			color=Black,
			mark=*,
			]
			coordinates {
				(7, 5) [$z = x + iy$]
			};
%			\legend{$z = x + iy$}
		\end{axis}
	\end{tikzpicture}
	\end{center}
	We add and multiply complex numbers in the usual way: 
	\begin{align*}
		z_1 + z_2 &= (x_1 + iy_1) + (x_2 + iy_2) & z_1 z_2 &= (x_1 + iy_1) (x_2 + iy_2) \\
			&= (x_1 + x_2) + i(y_1 + y_2) & &=(x_1 x_2 - y_1 y_2) + i(x_1 y_2 + x_2 y_1)
	\end{align*}
	
	$\forall z \in \mathbb{C}$, there is an unique additive inverse $(-z)$ and $\forall z \in \mathbb{C}\setminus\{0\}$, there is an unique multiplicative inverse $(z^{-1})$ such that 
	\begin{align*}
		&z + (-z) = 0  & &zz^{-1} = 1 \\
		&\implies -z = -x - iy & &\implies (x_1 x_2 - y_1 y_2) = 1 \land (x_1 y_2 + x_2 y_1) = 0 \\
		& & &\implies z^{-1} = \frac{x_1}{x_1^2 + y_1^2} - i \frac{y_1}{x_1^2 + y_1^2}
	\end{align*}
	The existence and uniqueness of the inverses can be easily proven. 
	
	The addition of complex numbers may also be interpreted as akin to vector addition. 
	\begin{center}
		\begin{tikzpicture}
			\begin{axis}[
				small,
				%			title={},
				xlabel={$\Re$},
				ylabel={$\Im$},
				xmin=0, xmax=10,
				ymin=0, ymax=10,
				xtick={100},
				ytick={100},
				axis lines = left,
				%			legend pos=north west,
				%			ymajorgrids=true,
				%			grid style=dashed,
				]
				
				\addplot[
				only marks,
				nodes near coords,
				point meta = explicit symbolic, 
				color=Black,
				mark=*,
				]
				coordinates {
					(7, 5) [$z_1$]
					(1, 4) [$z_2$]
					(8, 9) [$z_1 + z_2$]
				};
				%			\legend{$z = x + iy$}
				\draw[line width=1pt,Grey,-stealth](0,0)--(70,50);
				\draw[line width=1pt,Grey,-stealth](0,0)--(10,40);
				\draw[line width=1pt,Grey,-stealth](70,50)--(80,90);
				\draw[line width=1pt,black,-stealth](0,0)--(80,90);
			\end{axis}
	\end{tikzpicture}
	\end{center}
	
	\section{Triangle Inequality} \label{Triangle Inequality Section - Complex}
	It is not analysis without a section dedicated to the triangle inequality.
	
	\begin{definition}[Modulus]
		$\abs{z} = \sqrt{x^2 + y^2} = \sqrt{z \bar{z}}$
	\end{definition}
	It is obvious why the definition is not $\abs{z} = \sqrt{x^2 + (iy)^2}$ as problems arise when $x = y$. The modulus is the distance of $z$ from $(0, 0)$. $\bar{z}$ is the complex conjugate of $z$, which is explored in \cref{Complex Conjugate Section - Complex}
	
	\begin{theorem}[Triangle Inequality]
		$\forall z_1, z_2 \in \mathbb{C} [\abs{z_1 + z_2} \leq \abs{z_1} + \abs{z_2}]$
		\label{Triange Inequality - Complex}
	\end{theorem}
	
	From the theorem, we can derive a similar inequality: 
	\begin{align*}
		\abs{z_1} = \abs{z_1 + z_2 - z_2} \leq \abs{z_1 + z_2} + \abs{-z_2}
		 &\implies \abs{z_1} - \abs{z_2} \leq \abs{z_1 + z_2}
	\end{align*}
	
	An important property of polynomials is observed when \cref{Triange Inequality - Complex} is applied to polynomials.
	
	\begin{corollary}
		Consider the polynomial $P(z)$ where $a_n \in \mathbb{C}$, $n \in \mathbb{N}$, $a_0 \neq 0$, and $z \in \mathbb{C}$.
		$$P(z) = a_0 + a_1 z + a_2 z^2 + \ldots + a_n z^n$$
		Then $\forall z, \exists R \in \mathbb{R}_{>0}, \abs{z} < R$ such that
		$$\abs*{\frac{1}{P(z)}} < \frac{2}{\abs{a_n} R^n}$$
		\label{Complex Poly Reciprocal Bounded}
	\end{corollary}
	\begin{proof}
		Consider 
		\begin{align*}
			w &= \frac{P(z)}{z_n} - a_n = \frac{a_0}{z^n} + \frac{a_1}{z^{n-1}} + \ldots + \frac{a_{n-1}}{z} 
				& z \neq 0 \\
			&\implies wz^n = a_0 + a_1 z + \ldots + a_{n-1}z^{n-1} \\
			&\implies \abs{w}\abs{z}^n \leq \abs{a_0} + \abs{a_1}\abs{z} + \ldots + \abs{a_{n-1}}\abs{z}^{n-1} \\
			&\implies \abs{w} \leq \frac{\abs{a_0}}{\abs{z}^n} + \frac{\abs{a_1}}{\abs{z}^{n-1}} + \ldots + \frac{\abs{a_{n-1}}}{\abs{z}} 
				& \\
			&\implies \abs{w} < n\frac{\abs{a_n}}{2n} = \frac{\abs{a_n}}{2} 
				& \exists \text{ sufficiently large } R < \abs{z}  \text{ s.t.}\\
			& 	& \forall m, \ 0 \leq m \leq n-1, \ \frac{\abs{a_m}}{\abs{z}^{n-m}} < \frac{\abs{a_n}}{2n} \\
			&\implies \abs{a_n + w} \geq \abs{\abs{a_n} - \abs{w}} > \frac{\abs{a_n}}{2}
				& R < \abs{z} \\
			&\implies \abs{P_n(z)} = \abs{a_n + w} \abs{z}^n > \frac{\abs{a_n}}{2}\abs{z}^n > \frac{\abs{a_n}}{2} R^n 
				& R < \abs{z} \\
			&\implies \abs*{\frac{1}{P(z)}} < \frac{2}{\abs{a_n} R^n}
		\end{align*}
	\end{proof}
	This tells us that if $z$ is a solution to a polynomial $P(z)$, then the reciprocal of the polynomial $1/P(z)$ is bounded above by $R = \abs{z}$. (i.e. It is bounded by a circle of radius $\abs{z}$.)
	
	\section{Polar and Exponential Form} \label{Polar and Exponential Form Section - Complex}
	\begin{definition}[Argument of $z$]
		Consider any $z \in \mathbb{C}$ where $z \neq 0$.
		Let $\theta$ be the angle in radians between $z$ and the real axis .
		Then $\forall n \in \mathbb{N}$, $0 \leq \theta < 2\pi$, the argument of $z$:
		$$ \operatorname{arg}(z) = \theta + 2 n \pi$$
		\label{Argument - Complex}
	\end{definition}
	
	We know $\forall n \in \mathbb{N}$, $\theta + 2 \pi n = \theta$. This leads us to the definition of the principal argument of $z$.
	\begin{definition}[Principal Argument of $z$]
		Consider any $z \in \mathbb{C}$ where $z \neq 0$.
		Let $\theta$ be the angle in radians between $z$ and the real axis.
		Then for $0 \leq \theta < 2\pi$, the principal argument of $z$:
		$$\operatorname{Arg}(z) = \theta$$
		\label{Principal Argument - Complex}
	\end{definition}
	It is clear that $\operatorname{arg}(z) = \operatorname{Arg}(z) + 2n \pi$.
	\begin{definition}[Polar Form of $z$]
		Consider $z \in \mathbb{C}$. Let $r = \abs{z}$, and $\theta = \operatorname{arg}(z)$. 
		Then $\forall z \in \mathbb{C}, z \neq 0$:
		$$z = x + iy = r(\cos(\theta) + i \sin(\theta))$$
		\label{Polar Form of z - Complex}
	\end{definition}

	Notice that all three definitions require that $z \neq 0$ as $\theta$ is undefined at $z = 0$.
	
	\begin{theorem}[Euler's Formula]
		$$e^{i \theta} = \cos(\theta) + i \sin(\theta)$$
		\label{Euler's Formula - Complex}
	\end{theorem}
	Combining \cref{Polar Form of z - Complex} with \cref{Euler's Formula - Complex}, we obtain the Exponential Form of $z$: 
	
	\begin{definition}[Exponential Form of $z$]
	Consider any $z \in \mathbb{C}$, and let $r = \abs{z}$ and $\theta = \operatorname{Arg}(z)$. Then the exponential form of $z$:
		$$z = r e^{i \theta}$$
		\label{Exponential Form of z - Complex}
	\end{definition}
	Note: $\theta = \tan^{-1}(y/x)$ and $r = \sqrt{x^2 + y^2}$.
	\begin{center}
		\begin{tikzpicture}
			\begin{axis}[
				small,
				%			title={},
				xlabel={$\Re$},
				ylabel={$\Im$},
				xmin=0, xmax=10,
				ymin=0, ymax=10,
				xtick={100},
				ytick={100},
				axis lines = left,
				%			legend pos=north west,
				%			ymajorgrids=true,
				%			grid style=dashed,
				]
				
				\addplot[
				only marks,
				nodes near coords,
				point meta = explicit symbolic, 
				color=Black,
				mark=*,
				]
				coordinates {
					(7, 5) [$z = x + iy = re^{i\theta}$]
				};
				%			\legend{$z = x + iy$}
				\draw[thick]  (0,0) to ["$r=\abs{z}$"] (70,50);
				% angle
				\draw[draw=blue] (0,0) ++(35:50) arc (35:0:50)
				node[midway,above right,inner sep=2pt,font={\footnotesize}]{$\theta$};
			\end{axis}
		\end{tikzpicture}
	\end{center}
	
	\subsection{Properties of Polar and Exponential Form} \label{Properties of Polar and Exponential Form Subsection - Complex}
	It would be easier to work with the exponential form of $z$ then convert it to the polar form later. The exponential form of a complex number is part of the exponential family of functions, thus possess all the properties of the family. Consider any complex number $z_1 = r_1 e^{i\theta_1}$ and $z_2 = r_2 e^{i\theta_2}$.
	\begin{align*}
		z_1  z_2 &= r_1 r_2 e^{i(\theta_1 + \theta_2)} 
			& z^n &= r^n e^{i n \theta} \qquad \forall n \in \mathbb{Z}
	\end{align*}
	A special case arrives for integer exponential of $z$ on the unit circle.
	\begin{theorem}[de Moivre's Formula]
		Consider any $z = e^{i \theta} \in \mathbb{C}$ on the unit circle, and let $n \in \mathbb{Z}$.
		\begin{align*}
			\forall z\in \mathbb{C} \ \forall n \in \mathbb{Z}
			\left[\abs{z} = 1 \implies (\cos(\theta) + i\sin(\theta))^n = \cos(n \theta) + i\sin(n \theta)\right]
		\end{align*}
		\label{de Moivre's Formula Theorem - Complex}
	\end{theorem}
	\begin{proof}
		Consider $z = e^{i \theta}$ and let $n \in \mathbb{Z}$. 
		\begin{align*}
			z^n = (e^{i \theta})^n = e^{in\theta} = \cos(n\theta) + i\sin(n\theta)
		\end{align*}
	\end{proof}

	The proof hints that \cref{de Moivre's Formula Theorem - Complex} can be generalized to $\forall n \in \mathbb{R}$, which we will see shortly in \cref{Roots of z Section - Complex}. Using \cref{de Moivre's Formula Theorem - Complex}, we can obtain the double angle identities.
	
	\begin{corollary}[Double Angle Identities]
		\begin{align*}
			\cos(2 \theta) &= \cos^2(\theta) - \sin^2(\theta) 
				& \sin(2\theta) &= 2\sin(\theta)\cos(\theta)
		\end{align*}
	\end{corollary}
	\begin{proof}
		Consider any $z$ on the unit circle, that is $z=e^{i\theta}$.
		\begin{align*}
			&(\cos(\theta) + i \sin(\theta))^2 = \cos(2\theta) + i \sin(2\theta)
				&\text{\Cref{de Moivre's Formula Theorem - Complex}} \\
			&\implies \cos^2(\theta) - \sin^2(\theta) + i2\sin(\theta)\cos(\theta) = cos(2\theta) + i\sin(2\theta) 
		\end{align*}
		Equating the real and imaginary parts yield the desired results. 
	\end{proof}

	\subsection{Properties of Arguments} \label{Properties of Argumetns Subsection - Complex}
	Recall from \cref{Properties of Polar and Exponential Form Subsection - Complex}: 
	\begin{align*}
		z_1  z_2 &= r_1 r_2 e^{i(\theta_1 + \theta_2)} 
		& z^n &= r^n e^{i n \theta} \qquad \forall n \in \mathbb{Z}
	\end{align*}
	The arguments for the arguments of products of any $z_1, z_2 \in \mathbb{C}$ follows immediately from the properties of the exponential.
	\begin{corollary}[Arguments of Products]
		\begin{align*}
			\arg(z_1 z_2) &= \arg(z_1) + \arg(z_2)
				& \operatorname{Arg}(z_1 z_2) = \operatorname{Arg}(z_1) + \operatorname{Arg}(z_2) \\
			\arg(z^n) &= n\arg(z)  
				& \operatorname{Arg}(z^n) = n\operatorname{Arg}(z)
		\end{align*}
		\label{Arguments of Products Corollary - Complex}
	\end{corollary}
	\begin{proof}
		\begin{align*}
			z_1  z_2 &= r_1 r_2 e^{i(\theta_1 + \theta_2)} & \\
			&\implies \arg(z_1 z_2) = \arg(z_1) + 2n_1 \pi + \arg(z_2) + 2n_2\pi
				& n_1, n_2 \in \mathbb{Z} \\
			&\implies \arg(z_1 z_2) = \arg(z_1) + \arg(z_2) & \\
			&\implies \operatorname{Arg}(z_1 z_2) = \operatorname{Arg}(z_1) = \operatorname{Arg}(z_2)  \\
			\\
			z^n &= r^n e^{i n\theta} & \\
			&\implies \arg(z^n) = n\arg(z) + 2n\pi & n \in \mathbb{Z} \\
			&\implies \arg(z^n) = n\arg(z)  \\
			&\implies \operatorname{z^n} = n\operatorname{Arg}(z)
		\end{align*}
	\end{proof}
	It is clear that: 
	\begin{align*}
		\arg\left(\frac{z_1}{z_2}\right) &= \arg(z_1) - \arg(z_2)
			& \operatorname{Arg}\left(\frac{z_1}{z_2}\right) &= \operatorname{Arg}(z_1) - \operatorname{Arg}(z_2)
	\end{align*}

	\section{Roots of $z$} \label{Roots of z Sections - Complex}
	In \cref{Exponential Form of z - Complex}, you might be wondering why $z^n = r^n e^{i n \theta}$ is not for $n \in \mathbb{R}$. That is because there is more things to consider, which we will explore in this section. Recall that $z = re^(i \theta) = re^{i (\theta + 2n \pi)}$ for $n \in \mathbb{Z}$. 
	
	\begin{definition}[Exponential of $z$]
		Consider any $z \in \mathbb{C}$ and any $x \in \mathbb{R}$
		$$z^x = \left(r e^{i(\theta + 2 n \pi)} \right)^x = r^x e^{i x(\theta + 2n \pi)}$$
		\label{Exponential of z Definiion - Complex}
	\end{definition}

	For $x \nin \mathbb{Z}$, it is clear that $z^x =  r^x e^{ix(\theta + 2 n \pi)} \neq r^x e^{ix\theta}$, since $2nx\pi = 0 \iff nx \in \mathbb{Z}$. In order to define the roots of $z$ we must need a more general and proper definition of $z$.
	
	\begin{definition}[Roots of $z_0$]
		\label{Roots of z Definition - Complex}
		Consider any $z_0 \in \mathbb{C}$ and any $m \in \mathbb{N}$.
		$$z_0^{\frac{1}{m}} = r_0^\frac{1}{m} e^{i\left(\frac{\theta_0 + 2n \pi}{m}\right)} = r_0^\frac{1}{m} e^{i \left(\frac{\theta_0}{m} + \frac{2n \pi}{m}\right)}$$
	\end{definition}  
	
	Taking the $m$-th root of $z_0 \in C$ scales $\theta_0$ by $1/m$, and provides solutions at equally spaced by $2\pi / m$ on a circle of radius $r^{1/m}$. That is, the roots lie on the vertices of a regular n-sided polygon inscribed in a circle of radius $\abs{z}^{1/m}$. 
	
	\begin{example}
		Consider $z_0 = 32 e^{i(5/6)\pi}$, then $z_0^{(1/5)} = 3e^{i(\pi/6) + i(2/5) n \pi}$ for $n \in \mathbb{Z}$. The radius went from $35$ to $35^{(1/5)} = 2$, and five roots appear equally spaced with distance of $(2/5)\pi$ on a circle with radius $2$. Before and after graphs are as follows, note graph on right is zoomed in: 
		\begin{figure}[H]
			\centering
			\begin{tikzpicture}[remember picture]
				\begin{axis}[
					width=8.9cm,
					unit vector ratio=1 1 1,
					xlabel={$\Re$},
					ylabel={$\Im$},
					xmin=-30, xmax=30,
					ymin=-30, ymax=30,
%					xtick={-20,-10,10,20},
%					ytick={-20,-10,10,20},
					xtick=100,
					ytick=100,
					axis lines = middle,
					]
					
					\addplot[
					only marks,
					nodes near coords,
					point meta = explicit symbolic, 
					color=Black,
					mark=*,
					]
					coordinates {
						(-27.7128, 16) [$z_0 = 32e^{i(5/6)\pi}$]
					};
				\end{axis}
				\path (current bounding box.north east) -- 
				(current bounding box.south east) coordinate[midway] (2BL);
			\end{tikzpicture}
			\hspace*{2.8cm}
			\begin{tikzpicture}[remember picture]
				\begin{axis}[
%					small,
					width=8.9cm,
					unit vector ratio=1 1 1,
					xlabel={$\Re$},
					ylabel={$\Im$},
					xmin=-3.1, xmax=3.1,
					ymin=-3.1, ymax=3.1,
%					xtick distance=2,
%					ytick distance=2,
					xtick=10,
					ytick=10,
					axis lines = middle,
					axis line style = Black,
					]
					\addplot[
					only marks,
					nodes near coords,
					point meta = explicit symbolic, 
					color=Black,
					mark=*,
					]
					coordinates {
						(1.73205, 1) [$2e^{i\pi/6}$]
						(-0.4158, 1.9563) [$2e^{i17\pi/30}$]
						(-1.9890, 0.20906) [$2e^{i29\pi/30}$]
						(-0.8135, -1.8271) [$2e^{-i19\pi/30}$]
						(1.4863, -1.3383) [$2e^{-i7\pi/30}$]
					};
					\draw[line width=1pt,LightGrey](axis cs: 0, 0) circle [radius=200];
					\draw[line width=1pt,LightGrey](axis cs: 0,0)--(axis cs: 1.73205, 1);
					\draw[line width=1pt,LightGrey](axis cs:0,0)--(axis cs:-0.4158, 1.9563);
					\draw[line width=1pt,LightGrey](axis cs:0,0)--(axis cs:-1.9890, 0.20906);
					\draw[line width=1pt,LightGrey](axis cs:0,0)--(axis cs:-0.8135, -1.8271);
					\draw[line width=1pt,LightGrey](axis cs:0,0)--(axis cs:1.4863, -1.3383);
				\end{axis}
				\path (current bounding box.north west) -- 
				(current bounding box.south west) coordinate[midway] (2BR);
			\end{tikzpicture}
		\end{figure}
		\tikz[overlay,remember picture]{\draw[->, very thick] 
			($(2BL)+(0.5, 0)$) -- ($(2BR)+(-0.5,0)$)
			node[midway,above,text width=2cm]{$f(z) = z^\frac{1}{5}$};} 
	\end{example} 
	
	We can see that the roots of $z_0$ form a set:
	\begin{definition}[Set of roots of $z_0$]
		\label{Set of roots of z - Complex}
		Consider the $m$-th root of any $z_0 \in \mathbb{C}$. Let: 
		\begin{align*}
			z_0 &= r_0 e^{i\theta_0}
			&c_0 &= r_0^{1/m} e^{i\theta_0/m} 
			&\omega_n &= e^{\frac{i2\pi}{m}} & \ m \in \mathbb{N}
		\end{align*}
		Then the set of roots of $z_0$:
		\begin{align*}
			z_0^{1/m} = \left\{c_k = c_0 \omega_m^k \mid k\in \mathbb{N}, \ 0 \leq k < m\right\}
		\end{align*}
	\end{definition}
	$c_0$ is the principal root. The root corresponding to the principal argument of $z$.
	\begin{definition}[Principal Root]
	Consider the $m$-th root of any $z_0 \in \mathbb{C}$. The principal root of $z_0$ is defined as:
		$$c_0 = r_0^{\frac{1}{m}} e^{i\frac{\theta_0}{m}}$$
	\end{definition}
	
	\begin{example}
		Recall from the previous example: $z_0 = 32 e^{i(5/6)\pi}$. This gives us
		\begin{align*}
			c_0 &= 32^{1/5} e^{i\pi/6}= 2 e^{i\pi/6}
				&\omega_5 &= e^{i2\pi/5}
		\end{align*}
		Then
		\begin{align*}
			c_0 &= c_0 \omega_5^0 = 2 e^{i\pi/6} \\
			c_1 &= c_0 \omega_5^1 = 2 e^{i\pi/6} e^{i2\pi/5} = 2 e^{i17\pi/30} \\
			c_2 &= c_0 \omega_5^1 = 2 e^{i\pi/6} e^{i4\pi/5} = 2 e^{i29\pi/30} \\
			c_3 &= c_0 \omega_5^1 = 2 e^{i\pi/6} e^{i6\pi/5} = 2 e^{i41\pi/30} = 2 e^{-i19\pi/30}\\
			c_4 &= c_0 \omega_5^1 = 2 e^{i\pi/6} e^{i8\pi/5} = 2 e^{i53\pi/30} = 2 e^{-i7\pi/30}\\
		\end{align*}
		\begin{figure}[H]
			\centering
			\begin{tikzpicture}[remember picture]
				\begin{axis}[
					%					small,
					width=10cm,
					unit vector ratio=1 1 1,
					xlabel={$\Re$},
					ylabel={$\Im$},
					xmin=-3.1, xmax=3.1,
					ymin=-3.1, ymax=3.1,
					%					xtick distance=2,
					%					ytick distance=2,
					xtick=10,
					ytick=10,
					axis lines = middle,
					axis line style = Black,
					]
					\addplot[
					only marks,
					nodes near coords,
					point meta = explicit symbolic, 
					color=Black,
					mark=*,
					]
					coordinates {
						(1.73205, 1) [$c_0 = 2e^{i\pi/6}$]
						(-0.4158, 1.9563) [$c_1 = 2e^{i17\pi/30}$]
						(-1.9890, 0.20906) [$c_2 = 2e^{i29\pi/30}$]
						(-0.8135, -1.8271) [$c_3 = 2e^{-i19\pi/30}$]
						(1.4863, -1.3383) [$c_4 = 2e^{-i7\pi/30}$]
					};
					\draw[line width=1pt,LightGrey](axis cs: 0, 0) circle [radius=200];
					\draw[line width=1pt,LightGrey](axis cs: 0,0)--(axis cs: 1.73205, 1);
					\draw[line width=1pt,LightGrey](axis cs:0,0)--(axis cs:-0.4158, 1.9563);
					\draw[line width=1pt,LightGrey](axis cs:0,0)--(axis cs:-1.9890, 0.20906);
					\draw[line width=1pt,LightGrey](axis cs:0,0)--(axis cs:-0.8135, -1.8271);
					\draw[line width=1pt,LightGrey](axis cs:0,0)--(axis cs:1.4863, -1.3383);
				\end{axis}
				\path (current bounding box.north west) -- 
				(current bounding box.south west) coordinate[midway] (2BR);
			\end{tikzpicture}
		\end{figure}
	\end{example}
	
	
	\section{Complex Conjugate} \label{Complex Conjugate Section - Complex}
	\begin{definition}[Complex Conjugate]
		The complex conjugate of $z \in \mathbb{C}$ is denoted $\bar{z}$.
		$$\bar{z} = x - iy = r(\cos(\theta) - i\sin(\theta)) = re^{-i\theta}$$
		\label{Complex Conjugate}
	\end{definition}
	Graphically, it is the reflection of $z$ across the real axis.
	\begin{center}
		\begin{tikzpicture}
			\begin{axis}[
				small,
				xlabel={$\Re$},
				ylabel={$\Im$},
				xmin=0, xmax=10,
				ymin=-10, ymax=10,
				xtick={100},
				ytick={100},
				axis lines = middle,
				]
				\addplot[
				only marks,
				nodes near coords,
				point meta = explicit symbolic, 
				color=Black,
				mark=*,
				]
				coordinates {
					(7, 5) [$z = x + iy$]
					(7, -5) [$\bar{z} = x - iy$]
				};
			\end{axis}
		\end{tikzpicture}
	\end{center}
	It is then easy to see
	\begin{align*}
		\operatorname{Re}(z) &= \frac{z + \bar{z}}{2} & \operatorname{Im}(z) &= \frac{z - \bar{z}}{2i} & \abs{z}^2 = z \bar{z}
	\end{align*}
	As $\operatorname{Re}(z) = x = r \cos(\theta)$ and $\operatorname{Im}(z) = y = r \sin(\theta)$ and using \cref{Exponential Form of z - Complex}, we can obtain the complex forms of sine and cosine: 
	\begin{definition}[Complex Sine and Cosine]
		\begin{align*}
			\cos(\theta) &= \frac{e^{i \theta} + e^{-i \theta}}{2} 
			&\sin(\theta) &= \frac{e^{i \theta} - e^{-i \theta}}{2i}
		\end{align*}
		\label{Trig Identities - Complex}
	\end{definition}


	\section{Operations as Transformations} \label{Operations as Transformations Section - Complex}
	
	Consider any $z \in \mathbb{C}$. A function $f: \mathbb{C} \rightarrow \mathbb{C}$ can be viewed as transformations of the complex plane. 
	
	\begin{example}[Addition as sliding]
		Consider any $z_0 \in \mathbb{C}$, $z_0 = a + ib$ for $a, b \in \mathbb{R}$. Addition by $z_0$ can be seen as a shift in the complex plane by $a + bi$. (i.e. It takes the origin and shifts it by $z_0$.)
		\begin{figure}[H]
			\centering
			\begin{tikzpicture}[remember picture]
				\begin{axis}[
					width=8.9cm,
					unit vector ratio=1 1 1,
					xlabel={$\Re$},
					ylabel={$\Im$},
					xmin=-10, xmax=10,
					ymin=-10, ymax=10,
					axis lines = middle,
					xticklabels={}, yticklabels={},
					grid=both,
					axis lines=middle,
					minor tick num=4,
					minor tick style={draw=none},
					]
					
					\addplot[
					only marks,
					nodes near coords,
					point meta = explicit symbolic, 
					color=Blue,
					mark=*,
					]
					coordinates {
						(0,0)
					};
				\end{axis}
				\path (current bounding box.north east) -- 
				(current bounding box.south east) coordinate[midway] (2BL);
			\end{tikzpicture}
			\hspace*{2.8cm}
			\begin{tikzpicture}[remember picture]
				\begin{axis}[
					%					small,
					width=8.9cm,
					unit vector ratio=1 1 1,
					xlabel={$\Re$},
					ylabel={$\Im$},
					xmin=-10, xmax=10,
					ymin=-10, ymax=10,
					xticklabels={}, yticklabels={},
					grid=both,
					axis lines=middle,
					minor tick num=4,
					minor tick style={draw=none},
					]
					\addplot[
					only marks,
					nodes near coords,
					point meta = explicit symbolic, 
					color=Blue,
					mark=*,
					]
					coordinates {
						(4, 3) [$z_0=a+bi$]
					};
					\draw[->, very thick](axis cs:0,0)--(axis cs:4, 3);
				\end{axis}
				\path (current bounding box.north west) -- 
				(current bounding box.south west) coordinate[midway] (2BR);
			\end{tikzpicture}
			\tikz[overlay,remember picture]{\draw[->, very thick] 
				($(2BL)+(0.5, 0)$) -- ($(2BR)+(-0.5,0)$)
				node[midway,above,text width=2.5cm]{$f(z_0) = z + z_0$};}
		\end{figure}
	\end{example}

	\begin{example}[Multiplication as scaling and rotation]
		Consider any $z_0 \in \mathbb{C}$, $z_0 = re^{i\theta}$. Multiplication by $z_0$ scales the entire complex plane by $r$ and rotates it by $\theta$. (Imagine rotating and stretching out a net.)
		\begin{figure}[H]
			\centering
			\begin{tikzpicture}[remember picture]
				\begin{axis}[
					width=8.9cm,
					unit vector ratio=1 1 1,
					xlabel={$\Re$},
					ylabel={$\Im$},
					xmin=-10, xmax=10,
					ymin=-10, ymax=10,
					axis lines = middle,
					xticklabels={}, yticklabels={},
					grid=both,
					axis lines=middle,
					minor tick num=4,
					minor tick style={draw=none},
					]
					
					\addplot[
					only marks,
					nodes near coords,
					point meta = explicit symbolic, 
					color=Blue,
					mark=*,
					]
					coordinates {
						(1,0) [$1 = e^{0}$]
					};
				\draw[->, very thick](axis cs:0,0)--(axis cs:1, 0);
				\end{axis}
				\path (current bounding box.north east) -- 
				(current bounding box.south east) coordinate[midway] (2BL);
			\end{tikzpicture}
			\hspace*{2.8cm}
			\begin{tikzpicture}[remember picture]
				\begin{axis}[
					width=8.9cm,
					unit vector ratio=1 1 1,
					xlabel={$\Re$},
					ylabel={$\Im$},
					xmin=-10, xmax=10,
					ymin=-10, ymax=10,
					xticklabels={}, yticklabels={},
%					ytick distance={3},
%					xtick distance={4},
%					grid=both,
					axis lines=middle,
					minor tick num=0,
					minor tick style={draw=none},
					]
					\addplot[
					only marks,
					nodes near coords,
					point meta = explicit symbolic, 
					color=Blue,
					mark=*,
					]
					coordinates {
						(4, 3) [$z_0=re^{i\theta}$]
					};
%					\draw[->,thick] (axis cs:0,0)--(axis cs:4,3);
					\draw[->,thick,draw=Blue]  (axis cs: 0,0) to ["$r$"] (axis cs: 4,3);
					% angle
					\draw[->, draw=Blue] (axis cs:0,0)++(0:30) arc (0:35:30)
					node[midway,above right,inner sep=2pt,font={\footnotesize}]{$\theta$};
				\end{axis}
%				\begin{axis}[
%					anchor=center, % Shift the axis so its origin is at (0,0)
%					rotate around={36.87:(current axis.origin)}, % Rotate around the origin
%					width=8.9cm,
%					unit vector ratio=1 1 1,
%					xmin=-10, xmax=10,
%					ymin=-10, ymax=10,
%					xticklabels={}, yticklabels={},
%					grid=both,
%					axis lines=middle,
%					minor tick num=0,
%					minor tick style={draw=none},
%					]
%				\end{axis}
				\path (current bounding box.north west) -- 
				(current bounding box.south west) coordinate[midway] (2BR);
%				\pgftransformrotate{36.87}
%				\pgfpathgrid[stepx=50,stepy=50]{\pgfpoint{0}{-50}}{\pgfpoint{200}{150}}
%				\pgfusepath{stroke}
			\end{tikzpicture}
			\tikz[overlay,remember picture]{\draw[->, very thick] 
			($(2BL)+(0.5, 0)$) -- ($(2BR)+(-0.5,0)$)
			node[midway,above,text width=2.5cm]{$f(z_0) = z \cdot z_0$};}
		\end{figure}
	\end{example}

%	\begin{example}[Logarithm as compression (Speculation)]
%		content...
%	\end{example}
	
	\section{Complex Analysis Definitions} \label{Complex Analysis Definitions Section - Complex}
	
	\begin{definition}[Neighbourhood]
		A neighbourhood of a point $z_0$ is the set of all points $z$ with distance less than $\epsilon$. 
		$$\{z : \abs{z-z_0} < \epsilon \}$$
		i.e. It is the set of all points that lie within a circle centred at $z_0$ with radius $\epsilon$. Points on the circumference not included. 
	\end{definition}
	\begin{figure}[H]
		\centering
		\begin{tikzpicture}[remember picture]
			\begin{axis}[
				small,
				unit vector ratio=1 1 1,
				xlabel={$\Re$},
				ylabel={$\Im$},
				xmin=0, xmax=10,
				ymin=0, ymax=8,
				%					xtick distance=2,
				%					ytick distance=2,
				xtick=100,
				ytick=100,
				axis lines = middle,
				axis line style = Black,
				]
				\addplot[
				only marks,
				nodes near coords,
				point meta = explicit symbolic, 
				color=Black,
				mark=*,
				]
				coordinates {
					(7, 4) [$z_0$]
				};
				\draw[dashed,line width=1pt,LightGrey](axis cs: 7, 4) circle [radius=20];
				\draw[-,draw=Grey]  (axis cs: 7,4) to ["$\epsilon$"] (axis cs: 5,4);;
			\end{axis}
		\end{tikzpicture}
	\end{figure}

	\begin{definition}[Deleted Neighbourhood]
		A deleted neighbourhood is the set of all points $z$ with distance less than $\epsilon$ from a point $z_0$, not including $z_0$. That is, it is a neighbourhood of $z_0$ without $z_0$.
		$$\{z : \abs{z-z_0} < \epsilon, \ z \neq z_0 \}$$
	\end{definition}
	\begin{figure}[H]
		\centering
		\begin{tikzpicture}[remember picture]
			\begin{axis}[
				small,
				unit vector ratio=1 1 1,
				xlabel={$\Re$},
				ylabel={$\Im$},
				xmin=0, xmax=10,
				ymin=0, ymax=8,
				%					xtick distance=2,
				%					ytick distance=2,
				xtick=100,
				ytick=100,
				axis lines = middle,
				axis line style = Black,
				]
				\addplot[
				only marks,
				nodes near coords,
				point meta = explicit symbolic, 
				color=Black,
				mark=o,
				]
				coordinates {
					(7, 4) [$z_0$]
				};
				\draw[dashed,line width=1pt,LightGrey](axis cs: 7, 4) circle [radius=20];
				\draw[-,draw=Grey]  (axis cs: 7,4) to ["$\epsilon$"] (axis cs: 5,4);;
			\end{axis}
		\end{tikzpicture}
	\end{figure}

	\begin{definition}[Interior Point]
		Let $S$ be a set. A point $z_0$ is an interior point of $S$ if $\exists \epsilon$ such that $\forall z$, $\abs{z-z_0} < \epsilon \implies z \in S$. That is, $z_0$ is an interior point of $S$ if it has a neighbourhood where all points in the neighbourhood is an element of $S$.
	\end{definition}
	\begin{figure}[H]
		\centering
		\begin{tikzpicture}[remember picture]
			\begin{axis}[
				small,
				unit vector ratio=1 1 1,
				xlabel={$\Re$},
				ylabel={$\Im$},
				xmin=0, xmax=10,
				ymin=0, ymax=8,
				%					xtick distance=2,
				%					ytick distance=2,
				xtick=100,
				ytick=100,
				axis lines = middle,
				axis line style = Black,
				]
				\addplot[
				only marks,
				nodes near coords,
				point meta = explicit symbolic, 
				color=Black,
				mark=*,
				]
				coordinates {
					(7, 4) [$z_0$]
				};
				\draw[dashed,line width=1pt,LightGrey](axis cs: 7, 4) circle [radius=10];
				\draw[line width=1pt, Black](axis cs: 5.2, 3.5) circle [radius=30];
				\draw[-,draw=Grey]  (axis cs: 7,4) to ["$\epsilon$"] (axis cs: 6,4);
				\draw[]  (axis cs: 5,2) to ["$S$"] (axis cs: 5,2);
			\end{axis}
		\end{tikzpicture}
	\end{figure}

	\begin{definition}[Exterior Point]
		Let $S$ be a set. A point $z_0$ is an exterior point of $S$ if $\exists \epsilon$ such that $\forall z$, $\abs{z - z_0}<\epsilon \implies z \notin S$. That is, $z_0$is an exterior point of $S$ if it has neighbourhood that does not contain any element of $S$. 
	\end{definition}
	\begin{figure}[H]
		\centering
		\begin{tikzpicture}[remember picture]
			\begin{axis}[
				small,
				unit vector ratio=1 1 1,
				xlabel={$\Re$},
				ylabel={$\Im$},
				xmin=0, xmax=10,
				ymin=0, ymax=8,
				%					xtick distance=2,
				%					ytick distance=2,
				xtick=100,
				ytick=100,
				axis lines = middle,
				axis line style = Black,
				]
				\addplot[
				only marks,
				nodes near coords,
				point meta = explicit symbolic, 
				color=Black,
				mark=*,
				]
				coordinates {
					(9, 5) [$z_0$]
				};
				\draw[dashed,line width=1pt,LightGrey](axis cs: 9, 5) circle [radius=10];
				\draw[line width=1pt, Black](axis cs: 5.2, 3.5) circle [radius=30];
				\draw[-,draw=Grey]  (axis cs: 9,5) to ["$\epsilon$"] (axis cs: 8,5);
				\draw[]  (axis cs: 5,2) to ["$S$"] (axis cs: 5,2);
			\end{axis}
		\end{tikzpicture}
	\end{figure}

	\begin{definition}[Boundary Point]
		Let $S$ be a set. A point $z_0$ is a boundary point of $S$ if $\forall \epsilon$, $\exists z \in S, z' \notin S$, such that $\abs{z - z_0} \epsilon$ and $\abs{z' - z_0} < \epsilon$. That is, for all neighbourhoods of $z_0$ there exists a point that is in $S$ and a point not in $S$.
	\end{definition}
	\begin{figure}[H]
		\centering
		\begin{tikzpicture}[remember picture]
			\begin{axis}[
				small,
				unit vector ratio=1 1 1,
				xlabel={$\Re$},
				ylabel={$\Im$},
				xmin=0, xmax=10,
				ymin=0, ymax=8,
				%					xtick distance=2,
				%					ytick distance=2,
				xtick=100,
				ytick=100,
				axis lines = middle,
				axis line style = Black,
				]
				\addplot[
				only marks,
				nodes near coords,
				point meta = explicit symbolic, 
				color=Black,
				mark=*,
				]
				coordinates {
					(5.2, 6.5) [$z_0$]
				};
				\draw[dashed,line width=1pt,LightGrey](axis cs: 5.2, 6.5) circle [radius=10];
				\draw[line width=1pt, Black](axis cs: 5.2, 3.5) circle [radius=30];
				\draw[-,draw=Grey]  (axis cs: 5.2,6.5) to ["$\epsilon$"] (axis cs: 4.2,6.5);
				\draw[]  (axis cs: 5,2) to ["$S$"] (axis cs: 5,2);
			\end{axis}
		\end{tikzpicture}
	\end{figure}

	Note: A boundary point of $S$ may or may not be in $S$.

	\begin{definition}[Boundary of a Set]
		A boundary of a set $S$ is the set of all boundary points of $S$. The set containing all boundary points of $S$.
		$$\{z_0 : \forall \epsilon \exists z\in S, z'\notin S(\abs{z-z_0}<\epsilon \land \abs{z'-z_0}<\epsilon )\}$$ 
	\end{definition}

	\begin{definition}[Open Set]
		A set that does not contain any boundary points. 
	\end{definition}

	\begin{theorem}
		Set $S$ is open $\iff$ $\forall s \in S$, $s$ is an interior point of $S$ 
	\end{theorem}
	\begin{proof}
		\textcolor{Grey}{
		\underline{$\implies$}:
		Suppose $S$ is open $\nRightarrow$ $\forall s \in S$, $s$ is an interior point of $S$, for contradiction. That is, $\exists s \in S$ that is either a boundary point or an exterior point. $s \in S$ implies $s$ is not an exterior point of $S$, so $s$ has to be a boundary point of $S$. This contradicts that $S$ is an open set. 
		$$S \text{ is open } \implies \forall s \in S (s \text{ is an interioir point of }S)$$
		\underline{$\impliedby$:}
		\begin{align*}
			&\forall s \in S (s \text{ is an interior point of S}) \\
			&\implies \forall s' \forall \epsilon (\abs{s'-s} < \epsilon \implies s' \in S)\\
			&\implies S \text{ does not contain boundary points} \implies S \text{ is open}
		\end{align*}
		}
	\end{proof}

	A set can be neither open or closed. Consider the set $S = \{z : 0 < \abs{z} \leq 1\}$. $S$ is not closed since it does not contain the boundary point $0$, and it is not open since it contains boundary points where $\abs{z} = 1$. The set $\mathbb{C}$ is both open and closed since it has no boundary points. 
	
	\begin{definition}[Closed Set]
		A set that contains all of its boundary points. 
	\end{definition}

	\begin{definition}[Closure of a Set]
		Let $S$ be a set. The closure of S is a closed set containing all points of $S$ and all boundary points of $S$. 
	\end{definition}

	\begin{definition}[Connected Set]
		An opens set $S$ is connected if $\forall z_1, z_2 \in S$, $z_1$ and $z_2$ can be connected by a polygonal line lying within $S$.
	\end{definition}
	\begin{figure}[H]
		\centering
		\begin{tikzpicture}[remember picture]
			\begin{axis}[
				small,
				unit vector ratio=1 1 1,
				xlabel={$\Re$},
				ylabel={$\Im$},
				xmin=0, xmax=10,
				ymin=0, ymax=8,
				%					xtick distance=2,
				%					ytick distance=2,
				xtick=100,
				ytick=100,
				axis lines = middle,
				axis line style = Black,
				]
				\addplot[
				only marks,
				nodes near coords,
				point meta = explicit symbolic, 
				color=Black,
				mark=*,
				]
				coordinates {
					(3, 3) [$z_1$]
					(6, 6) [$z_2$]
				};
				\draw[line width=1pt, Black](axis cs: 5, 4) circle [radius=10];
				\draw[line width=1pt, Black](axis cs: 5, 4) circle [radius=35];
				\draw[-,draw=Grey]  (axis cs: 3,3) -- (axis cs: 4,5);
				\draw[-,draw=Grey]  (axis cs: 4,5) -- (axis cs: 6,6);
				\draw[]  (axis cs: 5,2) to ["$S$"] (axis cs: 5,2);
44			\end{axis}
		\end{tikzpicture}
	\end{figure}

	\begin{definition}[Polygonal Line]
		A finite set of line segments joined end to end. 
	\end{definition}

	\begin{definition}[Domain]
		A nonempty connected set. 
	\end{definition}
	Note: All neighbourhoods are domains. 

	\begin{definition}[Region]
		A domain with none, some, or all of its boundary points. 
	\end{definition}
	
	\begin{definition}[Bounded Set]
		A set S is bounded if $\exists R$ such that $\forall s \in S$, $s < R$.
	\end{definition}

	\begin{definition}[Accumulation/Limit Point]
		A point $z_0$ is a accumulation point of a set $S$ if all deleted neighbourhood of $z_0$ contains an element of $S$. 
		$$\forall \epsilon \exists s \in S (s \neq z_0 \land \abs{z-s} < \epsilon)$$
	\end{definition}
	Note: Unlike a boundary point, an accumulation point does not require that all neighbourhood of $z_0$ contain an element not in $S$.
	
	\begin{theorem}
		Set $S$ is closed $\iff$ $\forall$ accumulation points $z_0$ of $S$, $z_0 \in S$
	\end{theorem}
	\begin{proof}
		\underline{$\implies$:}
		Let $S$ is closed and $z_0$ is an accumulation point of a set $S$ where $z_0 \notin S$ for contradiction. If $\exists z_0 \notin S$, then $z_0$ is a boundary point of $S$. Contradicts closed set contains all boundary points. 
		
		\textcolor{Grey}{
		\underline{$\impliedby$:}
		Suppose all accumulation points of $S$ are elements of $S$ but $S$ is not closed for contradiction. Then $S$ does not contain one or more boundary points. Suppose $z_0$ is a boundary point of $S$ that is not in $S$. Then $\forall \epsilon \ \exists s\in S$ where $\abs{s-z_0} < \epsilon$, so by considering the deleted neighbourhood of $z_0$, this makes $z_0$ an accumulation point of $S$. This contradicts that all accumulation points of $S$ is in $S$. 		
		}
	\end{proof}

	\chapter{Analytic Functions} \label{Analytic Functions Chapter - Complex}
	
	
	
	\chapter{Conformal Mapping} \label{Conformal Mapping Chapter - Complex}
	
	
	\part{Ordinary Differential Equations} \label{Ordinary Differential Equations Part}
	
	\part{Nonlinear Dynamics} \label{Nonlinear Dynamics Part}
	
	
	\part{Partial Differential Equations} \label{Partial Differential Equations Part}
	
	\paragraph{Calculus of Variations} \label{Calculus of Variations Part}
	
	\part{Integral Equations} \label{Integral Equations Part}
	
	
	\part{Linear Algebra} \label{Linear Algebra Part}
	
	\chapter{Markov Chains} \label{Markov Chains Chapter - Linear Algebra}
	
	
	\part{Tensors} \label{Tensors Part}
	
	
	\part{Riemann Geometry} \label{Reimann Geometry Part}
	
	
	\part{Abstract Algebra} \label{Abstract Algebra Part}
	
	\chapter{Groups} \label{Groups Chapter - Abstract Algebra}
	
	
	\chapter{Rings} \label{Rings Chapter - Abstract Algebra}
	
	\section{Ideals} \label{Ideals Section - Abstract Algebra}
	
	\chapter{Integral Domains} \label{Integral Domains Chapter - Abstract Algebra}
	
	\chapter{GCD Domains} \label{GCD Domains Chapter - Abstract Algebra}
	
	\chapter{Unique Factorization Domains} \label{Unique Factorization Domains Chapter - Abstract Algebra}
	
	\chapter{Principal Ideal Domains} \label{Principal Ideal Domains Chapter - Abstract Algebra}
	
	\chapter{Fields} \label{Fields Chapter - Abstract Algebra}
	
	
	\part{Galois Theory} \label{Galois Theory Part}
	
	\part{Lie Theory} \label{Lie Algebra Part}
	
	\chapter{Lie Groups}
	
	\chapter{Lie Algebra}
	
	\part{C-Star Algebra} \label{C-Star Algebra Part}
	
	\part{Set Theory} \label{Set Theory Part}
	
	\part{Model Theory} \label{Model Theory Part}
	
	\part{Statistics} \label{Statistics Part}
	\part{Tips and Tricks} \label{Tips and Tricks Part}
	
	\chapter{Integration Techniques} \label{Integration Techniques Chapter - Tips and Tricks}
	
	\section{DI Method (Integration Table)} \label{DI Method Section - Tips and Tricks}
	
	\section{Feynman Integration} \label{Feynman Integration Section - Tips and Tricks}
	
	\backmatter
	\part{Index} \label{Index Part}
	
	\part{Bibliography}
	\bibliographystyle{unsrt}
	\typeout{}
	\bibliography{Bibliography}
	

\end{document}
