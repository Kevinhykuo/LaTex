\documentclass[12pt, english]{book}
\usepackage[letterpaper, portrait, margin=2.5cm]{geometry} %Set paper type, format, margin
\usepackage[svgnames]{xcolor}
\usepackage{%abstract, 
	amsmath, amssymb, array, babel, booktabs, caption, censor, fancyhdr, flafter, float, framed, gensymb, hyperref, inputenc, indentfirst, mathtools, MnSymbol, multicol, multirow, longtable, lscape, ltablex, polynom, pgfplots, placeins, rotating, scalerel, setspace, stackengine, tabularx, threeparttable, titlesec, titling, wasysym, wrapfig}
\usepackage{cleveref}
\usetikzlibrary{calc,angles,positioning,intersections,quotes,decorations.markings,backgrounds,patterns}
\usepackage{amsthm} %Do not use amsthm with ntheorem unlees ntheorem is using the amsthm option
\usepackage{thmtools}
%\usepackage[standard]{ntheorem}

%\renewcommand{\abstractname}{}    % clear the title
%\renewcommand{\absnamepos}{empty} % originally center 
\setlength{\parindent}{0em} %Set indent on paragraph
\setlength{\parskip}{0.5em} %Set spaces between paragraphs

\DeclarePairedDelimiter\abs{\lvert}{\rvert}%
\DeclarePairedDelimiter\norm{\lVert}{\rVert}%

\newcommand\showdiv[1]{\overline{\smash{\hstretch{.5}{)}\mkern-3.2mu\hstretch{.5}{)}}#1}}
\newcommand\ph[1]{\textcolor{white}{#1}}
\setstackgap{S}{1.5pt}

% Swap the definition of \abs* and \norm*, so that \abs
% and \norm resizes the size of the brackets, and the 
% starred version does not.

% Variables for costomizing theorem enviroment
% \newtheoremstyle{style name}{space above}{space below}{body font}{indent amount}{head font}{head punct}{after head space}{head spec}

%\newtheoremstyle{break}
%	{\topsep}{\topsep}%
%	{\itshape}{}%
%	{\bfseries}{}%
%	{\newline}{}%

%\theoremstyle{break}
%\newtheorem{axiom}{Axiom}[chapter]
%\newtheorem{thm}{Theorem}[section]
%\newtheorem{cor}{Corollary}[thm]
%\newtheorem{lem}[thm]{Lemma}
%\newtheorem{conj}[thm]{Conjecture}
%\newtheorem{definition}{Definition}[section]
%
%\theoremstyle{plain}
%\newtheorem*{rmk}{Remark}
%\newtheorem*{obs}{Observation}
%\newtheorem*{qtn}{Question}
%\newtheorem*{example}{Example}

%\colorlet{headcolour}{DarkOrange}
%\colorlet{rulecolour}{Orange}
%\renewenvironment{leftbar}{%
%	\def\FrameCommand{{\color{rulecolour}\vrule width 3pt} \hspace{10pt}}%
%	\MakeFramed {\advance\hsize-\width \FrameRestore}}%
%{\endMakeFramed}

\declaretheoremstyle[
	spaceabove=6pt, 		
	spacebelow=6pt,
	%	headfont=\normalfont\bfseries,
	notefont=\bfseries, notebraces={}{},
	bodyfont=\slshape,
	postheadspace=1em,
	headpunct=,
	headformat={\NAME~\NUMBER:\NOTE \hfill \smallskip\linebreak},%
	]{axiomstyle}

\declaretheoremstyle[
	spaceabove=6pt, 		
	spacebelow=6pt,
	headfont=\bfseries, %\color{blue},
	notefont=\bfseries, notebraces={}{},
	bodyfont=\slshape \color{DarkBlue},
	postheadspace=1em,
	headpunct=,
	headformat={\NAME~\NUMBER:\NOTE \hfill \smallskip\linebreak},%
	]{theoremstyle}
	
\declaretheoremstyle[
	spaceabove=6pt, 		
	spacebelow=6pt,
	headfont=\bfseries, %\color{Blue}%
	notefont=\bfseries,%
	notebraces={}{},%
	headpunct=,%
	bodyfont=\itshape\color{DarkGreen},%
	headformat={\NAME~\NUMBER:\NOTE \hfill \smallskip\linebreak},%
	%	preheadhook=\begin{leftbar},%
	%	postfoothook=\end{leftbar},%
	]{minorstyle}
	
\declaretheoremstyle[
	spaceabove=6pt, 		
	spacebelow=6pt,
	headfont=\bfseries,%\color{Blue},%
	notefont=\bfseries,%
	notebraces={}{},%
	headpunct=,%
	bodyfont=\slshape \color{OrangeRed},%
	headformat={\NAME~\NUMBER:\NOTE \hfill \smallskip\linebreak},%
%	preheadhook=\begin{leftbar},%
%	postfoothook=\end{leftbar},%
	]{defstyle}
	

\declaretheorem{axiom}[
	numberwithin=chapter, style=axiomstyle, parent=chapter,
	]
	
\declaretheorem{theorem, conjecture}[
	numberwithin=section, style=theoremstyle, parent=section,
	]
	
\declaretheorem{lemma, proposition, corollary}[
	style=minorstyle,
	parent=theorem,
%	numberwithin=section
	]

\declaretheorem{definition}[parent=section, style=defstyle]

\declaretheorem{remark, observation, question}[numbered=no]

\declaretheorem{example}[numberwithin=section]

% Indenting the Proof enviroment
\makeatletter
\renewenvironment{proof}[1][\proofname]{\par
	\pushQED{\qed}%
	\normalfont \topsep6\p@\@plus6\p@\relax
	\list{}{%
		\settowidth{\leftmargin}{\itshape\proofname:\hskip\labelsep}%
		\setlength{\labelwidth}{0pt}%
		\setlength{\itemindent}{-\leftmargin}%
	}%
	\item[\hskip\labelsep\itshape#1\@addpunct{:}]\ignorespaces
}{%
	\popQED\endlist\@endpefalse
}
\makeatother

% Letters after part
\makeatletter
\renewcommand{\@endpart}{\vfil\newpage}
\makeatother
\newenvironment{partintro}
{\vspace*{\fill}
	\section*{\centering Resources used in part \thepart}
	\begin{quotation}}
	{\end{quotation}\vspace*{\fill}\newpage}
\newcommand{\nopartintro}{%
	\vspace*{\fill}
	\thispagestyle{empty}
	\newpage
}


\begin{document}
	\title{The Book of Math (Notes)}
	\author{Kevin Kuo}
%	\renewcommand{\abstractname}{\vspace{-\baselineskip}}
	
	\pagestyle{fancy}
	\fancyhead{} % clear all header fields
	\fancyhead[LO]{  }
	\fancyhead[CO]{  }
	\fancyhead[RO]{  }
	\renewcommand{\headrulewidth}{0pt}
	
	\fancyfoot{} % clear all footer fields
	\fancyfoot[LO]{}
	\fancyfoot[CO]{\thepage}
	\fancyfoot[RO]{}
	\renewcommand{\footrulewidth}{0pt}
	\vspace{0cm}	
	
	\frontmatter
	
	\maketitle
	
	\newpage
	\section*{Forward and Disclaimer}
	These are math notes made by a student (with a physics major and math minor) based off text books. It may contain misconceptions and misinterpretations, thus should not be viewed in the same light of a text book. Use at your own risk and mental sanity.
	
	\section*{Symbols}
	\begin{tabularx}{\textwidth}{ l c p{0.6\linewidth}}
		\multicolumn{3}{l}{\textbf{{\large Logic}}} \\ [10pt]
		\hline
		\textbf{Name} & \textbf{Symbol} & \textbf{Comment} \\
		\hline
		Exists 					& $\exists$ 		& There exists at least one\\
		For all 				& $\forall$ 		& \\
		Not exists 				& $\nexists$ 		& There does not exist\\ 
		Exists one				& $\exists!$ 		& There only exists one and only one \\
		And 					& $\land$			& \\
		Or						& $\lor$			& Inclusive or \\
		Not 					& $\neg$			& \\
		Logically implies 		& $\implies$ 		& If \\
		Logically implied by 	& $\Longleftarrow$ 	& Only if \\  
		Logically equivalent 	& $\iff$ 			& If and only if \\
		Implies 				& $\longrightarrow$	& \\
		Implied by 				& $\longleftarrow$ 	& \\  
		Double Implication 		& $\longleftrightarrow$	& \\
		\hline	
		
		& & \\
		\multicolumn{3}{l}{\textbf{{\large Set Notation}}} \\ [10pt]
		\hline
		\textbf{Name} & \textbf{Symbol} & \textbf{Comment} \\
		\hline
 		Empty Set 				& $\emptyset$ 		& The set that is empty \\
 		Natural Numbers 		& $\mathbb{N}$		& Set of natural numbers not containing 0, equivalent to the set of positive integers \\
 		Integers 				& $\mathbb{Z}$		& Set of integers \\
 		Rational Numbers 		& $\mathbb{Q}$		& \\
 		Algebraic Numbers		& $\mathbb{A}$		& \\
 		Real Numbers 			& $\mathbb{R}$		& \\
 		Complex Numbers 		& $\mathbb{C}$		& \\
 		
 		In 						& $\in$ 			& \\
 		Not in 					& $\nin$			& \\
 		Owns 					& $\ni$				& Has an element \\
 		
 		Proper Subset 			& $\subset$			& Subset that is not itself \\
 		Subset 					& $\subseteq$		& \\
 		Superset 				& $\supset$ 		& Superset that is not itself\\
 		Proper Superset 		& $\supseteq$		& \\
 		Power set				& $\wp$				& \\
 		Union 					& $\cup$			& \\
 		Intersection			& $\cap$			& \\
 		Difference				& $\setminus$		& \\
 		\hline
 		
 		& & \\
 		\multicolumn{3}{l}{\textbf{{\large Relationships}}} \\ [10pt]
 		\hline
 		\textbf{Name} & \textbf{Symbol} & \textbf{Comment} \\
 		\hline
 		Defined 				& $\doteq$ 			& \\
 		Approximate 			& $\approx$			& \\
 		Equivalent				& $\equiv$	 		& Isomorphic (Group Theory) \\
 		Congruent 				& $\cong$			& Homomorphic (Group Theory) \\
 		Proportional 			& $\propto$			& \\
 		\hline
 		
 		& & \\
 		\multicolumn{3}{l}{\textbf{{\large Operators}}} \\ [10pt]
 		\hline
 		\textbf{Name} & \textbf{Symbol} & \textbf{Comment} \\
 		\hline
 		& $\oplus$ & \\
 		& $\otimes$ & \\
 		& $\odot$ & \\
 		& $\circ$ & Convolution \\
 		Dagger& $\dagger$ & Complex conjugate transpose of a matrix \\
 		\hline
 		
 		& & \\
 		\multicolumn{3}{l}{\textbf{{\large Arrows}}} \\ [10pt]
 		\hline
 		\textbf{Name} & \textbf{Symbol} & \textbf{Comment} \\
 		\hline
 		Maps to 				& $\mapsto$			& \\
 		\hline
 		
 		& & \\
 		\multicolumn{3}{l}{\textbf{{\large Hebrew}}} \\ [10pt]
 		\hline
 		\textbf{Name} & \textbf{Symbol} & \textbf{Comment} \\
 		\hline
 		Aleph					& $\aleph$			& Carnality of infinite sets that can be well ordered \\
 		\hline
 		
 		& & \\
 		\multicolumn{3}{l}{\textbf{{\large Other}}} \\ [10pt]
 		\hline
 		\textbf{Name} & \textbf{Symbol} & \textbf{Comment} \\
 		\hline
 		Real part 				& $\Re$				& Real part of a number \\
 		Imaginary part 			& $\Im$				& Imaginary part of a number \\
 		\hline
	\end{tabularx}

	\newpage
	\section*{Book Constitution}
	\subsection*{Intents and Purpose}
	The goal of this book is to organize mathematical knowledge of topics related to the study of physics or the author's interest. It is meant to be used as a source of for future reference, not as a textbook for students new to the topics. It is a notebook of a student, thus should be treated as one and not as a textbook. At most, it could be used as a study guide along side a textbook. Definitely not as the main source for acquiring knowledge. 
	
	\subsection*{Layout and Organization}
	The book is split into parts each containing a field of study mathematics, or a topic large enough to justify giving it its own part. Each part contains chapters that focuses on a particular topic required to understand the field, with sections dedicated to describing a particular knowledge required for the topic. 
	
	As axioms, definitions, theorems, corollary, and proofs are integral and abundant to the study of mathematics, each will have a unique style. Each environment and its styles are displayed as follows: 
	
	\begin{axiom}[Axiom name]{Example Axiom}
		Axioms are the ``ground rules" of the set.
	\end{axiom}

	\begin{theorem}[Theorem name or citation]{Example Theorem}
		An important logical result from the axioms, with proof.
	\end{theorem}

	\begin{conjecture}[Name of conjecture or citation]{Example Conjecture}
		A hypothesis, without proof.
	\end{conjecture}
	
	\begin{corollary}{Example Corollary}
		An implication as a result of a theorem.
	\end{corollary}

	\begin{lemma}{Example Lemma}
		Small theorems that build up to a larger theorem. 
	\end{lemma}
	
	\begin{proposition}{Example Proposition}
		Example proposition.
	\end{proposition}
	
	\begin{proof}
		Logical deductions that results in a theorem.
		\textcolor{Grey}{Proofs I've written will be in grey, which may or may not be correct.}
	\end{proof}

%	\theoremstyle{break}
	\begin{definition}[Word]{Example Definition}
		The definition of a word.
	\end{definition}	

	\begin{example}
		An example.
	\end{example}

	\begin{remark}{Remark}
		A comment by the author in the textbooks used.
	\end{remark}
	
	\begin{observation}{Example Observation}
		A remark by me.
	\end{observation}

	\begin{question}{Example Question}
		A question from me for a mystery to be answered later. 
	\end{question}

	
	\section*{}
	\tableofcontents
	
	\mainmatter
	\part{Logic}
	
	\chapter{Proofs}
	
	
	\part{Numbers}
	\begin{partintro}
		content...
	\end{partintro}
	
	\chapter{Natural $\mathbb{N}$}
	
	\chapter{Integers $\mathbb{Z}$}
	
	\chapter{Rationals $\mathbb{Q}$}
	
	\chapter{Constructible}
	
	\chapter{Algebraic $\mathbb{A}$}
	
	\chapter{Reals $\mathbb{R}$}
	
	\chapter{Complex $\mathbb{C}$}
	

	
	\part{Real Analysis}
	\begin{partintro}
		\begin{itemize}
			\item[1.] Kenneth A. Ross - Elementary Analysis (2nd Ed.) \cite{Ross.K-Elementary-Analysis-2013}
		\end{itemize}
	\end{partintro}

	\chapter{Metric Spaces}	
	
	\part{Complex Analysis}
	\begin{partintro}
		\begin{itemize}
			\item[1.] Brown and Churchill - Complex Variables and Applications \cite{Brown.J;Churchill.R-Complex-Variables-2014}
		\end{itemize}
	\end{partintro}
	
	\chapter{Basics}
	\section{Complex Numbers}
	$$\mathbb{C} = \{x + iy \mid x, y \in \mathbb{R}, i = \sqrt{-1}\}$$
	Complex numbers are elements of the complex field $(\mathbb{C})$, therefore, they obey all the properties of a field. 
	
	We will denote complex numbers by $z = x + iy$ with $x, y \in \mathbb{R}$, and refer the real part as $\Re(z) = \operatorname{Re}(z) = x$ and imaginary part as $\Im(z) = \operatorname{Im}(z) = y$. Complex numbers can also be defined as an ordered pair $z = (x, y)$ which is interpreted as points in the complex plane. $(x, 0)$ are points on the real axis while $(0 , y)$ are points in the imaginary axis. 
	
	\begin{center}
		\begin{tikzpicture}
		\begin{axis}[
			small,
			%			title={},
			xlabel={$\Re$},
			ylabel={$\Im$},
			xmin=0, xmax=10,
			ymin=0, ymax=10,
			xtick={100},
			ytick={100},
			axis lines = left,
%			legend pos=north west,
%			ymajorgrids=true,
%			grid style=dashed,
			]
			
			\addplot[
			only marks,
			nodes near coords,
			point meta = explicit symbolic, 
			color=Black,
			mark=*,
			]
			coordinates {
				(7, 5) [$z = x + iy$]
			};
%			\legend{$z = x + iy$}
		\end{axis}
	\end{tikzpicture}
	\end{center}
	We add and multiply complex numbers in the usual way: 
	\begin{align*}
		z_1 + z_2 &= (x_1 + iy_1) + (x_2 + iy_2) & z_1 z_2 &= (x_1 + iy_1) (x_2 + iy_2) \\
			&= (x_1 + x_2) + i(y_1 + y_2) & &=(x_1 x_2 - y_1 y_2) + i(x_1 y_2 + x_2 y_1)
	\end{align*}
	
	$\forall z \in \mathbb{C}$, there is an unique additive inverse $(-z)$ and $\forall z \in \mathbb{C}\setminus\{0\}$, there is an unique multiplicative inverse $(z^{-1})$ such that 
	\begin{align*}
		&z + (-z) = 0  & &zz^{-1} = 1 \\
		&\implies -z = -x - iy & &\implies (x_1 x_2 - y_1 y_2) = 1 \land (x_1 y_2 + x_2 y_1) = 0 \\
		& & &\implies z^{-1} = \frac{x_1}{x_1^2 + y_1^2} - i \frac{y_1}{x_1^2 + y_1^2}
	\end{align*}
	The existence and uniqueness of the inverses can be easily proven. 
	
	The addition of complex numbers may also be interpreted as akin to vector addition. 
	\begin{center}
		\begin{tikzpicture}
			\begin{axis}[
				small,
				%			title={},
				xlabel={$\Re$},
				ylabel={$\Im$},
				xmin=0, xmax=10,
				ymin=0, ymax=10,
				xtick={100},
				ytick={100},
				axis lines = left,
				%			legend pos=north west,
				%			ymajorgrids=true,
				%			grid style=dashed,
				]
				
				\addplot[
				only marks,
				nodes near coords,
				point meta = explicit symbolic, 
				color=Black,
				mark=*,
				]
				coordinates {
					(7, 5) [$z_1$]
					(1, 4) [$z_2$]
					(8, 9) [$z_1 + z_2$]
				};
				%			\legend{$z = x + iy$}
				\draw[line width=1pt,Grey,-stealth](0,0)--(70,50);
				\draw[line width=1pt,Grey,-stealth](0,0)--(10,40);
				\draw[line width=1pt,Grey,-stealth](70,50)--(80,90);
				\draw[line width=1pt,black,-stealth](0,0)--(80,90);
			\end{axis}
	\end{tikzpicture}
	\end{center}
	Likewise, this naturally extends to the definition of a modulus of a complex number. 
	\begin{definition}[Modulus]
		The absolute value of a real number:
		$\abs{z} = \sqrt{x^2 + y^2}$
	\end{definition}
	It is obvious why the definition is not $\abs{z} = \sqrt{x^2 + (iy)^2}$ as problems arise when $x = y$. The modulus is the distance of $z$ from $(0, 0)$.
	
	\section{Triangle Inequality}
	It is not analysis without a section dedicated to the triangle inequality.
	
	\begin{theorem}[Triangle Inequality]
		$\forall z_1, z_2 \in \mathbb{C} [\abs{z_1 + z_2} \leq \abs{z_1} + \abs{z_2}]$
		\label{Triange Inequality - Complex}
	\end{theorem}
	
	From the theorem, we can derive a similar inequality: 
	\begin{align*}
		\abs{z_1} = \abs{z_1 + z_2 - z_2} \leq \abs{z_1 + z_2} + \abs{-z_2}
		 &\implies \abs{z_1} - \abs{z_2} \leq \abs{z_1 + z_2}
	\end{align*}
	
	An important property of polynomials is observed when \cref{Triange Inequality - Complex} is applied to polynomials.
	
	\begin{corollary}
		Consider the polynomial $P(z)$ where $a_n \in \mathbb{C}$, $n \in \mathbb{N}$, $a_0 \neq 0$, and $z \in \mathbb{C}$.
		$$P(z) = a_0 + a_1 z + a_2 z^2 + \ldots + a_n z^n$$
		Then $\forall z, \exists R \in \mathbb{R}_{>0}, \abs{z} < R$ such that
		$$\abs*{\frac{1}{P(z)}} < \frac{2}{\abs{a_n} R^n}$$
		\label{Complex Poly Reciprocal Bounded}
	\end{corollary}
	\begin{proof}
		Consider 
		\begin{align*}
			w &= \frac{P(z)}{z_n} - a_n = \frac{a_0}{z^n} + \frac{a_1}{z^{n-1}} + \ldots + \frac{a_{n-1}}{z} 
				& z \neq 0 \\
			&\implies wz^n = a_0 + a_1 z + \ldots + a_{n-1}z^{n-1} \\
			&\implies \abs{w}\abs{z}^n \leq \abs{a_0} + \abs{a_1}\abs{z} + \ldots + \abs{a_{n-1}}\abs{z}^{n-1} \\
			&\implies \abs{w} \leq \frac{\abs{a_0}}{\abs{z}^n} + \frac{\abs{a_1}}{\abs{z}^{n-1}} + \ldots + \frac{\abs{a_{n-1}}}{\abs{z}} 
				& \\
			&\implies \abs{w} < n\frac{\abs{a_n}}{2n} = \frac{\abs{a_n}}{2} 
				& \exists \text{ sufficiently large } R < \abs{z}  \text{ s.t.}\\
			& 	& \forall m, \ 0 \leq m \leq n-1, \ \frac{\abs{a_m}}{\abs{z}^{n-m}} < \frac{\abs{a_n}}{2n} \\
			&\implies \abs{a_n + w} \geq \abs{\abs{a_n} - \abs{w}} > \frac{\abs{a_n}}{2}
				& R < \abs{z} \\
			&\implies \abs{P_n(z)} = \abs{a_n + w} \abs{z}^n > \frac{\abs{a_n}}{2}\abs{z}^n > \frac{\abs{a_n}}{2} R^n 
				& R < \abs{z} \\
			&\implies \abs*{\frac{1}{P(z)}} < \frac{2}{\abs{a_n} R^n}
		\end{align*}
	\end{proof}
	This tells us that if $z$ is a solution to a polynomial $P(z)$, then the reciprocal of the polynomial $1/P(z)$ is bounded above by $R = \abs{z}$. (i.e. It is bounded by a circle of radius $\abs{z}$.)
	
	\section{Polar and Exponential Form}
	\begin{definition}[Argument of $z$]
		Consider any $z \in \mathbb{C}$ where $z \neq 0$.
		Let $\theta$ be the angle in radians between $z$ and the real axis .
		Then $\forall n \in \mathbb{N}$, $0 \leq \theta < 2\pi$, the argument of $z$:
		$$ \operatorname{arg}(z) = \theta + 2 n \pi$$
		\label{Argument - Complex}
	\end{definition}
	
	We know $\forall n \in \mathbb{N}$, $\theta + 2 \pi n = \theta$. This leads us to the definition of the principal argument of $z$.
	\begin{definition}[Principal Argument of $z$]
		Consider any $z \in \mathbb{C}$ where $z \neq 0$.
		Let $\theta$ be the angle in radians between $z$ and the real axis.
		Then for $0 \leq \theta < 2\pi$, the principal argument of $z$:
		$$\operatorname{Arg}(z) = \theta$$
		\label{Principal Argument - Complex}
	\end{definition}
	It is clear that $\operatorname{arg}(z) = \operatorname{Arg}(z) + 2n \pi$.
	\begin{definition}[Polar Form of $z$]
		Consider $z \in \mathbb{C}$. Let $r = \abs{z}$, and $\theta = \operatorname{arg}(z)$. 
		Then $\forall z \in \mathbb{C}, z \neq 0$:
		$$z = x + iy = r(\cos(\theta) + i \sin(\theta))$$
		\label{Polar Form of z - Complex}
	\end{definition}

	Notice that all three definitions require that $z \neq 0$ as $\theta$ is undefined at $z = 0$.
	
	\begin{theorem}[Euler's Formula]
		$$e^{i \theta} = \cos(\theta) + i \sin(\theta)$$
		\label{Euler's Formula - Complex}
	\end{theorem}
	Combining \cref{Polar Form of z - Complex} with \cref{Euler's Formula - Complex}, we obtain the Exponential Form of $z$: 
	
	\begin{definition}[Exponential Form of $z$]
	Consider any $z \in \mathbb{C}$, and let $r = \abs{z}$ and $\theta = \operatorname{arg}(z)$. Then the exponential form of $z$:
		$$z = r e^{i \theta}$$
		\label{Exponential Form of z}
	\end{definition}

	\begin{center}
		\begin{tikzpicture}
			\begin{axis}[
				small,
				%			title={},
				xlabel={$\Re$},
				ylabel={$\Im$},
				xmin=0, xmax=10,
				ymin=0, ymax=10,
				xtick={100},
				ytick={100},
				axis lines = left,
				%			legend pos=north west,
				%			ymajorgrids=true,
				%			grid style=dashed,
				]
				
				\addplot[
				only marks,
				nodes near coords,
				point meta = explicit symbolic, 
				color=Black,
				mark=*,
				]
				coordinates {
					(7, 5) [$z = x + iy = re^{i\theta}$]
				};
				%			\legend{$z = x + iy$}
				\draw[thick]  (0,0) to ["$r=\abs{z}$"] (70,50);
				% angle
				\draw[draw=blue] (0,0) ++(35:50) arc (35:0:50)
				node[midway,above right,inner sep=2pt,font={\footnotesize}]{$\theta$};
			\end{axis}
		\end{tikzpicture}
	\end{center}
	
	
	\subsection{Properties of Polar Form}
	
	
	\subsection{Properties of Exponential Form}
	
	\section{Complex Conjugates}
	\begin{definition}[Complex Conjugate]
		The complex conjugate of $z \in \mathbb{C}$ is denoted $\bar{z}$.
		$$\bar{z} = x - iy$$
	\end{definition}
	Graphically, it is the reflection of $z$ across the real axis.
	\begin{center}
		\begin{tikzpicture}
			\begin{axis}[
				small,
				%			title={},
				xlabel={$\Re$},
				ylabel={$\Im$},
				xmin=0, xmax=10,
				ymin=-10, ymax=10,
				xtick={100},
				ytick={100},
				axis lines = middle,
				%			legend pos=north west,
				%			ymajorgrids=true,
				%			grid style=dashed,
				]
				
				\addplot[
				only marks,
				nodes near coords,
				point meta = explicit symbolic, 
				color=Black,
				mark=*,
				]
				coordinates {
					(7, 5) [$z = x + iy$]
					(7, -5) [$\bar{z} = x - iy$]
				};
				%			\legend{$z = x + iy$}
			\end{axis}
		\end{tikzpicture}
	\end{center}
	It is then easy to see
	\begin{align*}
		\operatorname{Re}(z) &= \frac{z + \bar{z}}{2} & \operatorname{Im}(z) &= \frac{z - \bar{z}}{2i} & \abs{z}^2 = z \bar{z}
	\end{align*}
	As $\operatorname{Re}(z) = x = r \cos(\theta)$ and $\operatorname{Im}(z) = y = r \sin(\theta)$ and using \cref{Exponential Form of z}, we can obtain the complex forms of sine and cosine: 
	\begin{align*}
		\cos(\theta) &= \frac{e^{i \theta} + e^{-i \theta}}{2} 
			&\sin(\theta) &= \frac{e^{i \theta} - e^{-i \theta}}{2i}
	\end{align*}
	
	\chapter{Conformal Mapping}
	
	
	\part{Ordinary Differential Equations}
	
	\part{Nonlinear Dynamics}
	
	
	\part{Partial Differential Equations}
	
	\paragraph{Calculus of Variations}
	
	\part{Integral Equations}
	
	
	\part{Linear Algebra}
	
	\chapter{Markov Chains}
	
	
	\part{Tensors}
	
	
	\part{Riemann Geometry}
	
	
	\part{Abstract Algebra}
	
	\chapter{Groups}
	
	
	\chapter{Rings}
	
	\section{Ideals}
	
	\chapter{Integral Domains}
	
	\chapter{GCD Domains}
	
	\chapter{Unique Factorization Domains}
	
	\chapter{Principal Ideal Domains}
	
	\chapter{Fields}
	
	
	\part{Galois Theory}
	
	\paragraph{Lie Algebra}
	
	\part{C-Star Algebra}
	
	\part{Set Theory}
	
	\part{Model Theory}
	
	\part{Statistics}
	\part{Tips and Tricks}
	
	\chapter{Integration Techniques}
	
	\section{DI Method (Integration Table)}
	
	\section{Feynman Integration}
	
	\backmatter
	\part{Index}
	
	\part{Bibliography}
	\bibliographystyle{unsrt}
	\typeout{}
	\bibliography{Bibliography}
	

\end{document}
