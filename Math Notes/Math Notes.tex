\documentclass[12pt, english]{book}
\usepackage[letterpaper, portrait, margin=2.5cm]{geometry} %Set paper type, format, margin
\usepackage{%abstract, 
	amsmath, amssymb, array, babel, booktabs, caption, censor, fancyhdr, flafter, float, gensymb, hyperref, inputenc, indentfirst, mathtools, MnSymbol, multicol, multirow, longtable, lscape, ltablex, polynom, pgfplots, placeins, rotating, scalerel, setspace, stackengine, tabularx, threeparttable, titlesec, titling, wasysym, xcolor}
\usepackage{cleveref}

%\renewcommand{\abstractname}{}    % clear the title
%\renewcommand{\absnamepos}{empty} % originally center 
\setlength{\parindent}{0em} %Set indent on paragraph
\setlength{\parskip}{0.5em} %Set spaces between paragraphs

\DeclarePairedDelimiter\abs{\lvert}{\rvert}%
\DeclarePairedDelimiter\norm{\lVert}{\rVert}%

\newcommand\showdiv[1]{\overline{\smash{\hstretch{.5}{)}\mkern-3.2mu\hstretch{.5}{)}}#1}}
\newcommand\ph[1]{\textcolor{white}{#1}}
\setstackgap{S}{1.5pt}

% Swap the definition of \abs* and \norm*, so that \abs
% and \norm resizes the size of the brackets, and the 
% starred version does not.


\begin{document}
	\title{The Book of Math (Notes)}
	\author{Kevin Kuo}
%	\renewcommand{\abstractname}{\vspace{-\baselineskip}}
	
	\pagestyle{fancy}
	\fancyhead{} % clear all header fields
	\fancyhead[LO]{  }
	\fancyhead[CO]{  }
	\fancyhead[RO]{  }
	\renewcommand{\headrulewidth}{0pt}
	
	\fancyfoot{} % clear all footer fields
	\fancyfoot[LO]{}
	\fancyfoot[CO]{\thepage}
	\fancyfoot[RO]{}
	\renewcommand{\footrulewidth}{0pt}
	\vspace{0cm}	
	
	\frontmatter
	
	\maketitle
	
	\newpage
	\section*{Forward and Disclaimer}
	These are math notes made by a student (with a physics major and math minor) based off text books. It may contain misconceptions and misinterpretations, thus should not be viewed in the same light of a text book. Use at your own risk and mental sanity.
	
	\section*{Symbols}
	\begin{tabularx}{\textwidth}{ l c p{0.6\linewidth}}
		\multicolumn{3}{l}{\textbf{{\large Logic}}} \\ [10pt]
		\hline
		\textbf{Name} & \textbf{Symbol} & \textbf{Comment} \\
		\hline
		Exists 					& $\exists$ 		& There exists at least one\\
		For all 				& $\forall$ 		& \\
		Not exists 				& $\nexists$ 		& There does not exist\\ 
		Exists one				& $\exists!$ 		& There only exists one and only one \\
		And 					& $\land$			& \\
		Or						& $\lor$			& Inclusive or \\
		Not 					& $\neg$			& \\
		Logically implies 		& $\implies$ 		& If \\
		Logically implied by 	& $\Longleftarrow$ 	& Only if \\  
		Logically equivalent 	& $\iff$ 			& If and only if \\
		Implies 				& $\longrightarrow$	& \\
		Implied by 				& $\longleftarrow$ 	& \\  
		Double Implication 		& $\longleftrightarrow$	& \\
		\hline	
		
		& & \\
		\multicolumn{3}{l}{\textbf{{\large Set Notation}}} \\ [10pt]
		\hline
		\textbf{Name} & \textbf{Symbol} & \textbf{Comment} \\
		\hline
 		Empty Set 				& $\emptyset$ 		& The set that is empty \\
 		Natural Numbers 		& $\mathbb{N}$		& Set of natural numbers not containing 0, equivalent to the set of positive integers \\
 		Integers 				& $\mathbb{Z}$		& Set of integers \\
 		Rational Numbers 		& $\mathbb{Q}$		& \\
 		Algebraic Numbers		& $\mathbb{A}$		& \\
 		Real Numbers 			& $\mathbb{R}$		& \\
 		Complex Numbers 		& $\mathbb{C}$		& \\
 		
 		In 						& $\in$ 			& \\
 		Not in 					& $\nin$			& \\
 		Owns 					& $\ni$				& Has an element \\
 		
 		Proper Subset 			& $\subset$			& Subset that is not itself \\
 		Subset 					& $\subseteq$		& \\
 		Superset 				& $\supset$ 		& Superset that is not itself\\
 		Proper Superset 		& $\supseteq$		& \\
 		Power set				& $\wp$				& \\
 		Union 					& $\cup$			& \\
 		Intersection			& $\cap$			& \\
 		Difference				& $\setminus$		& \\
 		\hline
 		
 		& & \\
 		\multicolumn{3}{l}{\textbf{{\large Relationships}}} \\ [10pt]
 		\hline
 		\textbf{Name} & \textbf{Symbol} & \textbf{Comment} \\
 		\hline
 		Defined 				& $\doteq$ 			& \\
 		Approximate 			& $\approx$			& \\
 		Equivalent				& $\equiv$	 		& Isomorphic (Group Theory) \\
 		Congruent 				& $\cong$			& Homomorphic (Group Theory) \\
 		Proportional 			& $\propto$			& \\
 		\hline
 		
 		& & \\
 		\multicolumn{3}{l}{\textbf{{\large Operators}}} \\ [10pt]
 		\hline
 		\textbf{Name} & \textbf{Symbol} & \textbf{Comment} \\
 		\hline
 		& $\oplus$ & \\
 		& $\otimes$ & \\
 		& $\odot$ & \\
 		& $\circ$ & Convolution \\
 		Dagger& $\dagger$ & Complex conjugate transpose of a matrix \\
 		\hline
 		
 		& & \\
 		\multicolumn{3}{l}{\textbf{{\large Arrows}}} \\ [10pt]
 		\hline
 		\textbf{Name} & \textbf{Symbol} & \textbf{Comment} \\
 		\hline
 		Maps to 				& $\mapsto$			& \\
 		\hline
 		
 		& & \\
 		\multicolumn{3}{l}{\textbf{{\large Hebrew}}} \\ [10pt]
 		\hline
 		\textbf{Name} & \textbf{Symbol} & \textbf{Comment} \\
 		\hline
 		Aleph					& $\aleph$			& Carnality of infinite sets that can be well ordered \\
 		\hline
 		
 		& & \\
 		\multicolumn{3}{l}{\textbf{{\large Other}}} \\ [10pt]
 		\hline
 		\textbf{Name} & \textbf{Symbol} & \textbf{Comment} \\
 		\hline
 		Real part 				& $\Re$				& Real part of a number \\
 		Imaginary part 			& $\Im$				& Imaginary part of a number \\
 		\hline
	\end{tabularx}

	\newpage
	\section*{Book Constitution}
	\subsection*{Intents and Purpose}
	The goal of this book is to organize mathematical knowledge of topics related to the study of physics or the author's interest. It is meant to be used as a source of for future reference, not as a textbook for students new to the topics. It is a notebook of a student, thus should be treated as one and not as a textbook. At most, it could be used as a study guide along side a textbook. Definitely not as the main source for acquiring knowledge. 
	
	\subsection*{Layout and Organization}
	The book is split into parts each containing a field of study mathematics, or a topic large enough to justify giving it its own part. Each part contains chapters that focuses on a particular topic required to understand the field, with sections dedicated to describing a particular knowledge required for the topic. 
	
	As axioms, definitions, theorems, corollary, and proofs are integral and abundant to the study of mathematics, each will have a unique style.  
	
	\section*{}
	\tableofcontents
	
	\mainmatter
	\part{Logic}
	
	\chapter{Proofs}
	
	
	\part{Numbers}
	
	\chapter{Natural $\mathbb{N}$}
	
	\chapter{Integers $\mathbb{Z}$}
	
	\chapter{Rationals $\mathbb{Q}$}
	
	\chapter{Constructible}
	
	\chapter{Algebraic $\mathbb{A}$}
	
	\chapter{Reals $\mathbb{R}$}
	
	\chapter{Complex $\mathbb{C}$}
	

	
	\part{Real Analysis}
	Books Used: 
	\begin{itemize}
		\item[1.] Kenneth A. Ross - Elementary Analysis (2nd Ed.) \cite{Ross.K-Elementary-Analysis-2013}
	\end{itemize}
	
	\part{Complex Analysis}
	Books Used: 
	\begin{itemize}
		\item[1.] Brown and Churchill - Complex Variables and Applications \cite{Brown.J;Churchill.R-Complex-Variables-2014}
	\end{itemize}

	
	
	\chapter{Conformal Mapping}
	
	
	\part{Ordinary Differential Equations}
	
	\part{Nonlinear Dynamics}
	
	
	\part{Partial Differential Equations}
	
	\paragraph{Calculus of Variations}
	
	\part{Integral Equations}
	
	
	\part{Linear Algebra}
	
	\chapter{Markov Chains}
	
	
	\part{Tensors}
	
	
	\part{Riemann Geometry}
	
	
	\part{Abstract Algebra}
	
	\chapter{Groups}
	
	
	\chapter{Rings}
	
	\section{Ideals}
	
	\chapter{Integral Domains}
	
	\chapter{GCD Domains}
	
	\chapter{Unique Factorization Domains}
	
	\chapter{Principal Ideal Domains}
	
	\chapter{Fields}
	
	
	\part{Galois Theory}
	
	\paragraph{Lie Algebra}
	
	\part{C-Star Algebra}
	
	\part{Set Theory}
	
	\part{Model Theory}
	
	\part{Statistics}
	\part{Tips and Tricks}
	
	\chapter{Integration Techniques}
	
	\section{DI Method (Integration Table)}
	
	\section{Feynman Integration}
	
	\backmatter
	\part{Index}
	
	\part{Bibliography}
	\bibliographystyle{unsrt}
	\typeout{}
	\bibliography{Bibliography}
	

\end{document}
