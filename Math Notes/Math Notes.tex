\documentclass[12pt, english]{book}
\usepackage[letterpaper, portrait, margin=2cm]{geometry} %Set paper type, format, margin
\usepackage{%abstract, 
	amsmath, amssymb, array, babel, booktabs, caption, censor, fancyhdr, flafter, float, gensymb, inputenc, indentfirst, mathtools, MnSymbol, multicol, multirow, lscape, polynom, pgfplots, placeins, rotating, scalerel, setspace, stackengine, tabularx, threeparttable, titlesec, titling, wasysym, xcolor}
%\renewcommand{\abstractname}{}    % clear the title
%\renewcommand{\absnamepos}{empty} % originally center 
\setlength{\parindent}{0em} %Set indent on paragraph
\setlength{\parskip}{0.5em} %Set spaces between paragraphs

\DeclarePairedDelimiter\abs{\lvert}{\rvert}%
\DeclarePairedDelimiter\norm{\lVert}{\rVert}%

\newcommand\showdiv[1]{\overline{\smash{\hstretch{.5}{)}\mkern-3.2mu\hstretch{.5}{)}}#1}}
\newcommand\ph[1]{\textcolor{white}{#1}}
\setstackgap{S}{1.5pt}

% Swap the definition of \abs* and \norm*, so that \abs
% and \norm resizes the size of the brackets, and the 
% starred version does not.


\begin{document}
	\title{The Book of Math (Notes)}
	\author{Kevin Kuo}
%	\renewcommand{\abstractname}{\vspace{-\baselineskip}}
	
	\pagestyle{fancy}
	\fancyhead{} % clear all header fields
	\fancyhead[LO]{  }
	\fancyhead[CO]{  }
	\fancyhead[RO]{  }
	\renewcommand{\headrulewidth}{0pt}
	
	\fancyfoot{} % clear all footer fields
	\fancyfoot[LO]{}
	\fancyfoot[CO]{\thepage}
	\fancyfoot[RO]{}
	\renewcommand{\footrulewidth}{0pt}
	\vspace{0cm}	
	
	\frontmatter
	
	\maketitle
	
	\newpage
	\section*{Forward and Disclaimer}
	These are math notes made by a student (with a physics major and math minor) based off text books. It may contain misconceptions and misinterpretations, thus should not be viewed in the same light of a text book. Use at your own risk and mental sanity.
	
	\section*{Symbols}
	
	
	\tableofcontents{}
	
	\mainmatter

	\part{Logic}

	
	\part{Real Analysis}
	
	
	\part{Complex Analysis}
	
	
	\chapter{Conformal Mapping}
	
	
	\part{Differential Equations}
	
	
	\part{Partial Differential Equations}
	
	
	\part{Linear Algebra}
	
	\chapter{Markov Chains}
	
	
	\part{Tensors}
	
	
	\part{Riemann Geometry}
	
	
	\part{Group Theory}
	
	
	\part{Galois Theory}
	
	
	

\end{document}
